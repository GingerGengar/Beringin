%Type of Document
\documentclass[class=report, 12pt, crop=false]{standalone}

%Load Pre-ambles
\usepackage{../Environment/Packages}
\usepackage{../Environment/Conventions}
\usepackage{../Environment/Hyahoos}

\begin{document}

\begin{center}
\chapter{Gauss Law}
\begin{comment}
\end{comment}
%Seperator
%Seperator
%Seperator
%Seperator
%Seperator
\stdsection{Gravitational Fields}
\begin{comment}
\end{comment}
The divergence theorem for any field $\b{F}$ is denoted as $\dst{\iiint \n \d F \,dV =  \oiint {F \d \b{n}}\,ds }$, wherein $div[F] = \n \d F$
\\For a spherical mass $m_1$, the gravitational field generated by the mass $m_1$ at any point: $\dst{\b{F_{field}} = -\f{Gm_1}{r^2}\h{r}}$, wherein $\h{r}$ represents unit vector with direction from mass to object. 
$$\dst{\h{r} = \frac{1}{\sqrt{x^2 + y^2 + z^2}}\begin{pmatrix} x\\y\\z \end{pmatrix}}$$
$$\h{r} = \begin{pmatrix} x(x^2+y^2+z^2)^{-\f{1}{2}}\\y(x^2+y^2+z^2)^{-\f{1}{2}} \\z(x^2+y^2+z^2)^{-\f{1}{2}} \end{pmatrix}$$
Therefore, given that $\quad x \neq 0,\quad y \neq 0, \quad z \neq 0$
$$\b{F_f} = -Gm_1\begin{pmatrix} x(x^2+y^2+z^2)^{-\f{3}{2}}\\y(x^2+y^2+z^2)^{-\f{3}{2}} \\z(x^2+y^2+z^2)^{-\f{3}{2}} \end{pmatrix}$$
$$\n \d \b{F_f} =-Gm_1\n \d \l[\begin{pmatrix} x(x^2+y^2+z^2)^{-\f{3}{2}}\\y(x^2+y^2+z^2)^{-\f{3}{2}} \\z(x^2+y^2+z^2)^{-\f{3}{2}} \end{pmatrix}\r]$$
\\Let $\dst{k =\begin{pmatrix} f\\g\\h\end{pmatrix}= \begin{pmatrix} x(x^2+y^2+z^2)^{-\f{3}{2}}\\y(x^2+y^2+z^2)^{-\f{3}{2}} \\z(x^2+y^2+z^2)^{-\f{3}{2}} \end{pmatrix}}$
$$\n \d \dst{k  =f_x + g_y + h_z =   \n \d \begin{pmatrix} x(x^2+y^2+z^2)^{-\f{3}{2}}\\y(x^2+y^2+z^2)^{-\f{3}{2}} \\z(x^2+y^2+z^2)^{-\f{3}{2}} \end{pmatrix}}$$
%Partial for x
$$\f{\p{f}}{\p{x}}= \f{\p}{\p{x}}\l[x(x^2+y^2+z^2)^{-\f{3}{2}}\r]$$
$$\f{\p{f}}{\p{x}}=-3x^2(x^2 +y ^2 + z^2)^{-\f{5}{2}}+(x^2 + y^2 + z^2)^{-\f{3}{2}}$$
%Partial for y
$$\f{\p{g}}{\p{y}}= \f{\p}{\p{y}}\l[y(x^2+y^2+z^2)^{-\f{3}{2}}\r]$$
$$\f{\p{g}}{\p{y}}=-3y^2(x^2 +y ^2 + z^2)^{-\f{5}{2}}+(x^2 + y^2 + z^2)^{-\f{3}{2}}$$
%Partual for z
$$\f{\p{f}}{\p{z}}= \f{\p}{\p{z}}\l[z(x^2+y^2+z^2)^{-\f{3}{2}}\r]$$
$$\f{\p{f}}{\p{z}}=-3z^2(x^2 +y ^2 + z^2)^{-\f{5}{2}}+(x^2 + y^2 + z^2)^{-\f{3}{2}}$$
$$\n \d k= \f{\p{f}}{\p{x}} + \f{\p{g}}{\p{y}} + \f{\p{h}}{\p{z}}$$
$$\n \d k= -3(x^2 + y^2 + z^2)(x^2 +y ^2 + z^2)^{-\f{5}{2}}+3(x^2 + y^2 + z^2)^{-\f{3}{2}}$$
$$\n \d k= -3(x^2 +y ^2 + z^2)^{-\f{3}{2}}+3(x^2 + y^2 + z^2)^{-\f{3}{2}}$$
$$\n \d k= 0$$
$$\n \d F_f = -Gm_1 \times 0$$
$$\n \d F_f = 0$$
\\Therefore, for any point outside of the sphere, the divergence of the gravitational field is nonexistent.
Consider a region R which is the space containing empty space and a mass sphere. Divergence in region R is the addition of divergence of free space and divergence of mass sphere. Since $div(F_f) = 0$ outside of the mass sphere as shown in previous example, divergence in R is only divergence of mass sphere. 
$$div(F_{R}) = div(F_{sphere}) + div(F_{free space})$$
\\By previous example, $div(F_{free space}) = 0$
$$div(F_{R}) = div(F_{sphere})$$
\\Consider a point mass of mass $M$ of uniform density with radius $R$:
$$\b{f_f} = \f{GM}{r^2}\h{r}$$
$$\dst{\iiint \n \d \b{f_f} \,dV =  \oiint {\b{f_f} \d \b{n}}\,ds }$$
\\By considering a surface to be enveloping the mass sphere,
$$\dst{\iiint \n \d \b{f_f} \,dV =  4 \pi R^2\b{f_f} }$$
\\For all points with radius $r = R$ away from the center of the mass sphere, $\dst{\b{f_f} = \f{GM}{r^2}\h{r}}$
$$\iiint \n \d \b{f_f} \,dV =  4 \pi R^2\f{GM}{R^2}$$
$$\iiint \n \d \b{f_f} \,dV =  4 \pi GM$$
\\By considering the predetermined property of this mass sphere to have uniform density:
$$(\n \d \b{f_f}) V =  4 \pi GM$$
$$(\n \d \b{f_f}) \lim_{V\to\ 0} [V] =  4 \pi G \lim_{M\to\ 0} [M]$$
$$(\n \d \b{f_f})dV =  4 \pi G dM$$
$$\n \d \b{f_f}=  4 \pi G \f{dM}{dV}$$
\\wherein $\rho$ is density,
$$\n \d \b{f_f}=  4 \pi G \rho$$
\\By considering the graviational field of any arbitrary object to be the summation of the gravitational field of small components of the object,
$$\b{F} = \b{f_1} + \b{f_2} + \b{f_3} + \dots  \b{f_i}$$
$$\b{F} = \sum_{n = 1}^{i} \l[\b{f_n} \r]$$
\\$\n \d$ could be considered as a linear transformation because $\n \d (\b{a} + \b{b}) = (\n \d \b{a}) + (\n \d \b{b})$ and also $\n \d (c\b{a}) =c  (\n \d \b{a})$. Therefore,
$$\n \d \b{F} = \sum_{n = 1}^{i} \l[\n \d \b{f_n} \r]$$
$$\n \d \b{F} = \sum_{n = 1}^{i} \l[4\pi G \rho_n \r]$$
$$\iiint_R \n \d \b{F} \,dV =\lim_{{\Delta V_n}\to\ 0}\l[ \sum_{n = 1}^{\infty} \l[4\pi G \rho_n\Delta V_n \r]\r]$$
$$\iiint_R \n \d \b{F} \,dV =4\pi G\lim_{{\Delta V_n}\to\ 0}\l[ \sum_{n = 1}^{\infty} \l[ \rho_n\Delta V_n \r]\r]$$
$$\rho_n \Delta V_n =\l(\f{dM_n}{dV_n}\r)dV_n = dM_n$$
$$\lim_{{\Delta V_n}\to\ 0}\l[ \sum_{n = 1}^{\infty} \l[ \rho_n\Delta V_n \r]\r] = \int_{0}^{m_{e}} dM = m_{e}$$
$$\iiint_R \n \d \b{F} \,dV =4\pi Gm_e$$
\\wherein $m_e$ represents the mass enclosed by the surface. By reiterating The Divergence Theorem,
$$\dst{\iiint \n \d \b{F} \,dV =  \oiint {\b{F} \d \b{n}}\,ds }$$
$$4\pi Gm_e = \oiint {\b{F} \d \b{n} \, ds }$$
\\For the special case wherein $\rho = f(r)$, the gravitational field would be perpendicular to an imaginary spherical surface at radius $r$ away from the center of the mass sphere. Therefore,
$$\b{F} \d \b{n} = |\b{F}|$$
$$\oiint {\b{F} \d \b{n}}\,ds = |\b{F}| \times 4 \pi r^2$$
$$4 \pi G m_e = |\b{F}| \times 4 \pi r^2$$
$$ |\b{F}| = \f{G m_e}{r^2} $$
\\The $|\b{F}|$ represents gravitational field produced by a spherical mass with varying radial density, $\rho = f(r)$. $m_e$ represents enclosed mass, which in this case is the entirety of the mass sphere. Since the force of gravity experienced by an object with mass $m_2$ is $F_{force} = \b{F_{field}} \times m_2$,
$$F_{force} = \f{Gm_em_2}{r^2} = \f{Gm_e}{r^2}\int_{0}^{R} f(r) dr$$
\\wherein $R$ is the radius of the mass sphere of varying radial density.
\end{center}

\end{document}
