%Type of Document
\documentclass[class=report, 12pt, crop=false]{standalone}

%Load Pre-ambles
\usepackage{../Environment/Packages}
\usepackage{../Environment/Conventions}
\usepackage{../Environment/Hyahoos}

\begin{document}

\begin{center}
\chapter{Binary Object Systems}
\begin{comment}
\end{comment}
%Seperator
%Seperator
%Seperator
%Seperator
%Seperator
\stdsection{Orbit Trajectory}
\begin{comment}
\end{comment}
For the scenario wherein an extraterrestial object of mass $m$ approaches a planet of mass $m_1$, energy within the system of the planet and the object remains conserved because there is no external force acting on the system. Therefore, the summation of energy,$\sum E$ can be considered a constant in this scenario.
$$\sum E = E_k + E_p$$
$$E_k =\frac{1}{2}  m |v|^2$$
By decomposing both radial and tangential components of velocity in polar coordinates:
$$|v|^2 = (\dot{\theta}r)^2 + \dot{r}^2$$
Therefore, 
$$E_k = \frac{1}{2}m[(\dot{\theta}r)^2 + \dot{r}^2] = \frac{1}{2}m(\dot{\theta}r)^2 + \frac{1}{2}m\dot{r}^2 = \frac{1}{2}m\dot{\theta}^2 r^2 + \frac{1}{2}m\dot{r}^2$$
Since $E_p$ represents gravitational potential energy the object has:
$$E_p = -\frac{Gmm_1}{r}$$
Let $\alpha = -Gmm_1$
$$E_p = \frac{\alpha}{r}$$
By substituting the expressions for kinetic energy and potential energy back into the equation for total energy:
$$\sum E = E_k + E_p = \frac{1}{2}m\dot{\theta}^2 r^2 + \frac{1}{2}m\dot{r}^2 + \frac{\alpha}{r}$$
The expression above is in the form of a differential equation involving two variables, $\theta$ and $r$ as a derivative in respect to time. Let $L$ represent angular momentum, and $I$ represent moment inertia:
$$L = I \dot{\theta} \quad,\quad I = mr^2$$
Substituting for the moment of inertia for a single particle,
$$L =  mr^2\dot{\theta}$$
making $\dst{\dot{\theta}}$ subject of the equation,
$$\dot{\theta} = \frac{L}{mr^2}$$
The gravitational forces are always parallel to the radial vector of the object to the planet. Therefore, no net torque is produced on the passing object and angular momentum is conserved for the object. $L$ could be treated as a constant in equation 2. Since $L$ is a constant, equation 2 provides a method to reduce the dual variable problem in equation 1 to a single variable problem in terms of $r$. By substituting $\dot{\theta}$ from equation 2 to equation 1:
$$\sum E = \frac{1}{2}m\dot{\theta}^2r^2 + \frac{1}{2}m\dot{r}^2 +\frac{\alpha}{r}$$
Using a change in notation for convenience, let $\sum E = E$,
$$E = \frac{1}{2}m\l(\frac{L}{mr^2}\r)^2r^2 + \frac{1}{2}m\dot{r}^2 +\frac{\alpha}{r} = \frac{1}{2}mr^2\frac{L^2}{(mr^2)^2} + \frac{1}{2}m\dot{r}^2 +\frac{\alpha}{r} = \frac{L^2}{2mr^2} + \frac{1}{2}m\dot{r}^2 +\frac{\alpha}{r}$$
By manipulating the equation to make $\dot{r}$ the subject of the equation:
$$\frac{1}{2}m\dot{r}^2 = E - \frac{L^2}{2mr^2} - \frac{\alpha}{r}$$
$$\dot{r}^2 = \frac{2E}{m}-\frac{L^2}{m^2r^2}-\frac{2\alpha}{mr}$$
$$\l(\frac{dr}{dt}\r)^2 =\frac{2E}{m}-\frac{L^2}{m^2r^2}-\frac{2\alpha}{mr}$$
The expression above involve $r$ as a derivative in respect to time. Manipulating on this expression further would produce $r$ as a function of time. Since the shape of the path traced by the object getting near to the planet is the subject in question, the derivative of $r$ with respect to time must be converted to the derivative of $r$ in respect to $\theta$.
$$\frac{dr}{dt} = \frac{dr}{d\theta}\frac{d\theta}{dt}$$
Fortunately, equation 2 reduces $\dot{\theta}$ or $\frac{d\theta}{dt}$ into an expression purely in terms of known constants and $r$.
$$\frac{dr}{dt} = \frac{dr}{d\theta}\dot{\theta} = \frac{dr}{d\theta}\frac{L}{mr^2}$$
$$\l(\frac{dr}{dt}\r)^2 = \l(\frac{dr}{d\theta}\frac{L}{mr^2}\r)^2 = \l(\frac{dr}{d\theta}\frac{1}{r^2}\r)^2\frac{L^2}{m^2}$$
By substituting the term $\dot{r}$ from equation 4 into equation 3:
$$\l(\frac{dr}{d\theta}\frac{1}{r^2}\r)^2\frac{L^2}{m^2}=\frac{2E}{m}-\frac{L^2}{m^2r^2}-\frac{2\alpha}{mr}$$
$$\l(\frac{dr}{d\theta}\frac{1}{r^2}\r)^2 = \frac{2E}{m}\frac{m^2}{L^2}-\frac{L^2}{m^2r^2}\frac{m^2}{L^2}-\frac{2\alpha}{mr}\frac{m^2}{L^2} = \frac{2Em}{L^2}-\frac{1}{r^2}-\frac{2\alpha m}{L^2r}$$
On the right hand side, there are lots of terms containing $\displaystyle\frac{1}{r}$ as well as $\displaystyle\frac{1}{r^2}$. For the reason that the left hand side also contain the term $\displaystyle\frac{dr}{d\theta}\frac{1}{r^2}$, let $\displaystyle u = \frac{1}{r}$
$$u = \frac{1}{r}$$
$$\frac{du}{d\theta} = -\frac{1}{r^2}\frac{dr}{d\theta}$$
$$-\frac{du}{d\theta} = \frac{1}{r^2}\frac{dr}{d\theta}$$
By substituting the term $\dst{\frac{dr}{d\theta}\frac{1}{r^2}}$ into the main equation,
$$\l(-\frac{du}{d\theta}\r)^2=\frac{2Em}{L^2}-\frac{1}{r^2}-\frac{2\alpha m}{L^2r}$$
$$\l(\frac{du}{d\theta}\r)^2=\frac{2Em}{L^2}-u^2-\frac{2\alpha m}{L^2}u = \frac{2Em}{L^2}-\l(u^2+\frac{2\alpha m}{L^2}u\r)$$
The right hand side takes the form of a quadratic in terms of $u$. The completing the square method should be performed on the right hand side of the expression to simplify the expression.
$$\l(\frac{du}{d\theta}\r)^2=\frac{2Em}{L^2}-\l[u^2+\frac{2\alpha m}{L^2}u + \l(\frac{\alpha m}{L^2}\r)^2 - \l(\frac{\alpha m}{L^2}\r)^2\r] = \frac{2Em}{L^2}+ \l(\frac{\alpha m}{L^2}\r)^2-\l[u^2+\frac{2\alpha m}{L^2}u + \l(\frac{\alpha m}{L^2}\r)^2\r]$$
$$\l(\frac{du}{d\theta}\r)^2=\frac{2EmL^2}{L^4}+ \frac{\alpha^2 m^2}{L^4}-\l(u+\frac{\alpha m}{L^2}\r)^2 = \frac{2EmL^2+\alpha^2 m^2}{L^4}-\l(u+\frac{\alpha m}{L^2}\r)^2$$
Let $\displaystyle \beta^2 = \frac{2EmL^2+\alpha^2 m^2}{L^4}$
$$\l(\frac{du}{d\theta}\r)^2=\beta^2-\l(u+\frac{\alpha m}{L^2}\r)^2$$
It can be noticed that within the right hand side, the completing square form contains the variable $u$ in arithmetic addition with some known constants. In order to "get rid" of these known constants to operate the equation further, let $\displaystyle \gamma = u +\frac{\alpha m}{L^2}$
$$\frac{d\gamma}{du} = \frac{d}{du}\l(u+\frac{\alpha m}{L^2}\r)$$
$$\frac{d\gamma}{du} = 1$$
$$d\gamma = du$$
$$\frac{d\gamma}{d\theta}=\frac{du}{d\theta}$$
By substituting $\gamma$ to the original expression:
$$\l(\frac{d\gamma}{d\theta}\r)^2 = \beta^2-(\gamma)^2$$
By solving for $\gamma$ as a function of $\theta$
$$\frac{d\gamma}{d\theta} = \sqrt{\beta^2-(\gamma)^2}$$
$$\frac{d\theta}{d\gamma} = \frac{1}{\sqrt{\beta^2-(\gamma)^2}}$$
$$d\theta= \frac{1}{\sqrt{\beta^2-(\gamma)^2}}d\gamma$$
$$\int{d\theta} = \int{\frac{1}{\sqrt{\beta^2-(\gamma)^2}}d\gamma}$$
By substitution of a inverse trigonometric derivative, let $\displaystyle p = \arcsin\l(\frac{\gamma}{\beta}\r)$
$$\frac{dp}{d\gamma} = \frac{1}{\sqrt{1-\frac{\gamma^2}{\beta^2}}}\times\frac{1}{\beta} = \frac{1}{\sqrt{\frac{\beta^2-\gamma^2}{\beta^2}}}\times\frac{1}\beta$$
$$\frac{dp}{d\gamma}=\frac{\beta}{\sqrt{\beta^2-\gamma^2}}\times\frac{1}{\beta}$$
$$d\gamma=\sqrt{\beta^2-\gamma^2}dp$$
By substuting$d\gamma$ into the original integral:
$$\theta = \int{\frac{1}{\sqrt{\beta^2-\gamma^2}}\sqrt{\beta^2-\gamma^2}dp}$$
$$\theta = p - \theta_0$$
$$\theta = \arcsin{\l(\frac{\gamma}{\beta}\r)}-\theta_0$$
$$\theta + \theta_0 = \arcsin{\l(\frac{\gamma}{\beta}\r)}$$
$$\sin{\l(\theta+\theta_0\r)} = \frac{\gamma}{\beta}$$
$$\beta\sin{\l(\theta+\theta_0\r)} =\gamma$$
$$\gamma = u + \frac{\alpha m}{L^2}$$
$$\beta\sin{\l(\theta+\theta_0\r)} =u + \frac{\alpha m}{L^2}$$
$$u = \frac{\beta L^2sin{\l(\theta+ \theta_0\r)}-\alpha m}{L^2}$$
$$u = \frac{1}{r}$$
$$\frac{1}{r}=\frac{\beta L^2sin{\l(\theta+ \theta_0\r)}-\alpha m}{L^2}$$
$$r = \frac{L^2}{\beta L^2sin{\l(\theta+ \theta_0\r)}-\alpha m}$$
$$\beta^2 = \frac{2EmL^2+\alpha^2 m^2}{L^4}$$
$$\beta = \sqrt{\frac{2EmL^2+\alpha^2 m^2}{L^4}}$$
$$\beta L^2 =\sqrt{2EmL^2+\alpha^2 m^2} $$
$$\beta L^2 =\alpha m\sqrt{\frac{2EL^2}{m\alpha^2}+1}$$
By substituting to the expression for $r$ as a function of $\theta$:
$$r = \frac{L^2}{\alpha m(\sqrt{\frac{2EL^2}{m\alpha^2}+1})sin{(\theta+ \theta_0)}-\alpha m}$$
Let choice of $\theta_0$ be such that $\sin{(\theta + \theta_0)} = \cos{(\theta)}$. In the real context, the variable $\theta_0$ provides a method to shift the point of reference rotationally. However, since the aim of the derivation is to investigate the matter of the orbit trajectory of a foreign object, the choice of $\theta_0$ does not affect the shape of the trajectory produced. Since we are free to choose any value of $\theta_0$, the choice of $\theta_0$ such that $\sin{(\theta + \theta_0)} = \cos{(\theta)}$ simplifies the problem. Let $\dst{\epsilon_0 = \sqrt{\frac{2EL^2}{m \alpha^2}+1}}$,
$$r = -\frac{L^2}{m\alpha(1-\epsilon_0\cos{(\theta)})}$$
Since $\dst{\frac{L^2}{m\alpha}}$ is a constant produced by the expressions of known constants, let $\dst k = -\frac{L^2}{m\alpha}$
$$r=\frac{k}{1-\epsilon_0\cos{(\theta)}}$$
For the values of $\epsilon_0$ to be $0$, the orbit trajectory of the object on the planet shall be a circle, for the values of $\epsilon_0$ to be $0<\epsilon_0<1$, the orbit trajectory shall be an ellipse. For values of $\epsilon_0>1$, the orbit trajectory shall be a hyperbola. By comparing the expression above to the expression for conic sections in polar coordinates, it can be inferred that $\epsilon_0$ takes the value of the eccentricity of the path the orbit takes.
%Seperator
%Seperator
%Seperator
%Seperator
%Seperator
\section{Time Dependent Quantities}
\begin{comment}
\end{comment}
The previous working shows the trajectory of a binary system. The purpose of this working is to establish a relationship between time and radius $r$. All variables and symbols written below rerpresent the same quantities as the previous working. A reiteration of the previous equation:
$$\dot{r}^2 = \frac{2E}{m}-\frac{L^2}{m^2r^2}-\frac{2\alpha}{mr}$$
$$\l(\f{dr}{dt}\r)^2  = \f{2Emr^2 - L^2 - 2 \a mr}{m^2 r^2}$$
\\By taking the sqare root and reciprocral of both sides,
$$\f{dt}{dr} = \f{mr}{\sqrt{2Emr^2 - 2 \a mr - L^2}}$$
\\Using completing the square method and rearranging the equations,
$$\f{dt}{dr} = \f{mr}{\sqrt{2Em\l[r^2 - \f{\a}{E}r + \l(\f{\a}{2E}\r)^2 - \l(\f{\a}{2E}\r)^2 -  \f{L^2}{2Em}\r]}}$$
$$\f{dt}{dr} = \sqrt{\f{m}{2E}}\l(\f{r}{\sqrt{r^2 - \f{\a}{E}r + \l(\f{\a}{2E}\r)^2 - \f{m\a ^ 2}{4 E^2 m} -  \f{2E L^2}{4 E^2 m}}}\r)$$
$$\f{dt}{dr} = \sqrt{\f{m}{2E}}\l(\f{r}{\sqrt{\l(r - \f{\a}{2E}\r)^2 -  \f{2E L^2 + m\a ^ 2}{4 E^2 m}}}\r)$$
\\Let
$$k_1 = \sqrt{\f{m}{2E}} \qquad k_2 = \f{\a}{2E} \qquad k_3 = \f{2E L^2 + m\a ^ 2}{4 E^2 m}$$
$$\f{dt}{dr} = k_1 \f{r}{\sqrt{(r - k_2)^2 - k_3}}$$
$$\f{dt}{dr} = k_1 \f{r - k_2}{\sqrt{(r - k_2)^2 - k_3}} + k_1 \f{k_2}{\sqrt{(r - k_2)^2 - k_3}}$$
$$\int dt= k_1 \int \f{r - k_2}{\sqrt{(r - k_2)^2 - k_3}} dr + k_1 \int \f{k_2}{\sqrt{(r - k_2)^2 - k_3}} dr$$
\\For the first integrand of the equation above, let $u = r - k_2$
$$\f{du}{dr} = 1$$
$$\int \f{r - k_2}{\sqrt{(r - k_2)^2 - k_3}} dr = \int \f{u}{\sqrt{u^2 - k_3}} du$$
\\Let $\gamma = \sqrt{u^2 - k_3}$
$$\f{d\gamma}{du} = \f{1}{2}\f{1}{\sqrt{u^2 - k_3}}(2u)$$
$$\int \f{r - k_2}{\sqrt{(r - k_2)^2 - k_3}} dr = \sqrt{u^2 - k_3} + C$$
$$\int \f{r - k_2}{\sqrt{(r - k_2)^2 - k_3}} dr = \sqrt{(r - k_2)^2 - k_3} + C$$
\\For the second integrand of the equation, let $u = r - k_2$
$$\f{du}{dr} = 1$$
$$\int \f{k_2}{\sqrt{(r - k_2)^2 - k_3}} = \int \f{k_2}{\sqrt{u^2 - k_3}} du$$
\\Using the method of hyperbolic substitution, let $u = \sqrt{k_3}\cosh(\t)$
$$u^2 = k_3\cosh^2\l(\t\r) = k_3\sinh^2(\t) + k_3$$
$$\int \f{k_2}{\sqrt{u^2 - k_3}} du = \int \f{k_2}{\sqrt{k_3\sinh^2(\t) + k_3 - k_3}}$$
$$\int \f{k_2}{\sqrt{u^2 - k_3}} du = \int \f{k_2}{\sqrt{k_3\sinh^2(\t)}}du = \int \f{k_2}{\sqrt{k_3}}\f{1}{\sinh(\t)}du$$
\\Finding and expression of the differential $du$ in terms of $\t$,
$$u = \sqrt{k_3}\cosh(\t)$$
$$\f{du}{d\t} = \sqrt{k_3}\sinh(\t)$$
$$du = \sqrt{k_3}\sinh(\t)d\t$$
\\Substituting the definition of the differential $du$,
$$\int \f{k_2}{\sqrt{u^2 - k_3}} du = \f{k_2}{\sqrt{k_3}} \int \f{1}{\sinh(\t)}\sqrt{k_3}\sinh(\t)d\t$$
$$\int \f{k_2}{\sqrt{u^2 - k_3}} du = k_2 \t + C$$
\\Expressing $\t$ in terms of $u$,
$$u = \sqrt{k_3}\cosh(\t)$$
$$\f{u}{\sqrt{k_3}} = \cosh(\t)$$
$$\t = arcosh\l(\f{u}{\sqrt{k_3}}\r)$$
\\Substituing the definition of $\t$
$$\int \f{k_2}{\sqrt{u^2 - k_3}} du = k_2 arcosh\l(\f{r - k_2}{\sqrt{k_3}}\r) + C$$
\\Therefore, the second integrand is:
$$\int \f{k_2}{\sqrt{(r - k_2)^2 - k_3}} = k_2 arcosh\l(\f{r - k_2}{\sqrt{k_3}}\r) + C$$
\\By placing the two integrands together, 
$$\int dt= k_1 \sqrt{(r - k_2)^2 - k_3} + k_1 k_2 arcosh\l(\f{r - k_2}{\sqrt{k_3}}\r) + C $$
\\By substituting the expressions for $k_1, k_2,$ and $k_3$:
$$t= \sqrt{\f{m}{2E}} \sqrt{(r - \f{\a}{2E})^2 - \f{2E L^2 + m\a ^ 2}{4 E^2 m}} + \sqrt{\f{m}{2E}} \f{\a}{2E} arcosh\l(\f{r - \f{\a}{2E}}{\sqrt{\f{2E L^2 + m\a ^ 2}{4 E^2 m}}}\r) + C $$
$$t= \f{\sqrt{m}}{(2E)^\f{3}{2}} \sqrt{(2Er - \a)^2 - \f{2E L^2 + m\a ^ 2}{m}} + \f{\a\sqrt{m}}{(2E)^\f{3}{2}}  arcosh\l(\f{\f{2Er}{2E} - \f{\a}{2E}}{\sqrt{\f{2E L^2 + m\a ^ 2}{4 E^2 m}}}\r) + C $$
$$t= \f{\sqrt{m}}{(2E)^\f{3}{2}} \sqrt{(2Er - \a)^2 - \f{2E L^2}{m} - \a^2} + \f{\a\sqrt{m}}{(2E)^\f{3}{2}}  arcosh\l(\f{2Er - \a}{2E} \times \sqrt{\f{4 E^2 m}{2E L^2 + m\a ^ 2}}\r) + C $$
$$t= \f{\sqrt{m}}{(2E)^\f{3}{2}} \sqrt{(2Er - \a)^2 - \f{2E L^2}{m} - \a^2} + \f{\a\sqrt{m}}{(2E)^\f{3}{2}}  arcosh\l[(2Er - \a)\sqrt{\f{m}{2E L^2 + m\a ^ 2}}  \r] + C $$
\end{center}

\end{document}
