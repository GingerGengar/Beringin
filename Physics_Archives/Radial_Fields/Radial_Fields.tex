%Type of Document
\documentclass[class=report, 12pt, crop=false]{standalone}

%Load Pre-ambles
\usepackage{../Environment/Packages}
\usepackage{../Environment/Conventions}
\usepackage{../Environment/Hyahoos}

\begin{document}

\begin{center}
\chapter{Radial Fields}
\begin{comment}
\end{comment}
%Seperator
%Seperator
%Seperator
%Seperator
%Seperator
\stdsection{Curl and Circulation}
\begin{comment}
\end{comment}
Let the general equation denoting force produced by a radial field be: $\dst{\b{F} = kr^n\h{r}}$ wherein k and n are arbitrary constants. Since $\dst{\h{r}}$ represents a unit vector in the radial direction, 
$$\h{r} = \f{1}{\sqrt{x^2 + y^2 + z^2}}\begin{pmatrix}x \\ y \\ z \end{pmatrix}$$
$$r = \l(x^2 +y ^2 + z^2\r)^\f{1}{2}$$
$$r^n = \l(x^2 +y ^2 + z^2\r)^\f{n}{2}$$
$$\b{F} = \f{k\l(x^2 +y ^2 + z^2\r)^\f{n}{2}}{\sqrt{x^2 + y^2 + z^2}}\begin{pmatrix} x\\y\\z \end{pmatrix}$$
$$\b{F} = k(x^2 + y^2 + z^2)^{\f{1}{2}(n-1)}\begin{pmatrix} x\\y\\z \end{pmatrix}$$
$$\b{F} = \begin{pmatrix} kx(x^2 + y^2 + z^2)^{\f{1}{2}(n-1)}\\ky(x^2 + y^2 + z^2)^{\f{1}{2}(n-1)}\\kz(x^2 + y^2 + z^2)^{\f{1}{2}(n-1)} \end{pmatrix}$$
\\Stokes theorem states: $\dst{\oint_R  \b{F} \cdot \b{dr} = \oiint_S \nabla \times \b{F} \cdot \h{n}dS}$ wgereub $R$ represents a closed path in $\mathbb{R}^3$ and $S$ represents the corresponding surface bounded by the closed path similarly in $\mathbb{R}^3$. In  conservative vector field, the circulation, $\dst{\oint_R  \b{F} \cdot \b{dr}} = 0$, therefore, the curl, $\nabla \times \b{F} = 0$.
\\Let $\dst{\b{F} = \begin{pmatrix} f\\g\\h \end{pmatrix}}$
\renewcommand\arraystretch{1.5}
$$\nabla \times \b{F} = \begin{pmatrix} \f{\p h}{\p y} - \f{\p g}{\p z} \\ \f{\p f}{\p z} - \f{\p h}{\p x} \\ \f{\p g}{\p x} - \f{\p f}{\p y}\end{pmatrix}$$
\renewcommand\arraystretch{1.0}
$$\f{\p h}{\p y} = kyz(n - 1)(x^2 + y^2 + z^2)^{\f{1}{2}(n-3)}$$
$$\f{\p g}{\p z} = kyz(n - 1)(x^2 + y^2 + z^2)^{\f{1}{2}(n-3)}$$
$$\f{\p h}{\p y} - \f{\p g}{\p z} = 0$$
$$\f{\p f}{\p z} = kxz(n - 1)(x^2 + y^2 + z^2)^{\f{1}{2}(n-3)}$$
$$\f{\p h}{\p x} = kxz(n - 1)(x^2 + y^2 + z^2)^{\f{1}{2}(n-3)}$$
$$\f{\p f}{\p z} - \f{\p h}{\p x} = 0$$
$$\f{\p g}{\p x} = kxy(n - 1)(x^2 + y^2 + z^2)^{\f{1}{2}(n-3)}$$
$$\f{\p f}{\p y} = kxy(n - 1)(x^2 + y^2 + z^2)^{\f{1}{2}(n-3)}$$
$$\f{\p g}{\p x} - \f{\p f}{\p y} = 0$$
Therefore, $\dst{\nabla\times \b{F} = \begin{pmatrix}0\\0\\0 \end{pmatrix}}$. Any radial vector field with the following expression $\b{F} = kr^n\h{r}$ is conservative. Since radial vector fields are considered as conservative force fields, a potential function must exist with the following conditions: 
$$f = \f{\p p}{\p x}\qquad g = \f{\p p}{\p y}\qquad h = \f{\p p}{\p z}$$
$$p = \int  kx(x^2 + y^2 + z^2)^{\f{1}{2}(n-1)} dx$$
$$p = \f{k}{n+1}(x^2+y^2+z^2)^{\f{1}{2}(n+1)} + C(y,z) + k_1$$
\\$C(y,z)$ represents functions strictly in terms of $y$ and $z$ and $k_1$ represents some arbitrary numerical integration constant.
$$\f{\p p}{\p y} = ky(x^2 + y^2 + z^2)^{\f{1}{2}(n-1)} + C'(y,z)$$
\\The potential function's partial derivative with respect to $y$ represents $y$ component of the force field. Therefore,
$$ky(x^2 + y^2 + z^2)^{\f{1}{2}(n-1)} + C'(y,z) = ky(x^2 + y^2 + z^2)^{\f{1}{2}(n-1)}$$
$$C'(y,z) = 0$$
$$\int dC(y,z) = 0\times\int dy$$
$$C(y,z) = C_2(z) + k_2$$
$$p = \f{k}{n+1}(x^2+y^2+z^2)^{\f{1}{2}(n+1)} + C_2(z) + k_1 + k_2$$
$$\f{\p p}{\p z} = kz(x^2 + y^2 + z^2)^{\f{1}{2}(n-1)} + C_2'(z) = kz(x^2 + y^2 + z^2)^{\f{1}{2}(n-1)}$$
$$C'_2(z) = 0$$
$$C_2(z) = k_3$$
$$p = \f{k}{n+1}(x^2+y^2+z^2)^{\f{1}{2}(n+1)} + k_1 + k_2 + k_3$$
By convention, the potential of a force field is defined as 0 infinitely away from the origin. Therefore, 
$$0 =p = \f{k}{n + 1}\lim_{(x,y,z)\to\infty} \l[(x^2+y^2+z^2)^{\f{1}{2}(n+1)}\r] + k_1 + k_2 + k_3$$
$$0 = k_1 + k_2 + k_3$$
\\Therefore, the potential function for any given radial force field is given as:
$$p = \f{k}{n+1}(x^2+y^2+z^2)^{\f{1}{2}(n+1)} $$
\end{center}

\end{document}
