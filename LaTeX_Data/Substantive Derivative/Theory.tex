%Seperator
%Seperator
%Seperator
%Seperator
\subsection{Eulerian Reference Frame}
\begin{comment}
\end{comment}
We need to introduce two reference frames to properly analyze continuum phenomena. 
The first reference frame is called the "Eulerian" reference frame where an observer is fixed in some time and space based on an inertial reference frame.
Imagine a river as our fluid and we have a floating probe that we put at some location in the river (maybe the center).
The probe is anchored to the bottom of that river and it just takes measurements (temperature, velocity, whatever you want) from its fixed location in space and in the river.
If we are the observer fixed to such a probe, then that corresponds to the eulerian reference frame.
\begin{figure}[H]\centering
\def\svgwidth{500px}
\import{Images/}{Eulerian_Reference_Frame.pdf_tex}
\caption{Illustration of an eulerian reference frame}
\label{eulerian reference frame fluid}
\end{figure}
Whatever quantity we are measuring such as temperature or velocity in some direction can be uniquely determined a function of position $x$, $y$, $z$ and time $t$ in the eulerian reference frame.
This just means that if we have a point in space and a certain time, that is enough information to determine whatever quantity we are interested in like temperature or velocity.


%Seperator
%Seperator
%Seperator
%Seperator
\subsection{Lagrangian Reference Frame}
\begin{comment}
\end{comment}
The second reference frame is called the "Lagrangian" reference frame where the observer is moving along with a group of fluid particles and keeps tracking of just those specific group of fluid particles.
Imagine the same river but this time the probe is freely floating and not anchored to the bottom of the river.
The probe follows the same fluid particles down the river and takes measurements (whatever quantity we want) of the same group of fluid particles.
If we are the observer fixed to the probe that is moving downstream along with the fluid, then that corresponds to the lagrangian reference frame.
\begin{figure}[H]\centering
\def\svgwidth{500px}
\import{Images/}{Lagrangian_Reference_Frame.pdf_tex}
\caption{Illustration of a lagrangian reference frame}
\label{lagrangian reference frame fluid}
\end{figure}
Since the lagrangian reference frame follows a package of particles travelling downstream, then whatever quantity we are interested in (temperature or velocity or whatever you want) is instead functions of its initial position and time. 


%Seperator
%Seperator
%Seperator
%Seperator
\subsection{Reconcilliation of Time Derivatives}
\begin{comment}
\end{comment}
Let's consider some quantity $\phi$ in some fluid. 
This quantity is arbitrary, so it can represent temperature, pressure, density, velocity in one direction or whatever we want. 
The concept of a substantive derivative is to relate the time derivative in the eulerian reference frame to the time derivative in the lagrangian reference frame.
In other words, the substantive derivative seeks to answer the question: "Given that we know what the time derivative for some quantity is based on an observer at a fixed location, can we tell what the time derivative is for the same quantity based on an observer that is moving along with the fluid?"
\begin{figure}[H]\centering
\def\svgwidth{500px}
\import{Images/}{Substantive_Derivative.pdf_tex}
\caption{Difference between the time derivative of the eulerian and lagrangian reference frame}
\label{eulerian-lagrangian reference frame derivative}
\end{figure}
We can express the arbitrary quantity $\phi$ in terms of the eulerian reference frame as functions of spatial position $x_{1}$, $x_{2}$, $x_{3}$ and time $t$ or we can express the same quantity $\phi$ in terms of its initial positions $x_{in,1}$, $x_{2,in}$, $x_{3,in}$ and time $t$.
\begin{equation}\phi = \phi_{E}(x_{1,},x_{2},x_{3},t) = \phi_{L}(x_{in,1},x_{in,2},x_{in,3},t) \label{Substantive Derivative Functional Representation}\end{equation}
wherein $\phi_{E}$ is a function that is used to describe $\phi$ in the eulerian reference frame and $\phi_{L}$ is a function that is used to describe $\phi$ in the lagrangian reference frame.
In general, $\phi_{E}$ is going to be different than $\phi_{L}$.
We are going to define a mapping $M_{el}$ that maps the current fluid at position $x_{1}$, $x_{2}$, $x_{3}$, to its initial position $x_{in,1}$, $x_{in,2}$, $x_{in,3}$.
So this mapping works in the following way,
\begin{equation}x_{1} = M_{el,1}(x_{in,1}, x_{in,2}, x_{in,3}, t) \quad,\quad x_{2} = M_{el,2}(x_{in,1}, x_{in,2}, x_{in,3}, t) \quad,\quad x_{3} = M_{el,3}(x_{in,1}, x_{in,2}, x_{in,3}, t) \label{Definition of Lagrangian to Euler Reference Frame Mapping}\end{equation}
As a shorthand with the tensor index notation,
$$x_{i} = M_{el,i}(x_{in,1}, x_{in,2}, x_{in,3}, t)$$
We are going to take the time derivative of equation $\ref{Substantive Derivative Functional Representation}$ to try to gain a relationship between the time derivative in the eulerian reference frame and the lagrangian reference frame.
However, the representation in equation $\ref{Substantive Derivative Functional Representation}$ has one part in initial position and another part in a fixed spatial position.
In order to reconcile these two representations, we need to express them as functions of the same variables.
Therefore, we need to apply the mapping $M_{el,i}$, to equation $\ref{Substantive Derivative Functional Representation}$ first before taking time derivative.
Applying the mapping to equation $\ref{Substantive Derivative Functional Representation}$,
$$\phi_{E}(M_{el,1}, M_{el,2}, M_{el,3}, t) = \phi_{L}(x_{in,1},x_{in,2},x_{in,3},t) $$
Note that we didn't write down the arguments to $M_{el,i}$ just because its tedious and long, but we have to remember that each $M_{el,i}$ are functions with inputs $x_{in,1}$, $x_{in,2}$, $x_{in,3}$, and $t$.
We are now going to take the time derivative of both sides of the equation above,
$$\phi_{L}(x_{in,1},x_{in,2},x_{in,3},t) = \phi_{E}(M_{el,1}, M_{el,2}, M_{el,3}, t)$$
$$\frac{d}{d t}\left[\phi_{L}(x_{in,1},x_{in,2},x_{in,3},t)\right] = \frac{d}{d t}\left[\phi_{E}(M_{el,1}, M_{el,2}, M_{el,3}, t)\right]$$
Now we need to understand that both left hand side and right hand side of our equation only depends on initial positions $x_{in,1}$, $x_{in,2}$, $x_{in,3}$, and time $t$.
The initial positions and the time variable are all independent variables.
Therefore, the left hand side of our equation is just going to be the partial derivative with respect to time,
$$\frac{d}{d t}\left[\phi_{L}(x_{in,1},x_{in,2},x_{in,3},t)\right] = \frac{\partial}{\partial t}\left[\phi_{L}(x_{in,1},x_{in,2},x_{in,3},t)\right]$$
Substituting for our left hand side,
\begin{equation}\frac{\partial}{\partial t}\left[\phi_{L}(x_{in,1},x_{in,2},x_{in,3},t)\right] = \frac{d}{d t}\left[\phi_{E}(M_{el,1}, M_{el,2}, M_{el,3}, t)\right] \label{Almost Done Substantive Derivative primitive 1}\end{equation}
Now our right hand side can be expanded using chain rule in partial derivatives,
$$\frac{d}{d t}\left[\phi_{E}(M_{el,1}, M_{el,2}, M_{el,3}, t)\right] = \frac{\partial \phi_{E}}{\partial t} + \frac{\partial \phi_{E}}{\partial M_{el,1}}\frac{\partial M_{el,1}}{\partial t} + \frac{\partial \phi_{E}}{\partial M_{el,2}}\frac{\partial M_{el,2}}{\partial t} + \frac{\partial \phi_{E}}{\partial M_{el,3}}\frac{\partial M_{el,3}}{\partial t}$$
Based on equation $\ref{Definition of Lagrangian to Euler Reference Frame Mapping}$, we can express $M_{el,i}$ as $x_{i}$ which is spatial coordinates in the eulerian reference frame,
$$\frac{d}{d t}\left[\phi_{E}(x_{1}, x_{2}, x_{3}, t)\right] = \frac{\partial \phi_{E}}{\partial t} + \frac{\partial \phi_{E}}{\partial x_{1}}\frac{\partial x_{1}}{\partial t} + \frac{\partial \phi_{E}}{\partial x_{2}}\frac{\partial x_{2}}{\partial t} + \frac{\partial \phi_{E}}{\partial x_{3}}\frac{\partial x_{3}}{\partial t}$$
Now our right hand side is no longer in terms of initial spatial positions $x_{in,i}$ but is now in terms of eulerian spatial positions $x_{i}$.
Let's examine one of the time derivatives of spatial position $x_{i}$,
$$\frac{\partial x_{1}}{\partial t} = \frac{\partial }{\partial t}\left[M_{el,1}(x_{in,1}, x_{in,2}, x_{in,3}, t)\right]$$
Remember that the mapping above takes all of the initial position of a certain point fluid particle to tell the location of the same fluid particle at some time $t$. 
This means that the time derivative of our mapping is just our $x_{1}$ velocity of the fluid particle.
$$\frac{\partial x_{1}}{\partial t} = v_{1}$$
wherein $v_{1}$ represents the velocity of our fluid particle in the $x_{1}$ direction.
\begin{figure}[H]\centering
\def\svgwidth{400px}
\import{Images/}{Particle_Trajectory.pdf_tex}
\caption{The trajectory of a particle from its initial position to some position at time $t$}
\label{particle trajectory substantive derivative}
\end{figure}
Generalizing this to all directions, we can say that,
$$\frac{\partial x_{i}}{\partial t} = v_{i}$$
Substituting this realization of the time derivatives,
$$\frac{d}{d t}\left[\phi_{E}(x_{1}, x_{2}, x_{3}, t)\right] = \frac{\partial \phi_{E}}{\partial t} + \frac{\partial \phi_{E}}{\partial x_{1}}v_{1} + \frac{\partial \phi_{E}}{\partial x_{2}}v_{2} + \frac{\partial \phi_{E}}{\partial x_{3}}v_{3}$$
Substituting the equation above into the right hand side of equation $\ref{Almost Done Substantive Derivative primitive 1}$,
$$\frac{\partial}{\partial t}\left[\phi_{L}(x_{in,1},x_{in,2},x_{in,3},t)\right] = \frac{d}{d t}\left[\phi_{E}(M_{el,1}, M_{el,2}, M_{el,3}, t)\right]$$
$$\frac{\partial}{\partial t}\left[\phi_{L}(x_{in,1},x_{in,2},x_{in,3},t)\right] = \frac{\partial \phi_{E}}{\partial t} + \frac{\partial \phi_{E}}{\partial x_{1}}v_{1} + \frac{\partial \phi_{E}}{\partial x_{2}}v_{2} + \frac{\partial \phi_{E}}{\partial x_{3}}v_{3}$$
Shortening the notation of the equation above by implying the input arguments to $\phi_{L}$,
$$\frac{\partial \phi_{L}}{\partial t} = \frac{\partial \phi_{E}}{\partial t} + \frac{\partial \phi_{E}}{\partial x_{1}}v_{1} + \frac{\partial \phi_{E}}{\partial x_{2}}v_{2} + \frac{\partial \phi_{E}}{\partial x_{3}}v_{3}$$
We have finally arrived at the definition of substantive derivative which relates time derivative in the eulerian reference frame to the time derivative in the lagrangian reference frame.
Now the expression above can be re-written in more compact vector form,
\begin{equation}\frac{\partial \phi_{L}}{\partial t} = \frac{\partial \phi_{E}}{\partial t} + \bar{v}\cdot(\nabla \phi_{E}) \label{Substantive Derivative Final Vector Notation}\end{equation}
wherein $\bar{v}$ is the velocity vector defined below,
$$\bar{v} = \begin{bmatrix}
v_{1} && v_{2} && v_{3}
\end{bmatrix}^{T}$$
The substantive derivative could also be expressed compactly in tensor index notation,
\begin{equation}\frac{\partial \phi_{L}}{\partial t} = \frac{\partial \phi_{E}}{\partial t} + v_{i}\frac{\partial \phi_{E}}{\partial x_{i}} \label{Substantive Derivative Final Tensor Index Notation}\end{equation}
For better visualization, we might manipulate the equation above into the following form,
$$\frac{\partial \phi_{E}}{\partial t} = \frac{\partial \phi_{L}}{\partial t} - v_{i}\frac{\partial \phi_{E}}{\partial x_{i}}$$
\begin{figure}[H]\centering
\def\svgwidth{480px}
\import{Images/}{Substantive_Derivative_Term_Explanations.pdf_tex}
\caption{Moving Fluid Particles with Respect to Volume Fixed in Space}
\label{Control Volume Eulerian Lagrangian Substantive Derivative}
\end{figure}
The term $\partial \phi_{L}/\partial t$ represents diffusion of the quantity of interest into or out of the main fluid particle.
The term $-v_{i}\partial \phi_{E}/\partial x_{i}$ represents convection of the quantity of itnerest out or into the volume that is fixed in some space in the eulerian reference frame.
It just comes to show that the substantive derivative actually has a good physical interpretation.
The figure above is an exaggeration in that the main fluid particle should not be that far out of the control volume.
The figure also displays a square volume and this is exaggerated.
In reality, the substantive derivative is used on points and not volumes.
The square volume is just an extremely zoomed out image of a particle so just imagine the figure above, but shrink it infinitely to the size of a single particle, and that would be an accurate representation of the substantive derivative.
Even so, the figure shows a good illustration of both the diffusive term and the convective term at play on a point in space in the substantive derivative.
The time derivative of some arbitrary quantity moving along with the fluid in the lagrangian reference frame (substantive derivative) is often considered with a separate notation shown below,
\begin{equation}\frac{D \phi}{Dt} = \frac{\partial \phi_{L}}{\partial t} = \frac{\partial \phi_{E}}{\partial t} + v_{i}\frac{\partial \phi_{E}}{\partial x_{i}} \label{Substantive Derivative Extra Notation}\end{equation}

