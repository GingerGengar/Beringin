\chapter{One-Dimensional Wave Equation}
\begin{comment} 
Physics Archives
\end{comment}
The equation for the one-dimensional wave equation is shown below,
$$\frac{\partial^2 y}{\partial t^2} = a^2\frac{\partial^2 y}{\partial x^2}$$ 
Using the familiar substitution $\displaystyle{y(x,t) = u(x)\times r(t)}$,
$$\frac{\partial^2}{\partial t^2}\left[u(x)r(t)\right] = a^2\frac{\partial^2}{\partial x^2}\left[u(x)r(t)\right]$$
$$u(x)r''(t) = a^2u''(x)r(t)$$ 
wherein $r''(t)$ and $u''(x)$ represents the second order derivative of $r(t)$ and $u(x)$ respectively. Manipulation of the equation,
$$\frac{1}{a^2}\frac{r''(t)}{r(t)} = \frac{u''(x)}{u(x)}$$
To fit for the somewhat arbitrary conditions,
$$y(0,t) = y(L,t) = 0\quad,\quad y(x,0) = f(x) \quad,\quad \frac{\partial}{\partial t}\left[y(x,0)\right] = g(x)$$
it is useful to consider the derivative homogenous case (A) and the displacement homogenous case (B) seperately,
$$y_A(0,t) = y_A(L,t) = 0\quad,\quad y_B(0,t) = y_B(L,t) = 0$$
$$y_A(x,0) = f(x)\quad,\quad y_B(x,0) = 0$$
$$\frac{\partial }{\partial t}[y_A(x,0)] = 0\quad,\quad \frac{\partial }{\partial t}[y_B(x,0)] = g(x)$$
The somewhat arbitray boundary condition case would be the satisfied by the  addition of the derivative homogenous case  and the displacement homogenous case. Both function $y_A(x,t)$ and $y_B(x,t)$ satisifes the partial differential equation, therefore, the algebraic addition of them must also satisfy the one dimensional wave equation. When they are added algebraically,
$$y_A(0,t) + y_B(0,t) = y_A(L,t) + y_B(L,t) = 0 + 0$$
$$y_A(x,0) + y_B(x,0) = f(x) + 0\quad,\quad \frac{\partial}{\partial t}\left[y_A(x,0) + y_B(x,0)\right] = 0 + g(x)$$
Therefore, the algebraic addition of the derivative homogenou and displacement homogenous satisfies the somewhat arbitrary conditions provided earlier.
%Seperator
%Seperator
%Seperator
%Seperator
%Seperator
\section{Derivative Homogenous Case}
\begin{comment}
\end{comment}
The boundary conditions for the derivative homogenous case,
$$y(0,t) = y(L,t) = 0\quad,\quad y(x,0) = f(x)\quad,\quad\frac{\partial }{\partial t}[y(x,0)] = 0$$
To satisfy the first boundary condition listed above, it would be convenient for the eigenvalues to be considered negative,
$$\frac{1}{a^2}\frac{r''(t)}{r(t)} = \frac{u''(x)}{u(x)} = -\lambda$$
Therefore, the two ordinary differential equations,
$$u''(x) = -\lambda u(x)\quad,\quad r''(t) = -\lambda a^2 r(t)$$
The characteristic equation associated to the displacement differential equation,
$$r^2 = -\lambda \quad,\quad r = \sqrt{\lambda}i$$
Therefore, the displacement function $u(x)$ is in terms of the familiar linear combination of trigonometric functions,
$$u(x) = c_1\cos\left(\sqrt{\lambda}x\right) + c_2\sin\left(\sqrt{\lambda}x\right)$$
The only constants that will satisfy the condition $\displaystyle{y(0,t) = y(L,t) = 0}$, $c_1 = 0$, and $\sqrt{\lambda}L = n\pi$. Therefore, the eigenvalues are, $\displaystyle{\lambda = \frac{n^2\pi^2}{L^2}}$. Substituting for eigenvalues and arbitray constant $c_1$ into $u(x)$,
$$u(x) = c_2\sin\left(\frac{n\pi}{L}x\right)$$
The characteristic equation associated to the time dependent differential equation,
$$r^2 = -\lambda a^2\quad,\quad r = a\sqrt{\lambda}i$$
Therefore, the trigonometric solution to the above characteristic equation,
$$r(t) = k_1\cos\left(a\sqrt{\lambda}t\right) + k_2\sin\left(a\sqrt{\lambda}t\right)$$
$$r(t) = k_1\cos\left(\frac{n\pi a}{L}t\right) + k_2\sin\left(\frac{n\pi a}{L}t\right)$$
%Seperator
%Seperator
%Seperator
%Seperator
%Seperator
\section{Displacement Homogenous Case}
\begin{comment}
\end{comment}
The boundary conditions for the displacement homogenous case,
$$y(0,t) = y(L,t) = 0\quad,\quad y(x,0) = 0\quad,\quad\frac{\partial }{\partial t}[y(x,0)] = g(x)$$
