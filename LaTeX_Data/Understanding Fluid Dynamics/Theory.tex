Now that we have the basic tools, such as the Stokes-Kelvin Theorem, Divergence Theorem, Substantive Derivative, and the Reynold's Transport Theorem it is time to implement some physics into our fluids.
In classical physics, we have the following conservation laws:
\begin{enumerate}
\item Conservation of Mass
\item Conservation of Linear Momentum
\item Conservation of Angular Momentum
\item Conservation of Energy
\end{enumerate}
Conservation of angular momentum yields some useful results but is not widely needed.
This leaves us $3$ main conservation laws that we can describe our fluid in (mass, momentum, and energy).
%Seperator
\\~\\The conservation of energy is important in the existence of shocks when we have vehicles flying at supersonic or even hypersonic speeds and also typically in large heat transfer situations like within a gas turbine (jet) engine.
%Seperator
\\~\\For many applications where we have incompressible subsonic flow without large heat transfers, we can negelct the energy equation because it does not really yield additional useful information.

