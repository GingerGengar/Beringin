
The conservation of mass states that mass cannot be created or destroyed.
If we follow a volume that always contains the same set of fluid particles, then that particular volume we are following must always have the same amount of mass.
Mathematically, this just means that the substantive derivative of a volume deforming and moving with the fluid must evaluate to zero.
Re-iterating the tensor form of the Reynold's transport theorem (equation $\ref{Reynolds Transport Theorem Tensor Index Notation}$),
$$\frac{D}{Dt}\iiint^{}_{V_{m}(t)} \phi \,dV_{o} = \iiint^{}_{V_{m}(t)} \frac{\partial \phi}{\partial t} + \frac{\partial}{\partial x_{k}} (\phi v_{k}) \,dV_{o}$$
Remmeber that $\phi$ was some arbitrary quantity. 
This time, let us consider $\phi$ to be the density of the fluid and we choose this so that the left hand side volumetric integral just implies total mass contained within our deforming volume $V_{m}(t)$.
$$\frac{D}{Dt}\iiint^{}_{V_{m}(t)} \rho \,dV_{o} = \iiint^{}_{V_{m}(t)} \frac{\partial \rho}{\partial t} + \frac{\partial}{\partial x_{k}} (\rho v_{k}) \,dV_{o}$$
wherein $\rho$ represents the density of the fluid.
The volumetric integral on the left hand side is just the total mass contained within our deforming volume,
$$M = \iiint^{}_{V_{m}(t)} \rho \,dV_{o}$$
Substituting this,
$$\frac{DM}{Dt} = \iiint^{}_{V_{m}(t)} \frac{\partial \rho}{\partial t} + \frac{\partial}{\partial x_{k}} (\rho v_{k}) \,dV_{o}$$
Based on conservation of mass, the mass contained within our deforming volume must always be constant because the deforming volume contains the same set of particles at all times.
Therefore, the left hand side of the equation above evaluates to zero.
$$0 = \iiint^{}_{V_{m}(t)} \frac{\partial \rho}{\partial t} + \frac{\partial}{\partial x_{k}} (\rho v_{k}) \,dV_{o}$$
We need to remember that the volume $V_{m}(t)$ is defined purely as a function of time $t$ because once we have chosen the volume initially, its deformation will deterministically change to always accomodate the same set of fluid particles.
However, the initial selection of our volume $V_{m}$ at time $t = 0$ was arbitrary. 
Since $V_{m}(0)$ can be chosen arbitrarily, we know that the volumetric integral can be performed over ANY region in the fluid.
This must only be possible if the integrand in the equation above evaluates to zero.
Therefore,
\begin{equation}0 = \frac{\partial \rho}{\partial t} + \frac{\partial}{\partial x_{k}} (\rho v_{k}) \label{Governing Equation Fluid Conservation of Mass}\end{equation}
We have finally arrived at our first governing equation, conservation of mass.

