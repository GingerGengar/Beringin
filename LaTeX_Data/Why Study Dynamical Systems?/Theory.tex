\chapter{Why Study Dynamical Systems?}

The equations derived here traces back its origins to Newton's laws of motion, which are meant to govern \textbf{all} moving things in the universe.
Thanks to Albert Einstein, we know that Newton's laws of motion are incorrect for incredibly high velocities approaching the speed of light, but since most of our life in this universe occurs at velocities much lower than speed of light, Newton's laws of motions remain true and very valid.
%Seperator
\\~\\Why should we learn the movements of all things around us? 
What is the point?
If we can somehow understand how the system around us behaves, we could better design our system or design good controllers that allow us to get our desired outcome in a predictable way. 
%Seperator
\\~\\An example of good dynamics application is that based on the equations of motions for an aircraft, it is possible to design aircrafts which have a "self-righting" tendency, meaning that if the aircraft is disturbed by a gust of wind, it has a tendency to correct itself.
It may come as a surprise but the very first aircraft had the opposite tendency where a gust of wind would have a tendency to make the aircraft pitch and crash into the ground.
Without understanding dynamics we would have no idea why our aircraft crashes to the ground, how to avoid it or how to just design a good aircraft in general.
This is an example of how good dynamics can lead us to design a better system (the aircraft).
%Seperator
\\~\\Another example, when we launch spacecrafts into the solar system, we need to understand how the spacecraft interacts with all the heavenly bodies around it.
Only after we understand how the spacecraft interacts can we begin to think about engine manuvers to allow the spacecraft to reach our intended destination such as the moon or Mars or whatever.
Without a clear understanding of the spacecraft gravitational interactions, it is impossible to get the spacecraft landing to whichever planet we want.
This is an example of "controlling" our "system" (spacecraft) to get our desired outcome (landing in another planet).
%Seperator
\\~\\Take another example of a helicopter.
A helicopter is not a sane system, if the pilot just does random control commands, the helicopter crashes. 
It is only after careful examination of how the helicopter interacts with the environment (its aerodynamic forces, the angular momentum of the blades and air, location of center of gravity) can the pilot understand how to keep the helicopter flying up in the air and less of crashing into the ground.
This is also another example of "controlling" our "system" (helicopter) to get our desired outcome (not crashing).
%Seperator
\\~\\Typically for a new "system" (I know this is abstract but think of anything you really want), you can attempt to use the laws of motions prescribed by Isaac Newton to then come up with a mathematical description of how the system behaves. 
Then, from this description, we can make a controller to force the "system" (some abstract thing) to get our desired outcome.
Note here that understanding the system mathematically is vital in the system's design and its controls.
%Seperator
\\~\\Now the next few sections are going to deal with developing the tools to properly analyze all moving things, and I want you to know that this is an incredibly powerful concept.
With this, we can analyze easily all flight vehicles, like aircrafts, gliders, autogyros, quadcopters.
%Seperator
\\~\\It doesn't just stop there, since our equations are completely general, they can describe \textbf{anything}, for example all automotive vehicles can also be described by these laws of motions.
Things like bikes, segways, skateboards, motorcycles, scooters, cars, trucks, rovers, and other multi-wheeled devices can be studied using these tools.
%Seperator
\\~\\Robotic armatures such as those being used in the most advanced factory assembly lines used to manufacture complex vehicles and applications can be completely studied using the tools that we are going to develop here as well. 
%Seperator
\\~\\We can also study biological systems like legged animals, bipedal animals, and also flying animals.
Perhaps that if these animals have been honed and "perfected" over thousands of years, we might learn a few things from studying them and perhaps making their clones in robots. 
A potential use for the study of legged systems is to make a powered exoskeleton for medical purposes or for construction work.

