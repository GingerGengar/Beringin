%Seperator
%Seperator
%Seperator
%Seperator
%Seperator
%Seperator
\chapter{Translational Dynamics: Newton's $2^{nd}$ Law}
\begin{comment}
\end{comment}
%Seperator
%Seperator
%Seperator
%Seperator
%Seperator
\section{Point Particle}
\begin{comment}
\end{comment}
%Seperator
%Seperator
%Seperator
%Seperator
\subsection{Inertial Reference Frame}
\begin{comment}
\end{comment}
Newton's $2^{nd}$ law for a point particle $p$ is shown below,
\begin{equation}
\sum^{n}_{i = 1}\left[{}^{e}\bar{F}^{p}_{i}\right] = m_{p}\overset{e}{\frac{\partial^{2}}{\partial t^{2}}}[\bar{r}^{o_{e}p}]
\label{Newtons Second Law}
\end{equation}
$LHS$ represents the summation of $n$ number of forces acting on particle $p$. 
The forces here are perceived through an inertial frame $e$, and hence, represent true forces and not 'fictitious forces'. 
$RHS$ represents the acceleration of particle $p$ relative to the origin of the inertial frame $e$ and perceived from the inertial frame $e$. 
$m_{p}$ represents the mass of the particle $p$ and $\bar{r}^{o_{e}p}$ represents the position vector of particle $p$ relative to the origin of the inertial frame $e$. 
Note that the left-right sub-superscripts follow the notation convention described in the terminology. 
$o_{e}p$ represents the position vector starting from the origin of frame $e$ and ending at the position of particle $p$. 

%Seperator
%Seperator
%Seperator
%Seperator
\subsection{Non-Inertial Reference Frame} 
\label{non-inertial ref frame newtons second law point particle}
\begin{comment}
\end{comment}
We can always re-write Newton's second law (equation $\ref{Newtons Second Law}$) in a non-inertial reference frame by expressing the second order derivative of position in the inertial reference frame as a second order derivative of position in the non-inertial reference frame using the kinematics transport theorem (equation $\ref{Kinematic Transport Theorem Main Result}$).
Re-iterating equation $\ref{Kinematic Transport Theorem Main Result}$,
\begin{equation}
\overset{e}{\frac{\partial^{2}}{\partial t^{2}}}[\bar{r}^{o_{e}p}] =
\overset{e}{\frac{\partial^{2}}{\partial t^{2}}}[\bar{R}^{o_{e}o_{b}}]     
+ \overset{b}{\frac{\partial^{2}}{\partial t^{2}}}[\bar{r}^{o_{b}p}] 
+ 2{}^{e}\bar{\omega}^{b}\times\overset{b}{\frac{\partial}{\partial t}}[\bar{r}^{o_{b}p}]
+ {}^{e}\bar{\omega}^{b}\times\left\{{}^{e}\bar{\omega}^{b}\times\left[\bar{r}^{o_{b}p}\right]\right\}    
+ \left\{\overset{b}{\frac{\partial}{\partial t}}[{}^{e}\bar{\omega}^{b}]\right\}\times\bar{r}^{o_{b}p}
\end{equation}
Substituting this into Newton's second law (equation $\ref{Newtons Second Law}$),
$$\sum^{n}_{i = 1}\left[{}^{e}\bar{F}^{p}_{i}\right] = m_{p}\left\{\overset{e}{\frac{\partial^{2}}{\partial t^{2}}}[\bar{r}^{o_{e}p}]\right\}$$
$$\sum^{n}_{i = 1}\left[{}^{e}\bar{F}^{p}_{i}\right] = m_{p}\left\{ \overset{e}{\frac{\partial^{2}}{\partial t^{2}}}[\bar{R}^{o_{e}o_{b}}]     
+ \overset{b}{\frac{\partial^{2}}{\partial t^{2}}}[\bar{r}^{o_{b}p}] 
+ 2{}^{e}\bar{\omega}^{b}\times\overset{b}{\frac{\partial}{\partial t}}[\bar{r}^{o_{b}p}]
+ {}^{e}\bar{\omega}^{b}\times\left\{{}^{e}\bar{\omega}^{b}\times\left[\bar{r}^{o_{b}p}\right]\right\}    
+ \left\{\overset{b}{\frac{\partial}{\partial t}}[{}^{e}\bar{\omega}^{b}]\right\}\times\bar{r}^{o_{b}p} \right\}$$
Expanding the equation above,
\begin{align*}
\sum^{n}_{i = 1}\left[{}^{e}\bar{F}^{p}_{i}\right] = &m_{p}\overset{e}{\frac{\partial^{2}}{\partial t^{2}}}[\bar{R}^{o_{e}o_{b}}]     
+ m_{p}\overset{b}{\frac{\partial^{2}}{\partial t^{2}}}[\bar{r}^{o_{b}p}] 
+ 2m_{p}\left[{}^{e}\bar{\omega}^{b}\times\overset{b}{\frac{\partial}{\partial t}}(\bar{r}^{o_{b}p})\right] \\ 
& + m_{p}\left\{{}^{e}\bar{\omega}^{b}\times\left[{}^{e}\bar{\omega}^{b}\times\left(\bar{r}^{o_{b}p}\right)\right]\right\}    
+ m_{p}\left\{\overset{b}{\frac{\partial}{\partial t}}[{}^{e}\bar{\omega}^{b}]\right\}\times\bar{r}^{o_{b}p}
\end{align*}
We can keep the only term that contains mass times acceleration in the non-inertial reference frame on the right hand side and "move" everything else over to the left hand side,
\begin{align}&\sum^{n}_{i = 1}\left[{}^{e}\bar{F}^{p}_{i}\right]
 - m_{p}\overset{e}{\frac{\partial^{2}}{\partial t^{2}}}[\bar{R}^{o_{e}o_{b}}]
 - 2m_{p}\left[{}^{e}\bar{\omega}^{b}\times\overset{b}{\frac{\partial}{\partial t}}(\bar{r}^{o_{b}p})\right]
 - m_{p}\left\{{}^{e}\bar{\omega}^{b}\times\left[{}^{e}\bar{\omega}^{b}\times\left(\bar{r}^{o_{b}p}\right)\right]\right\} \\ &
 - m_{p}\left\{\overset{b}{\frac{\partial}{\partial t}}[{}^{e}\bar{\omega}^{b}]\right\}\times\bar{r}^{o_{b}p}
 = m_{p}\overset{b}{\frac{\partial^{2}}{\partial t^{2}}}[\bar{r}^{o_{b}p}] 
\label{Newtons Second Law Non-Inertial Verbose}
\end{align}
Please remember that in section $\ref{Centrifugal, Coriolis, Azimuthal Introduction}$ we had introduced what the fictitious, coriolis, centrifugal, and azimuthal acceleration terms.
Here we see a multiplication product of the various fictitious accelerations with mass of the particle which we can consider to be "fictitious forces" because acceleration multiplied by mass gives us forces. 

Here we see that indeed we can still say $F = ma$ for non-inertial reference frame as long as we add the fictitious forces on to the left hand side of Newton's second law for a non-inertial reference frame.
%Seperator
%Seperator
\\~\\Here we see exactly how the fictitious forces appear, we write down Newton's second law for an inertial reference frame, convert the second order derivatives in inertial reference frame to that of a non-inertial reference frame, then group it together with the external force applied to the point particle and that's how we arrive at Newton's second law for a non-inertial reference frame.
To make our lives a little bit simpler, we can always replace multiplicative product of mass and fictitious acceeleration with their corresponding fictitious force names,
\begin{itemize}
\item Consider $\displaystyle  - m_{p}\overset{e}{\frac{\partial^{2}}{\partial t^{2}}}[\bar{R}^{o_{e}o_{b}}]$ as fictitious translational force $\bar{F}^{p}_{trans}$, consistent with fictitious translational acceleration,
\item Consider $\displaystyle  - 2m_{p}\left[{}^{e}\bar{\omega}^{b}\times\overset{b}{\frac{\partial}{\partial t}}(\bar{r}^{o_{b}p})\right]$ as coriolis force $\bar{F}^{p}_{cor}$ consistent with coriolis acceeleration
\item Consider $\displaystyle  - m_{p}\left\{{}^{e}\bar{\omega}^{b}\times\left[{}^{e}\bar{\omega}^{b}\times\left(\bar{r}^{o_{b}p}\right)\right]\right\}$ as centrifugal force $\bar{F}^{p}_{centr}$ consistent with centrifugal acceleration
\item Consider $\displaystyle  - m_{p}\left\{\overset{b}{\frac{\partial}{\partial t}}[{}^{e}\bar{\omega}^{b}]\right\}\times\bar{r}^{o_{b}p}$ as azimuthal force $\bar{F}^{p}_{azi}$ consistent with azimuthal acceleration
\end{itemize}
Then we can short-hand the verbose equation $\ref{Newtons Second Law Non-Inertial Verbose}$ into something more compact.
Note that this is just syntactic sugar, just to group the terms and make it more intuitive, but we have not added any new information here.
\begin{equation}
\sum^{n}_{i = 1}\left[{}^{e}\bar{F}^{p}_{i}\right] + \bar{F}^{p}_{trans} + \bar{F}^{p}_{cor} + \bar{F}^{p}_{centr} + \bar{F}^{p}_{azi} 
= m_{p}\overset{b}{\frac{\partial^{2}}{\partial t^{2}}}[\bar{r}^{o_{b}p}] 
\label{Newtons Second Law Non-Inertial Simple}
\end{equation}


%Seperator
%Seperator
%Seperator
%Seperator
%Seperator
\section{Rigid Bodies}
\begin{comment}
\end{comment}
%Seperator
%Seperator
%Seperator
%Seperator
\subsection{Inertial Reference Frame}
\begin{comment}
\end{comment}
Newton's $2^{nd}$ law for the $i^{th}$ point particle in a rigid body is shown below,
$$\sum^{n_{1}}_{j = 1}\left[{}^{e}\bar{F}^{p_{i}}_{j}\right] = m_{p_{i}}\overset{e}{\frac{\partial^{2}}{\partial t^{2}}}[\bar{r}^{o_{e}p_{i}}]$$
Making the substitutions,
\begin{equation}\overset{e}{\frac{\partial^{2}}{\partial t^{2}}}[\bar{r}^{o_{e}p_{i}}] = {}^{e}\bar{a}^{o_{e}p_{i}}\quad,\quad \sum^{n_{1}}_{j = 1}\left[{}^{e}\bar{F}^{p_{i}}_{j}\right] = {}^{e}\bar{F}^{p_{i}}_{ext} + \sum^{n_{1}}_{j = 1}\left[{}^{e}\bar{F}^{p_{i}}_{int,j}\right]\label{rigid-forces}\end{equation}
Newton's $2^{nd}$ law for the $i^{th}$ point particle in a rigid body becomes,
$${}^{e}\bar{F}^{p_{i}}_{ext} + \sum^{n_{1}}_{j = 1}\left[{}^{e}\bar{F}^{p_{i}}_{int,j}\right] = m_{i}{}^{e}\bar{a}^{o_{e}p_{i}}$$
wherein ${}^{e}\bar{F}^{p_{i}}_{ext}$ represents the resultant external force. 
The 'B' specifier represents that the external force is acting on the $i^{th}$ particle. 
The 'A' specifier represents that the external force referred to are perceived in an inertial reference frame. 
The external forces expressed here are real and does not contain fictitious elements.
\\~\\${}^{e}\bar{F}^{p_{i}}_{int,j}$ represents the forces acting on the $i^{th}$ particle due to intermolecular interaction between the particles in the rigid body. 
The 'A' specifier represents the intermolecular forces are perceived in an inertial reference frame. 
The 'B' specifier represents that the intermolecular forces are acting on the $i^{th}$ particle. 
The 'D' specifier holds the counter variable $j$ that is used to sum over all the intermolecular forces experienced by the $i^{th}$ particle. 
%Seperator
%Seperator
\\~\\$m_{i}$ represents the mass of the $i^{th}$ particle and $n_{2}$ represents the number of adjacent particles that has a force interaction with the $i^{th}$ particle.
\\~\\${}^{e}\bar{a}^{o_{e}p_{i}}$ represents acceleration of the $i^{th}$ particle. 
The 'A' specifier represents the reference frame that acceleration is perceived in. 
The 'B' specifier represents that this acceleration is for $i^{th}$ particle taken with reference to the origin of the inertial frame $e$.
\\~\\To find the resultant force on a rigid body, one must find the summation of all forces acting on each particle of the rigid body. 
Taking the summation of Newton's $2^{nd}$ law for point particles on the entire rigid body,
$$\sum^{n_{2}}_{i = 1}\left[{}^{e}\bar{F}^{p_{i}}_{ext}\right] + \sum^{n_{2}}_{i = 1}\sum^{n_{1}}_{j = 1}\left[{}^{e}\bar{F}^{p_{i}}_{int,j}\right] = \sum^{n_{2}}_{i = 1}\left[m_{i}{}^{e}\bar{a}^{o_{e}p_{i}}\right]$$
Here $n_{2}$ represents the total number of discrete particles that exist within the rigid body. 
There are $2$ statements that can be made regarding the intermolecular forces:
\begin{itemize}
    \item A particle cannot exert an intermolecular force on itself. 
Therefore the intermolecular force a particle exerts on itself is zero.
    \item By Newton's third law, the force acting on the $i^{th}$ particle by the $j^{th}$ particle is exactly equal and opposite to the force acting on the $j^{th}$ particle by the $i^{th}$ particle.
\end{itemize}
These $2$ statements lead to the summation of intermolecular forces within the rigid body evaluating to zero,
$$\sum^{n_{2}}_{i = 1}\sum^{n_{1}}_{j = 1}\left[{}^{e}\bar{F}^{p_{i}}_{int,j}\right] = 0$$
Simplifying Newton's $2^{nd}$ law,
\begin{equation}\sum^{n_{2}}_{i = 1}\left[{}^{e}\bar{F}^{p_{i}}_{ext}\right] = \sum^{n_{2}}_{i = 1}\left[m_{i}{}^{e}\bar{a}^{o_{e}p_{i}}\right]\label{mambah-1}\end{equation}
The center of mass definition for the rigid body is shown below,
\begin{equation}
\bar{r}^{o_{e}c} = \frac{\displaystyle \sum^{n_{2}}_{i = 1}\left[m_{i}\bar{r}^{o_{e}p_{i}}\right]}{\displaystyle \sum^{n_{2}}_{i = 1}\left[m_{i}\right]}
\label{Center of Mass Definition}
\end{equation}
$\displaystyle \sum^{n_{2}}_{i = 1}\left[m_{i}\right]$ represents the total mass of the rigid body. 
This is a scalar, and is constant due to the conservation of mass. 
Manipulating the equation,
$$\left\{\sum^{n_{2}}_{i = 1}\left[m_{i}\right]\right\}\bar{r}^{o_{e}c} = \sum^{n_{2}}_{i = 1}\left[m_{i}\bar{r}^{o_{e}p_{i}}\right]$$
Taking the frame derivative with respect to the inertial $e$ frame,
$$\overset{e}{\frac{\partial}{\partial t}}\left\{\left[\displaystyle \sum^{n_{2}}_{i = 1}\left(m_{i}\right)\right]\bar{r}^{o_{e}c}\right\} = \overset{e}{\frac{\partial}{\partial t}}\left\{\sum^{n_{2}}_{i = 1}\left[m_{i}\bar{r}^{o_{e}p_{i}}\right]\right\}$$
Since $m_{i}$ is a constant due to conservation of mass,
$$\left[\displaystyle \sum^{n_{2}}_{i = 1}\left(m_{i}\right)\right]\overset{e}{\frac{\partial}{\partial t}}\left\{\bar{r}^{o_{e}c}\right\} = \sum^{n_{2}}_{i = 1}\left[\overset{e}{\frac{\partial}{\partial t}}\left\{m_{i}\bar{r}^{o_{e}p_{i}}\right\}\right]$$
\begin{equation}
\left[ \sum^{n_{2}}_{i = 1}\left(m_{i}\right)\right]{}^{e}\bar{v}^{o_{e}c} = \sum^{n_{2}}_{i = 1}\left[m_{i}\overset{e}{\frac{\partial}{\partial t}}\left\{\bar{r}^{o_{e}p_{i}}\right\}\right]
\label{velocity shorthand notation Newtons second law rigid body}
\end{equation}
Here we introduce velocity of the center of mass perceived in the $e$-frame and relative to the origin of frame $e$ as ${}^{e}\bar{v}^{o_{e}c}$. 
Simplifying further,
$$\left[\sum^{n_{2}}_{i = 1}\left(m_{i}\right)\right]{}^{e}\bar{v}^{o_{e}c} = \sum^{n_{2}}_{i = 1}\left[m_{i}{}^{e}\bar{v}^{o_{e}p_{i}}\right]$$
Taking the frame derivative with respect to the $e$-frame once more,
$$\overset{e}{\frac{\partial}{\partial t}}\left\{\left[\sum^{n_{2}}_{i = 1}\left(m_{i}\right)\right]{}^{e}\bar{v}^{o_{e}c}\right\} = \overset{e}{\frac{\partial}{\partial t}}\left\{\sum^{n_{2}}_{i = 1}\left[m_{i}{}^{e}\bar{v}^{o_{e}p_{i}}\right]\right\}$$
Performing similar operations as before,
$$\left[\sum^{n_{2}}_{i = 1}\left(m_{i}\right)\right]\overset{e}{\frac{\partial}{\partial t}}\left\{{}^{e}\bar{v}^{o_{e}c}\right\} = \sum^{n_{2}}_{i = 1}\left[m_{i}\overset{e}{\frac{\partial}{\partial t}}\left\{{}^{e}\bar{v}^{o_{e}p_{i}}\right\}\right]$$
\begin{equation}\left[\sum^{n_{2}}_{i = 1}\left(m_{i}\right)\right]{}^{e}\bar{a}^{o_{e}c} = \sum^{n_{2}}_{i = 1}\left[m_{i}{}^{e}\bar{a}^{o_{e}p_{i}}\right]\label{mambah-2}\end{equation}
Here we introduce the acceleration of the center of mass perceived in the $e$-frame and relative to the origin of frame $e$ as ${}^{e}\bar{a}^{o_{e}p_{i}}$. 
Substituting equation $\ref{mambah-2}$ to equation $\ref{mambah-1}$,
$$\sum^{n_{2}}_{i = 1}\left[{}^{e}\bar{F}^{p_{i}}_{ext}\right] = \sum^{n_{2}}_{i = 1}\left[m_{i}{}^{e}\bar{a}^{o_{e}p_{i}}\right] = \left[\sum^{n_{2}}_{i = 1}\left(m_{i}\right)\right]{}^{e}\bar{a}^{o_{e}c}$$
Let $\displaystyle M = \left[\sum^{n_{2}}_{i = 1}\left(m_{i}\right)\right]$. 
Here $M$ represents the total mass of the rigid body. 
Substituting for $M$,
\begin{equation}
\sum^{n_{2}}_{i = 1}\left[{}^{e}\bar{F}^{p_{i}}_{ext}\right] = M{}^{e}\bar{a}^{o_{e}c}
\label{Newtons Second Law System of Particles Inertial Reference Frame}
\end{equation}
This is the final form for Newton's $2^{nd}$ law for a rigid body. 
On the $LHS$ there is the resultant external force acting on the rigid body. 
On the $RHS$ there is the total mass of the rigid body multiplied by acceleration of the center of mass. 
Due to Newton's $3^{rd}$ law, the internal forces of a rigid body can be safely ignored. 
The rigid body form of Newton's $2^{nd}$ law replaces forces on individual point particles with the summation of external forces and also replaces acceleration of single particles with the acceleration of the rigid body's center of mass. 
The rigid body form of Newton's $2^{nd}$ law replaces mass of the $i^{th}$ particle with the total mass of the rigid body $M$ when compared to Newton's $2^{nd}$ law for a single particle.
%Seperator
%Seperator
%Seperator
%Seperator
\subsection{Non-Inertial Reference Frame}
\label{non-inertial newtons second law system of particles section}
\begin{comment}
\end{comment}
From equation $\ref{velocity shorthand notation Newtons second law rigid body}$ and equation $\ref{mambah-2}$, we used the following shorthand notations,
$$\overset{e}{\frac{\partial}{\partial t}}\left\{\bar{r}^{o_{e}c}\right\} = {}^{e}\bar{v}^{o_{e}c} \quad,\quad 
\overset{e}{\frac{\partial}{\partial t}}\left\{{}^{e}\bar{v}^{o_{e}c}\right\} = {}^{e}\bar{a}^{o_{e}c}$$
From these $2$ definitions of the shorthand, we can deduce that,
$$\overset{e}{\frac{\partial^{2}}{\partial t^{2}}}[\bar{r}^{o_{e}c}] = {}^{e}\bar{a}^{o_{e}c}$$
Re-iterating Newton's second law for a system of particles (equation $\ref{Newtons Second Law System of Particles Inertial Reference Frame}$) and then substituting the equation above,
\begin{equation}
\sum^{n_{2}}_{i = 1}\left[{}^{e}\bar{F}^{p_{i}}_{ext}\right] = M{}^{e}\bar{a}^{o_{e}c} = M\overset{e}{\frac{\partial^{2}}{\partial t^{2}}}[\bar{r}^{o_{e}c}]
\label{Newtons second law verbose rigid system intermediate}
\end{equation}
You can already see where this is going. 

All we just said that acceleration is second order derivative of position in the inertial reference frame and then we can use the kinematic transport theorem (equation $\ref{Kinematic Transport Theorem Main Result}$) to express the acceleration in the inertial reference frame in terms of the acceleration in the non-inertial reference frame with the addition of fictitious forces to prove Neton's second law for a system of particles in a non-inertial reference frame.
Re-iterating equation $\ref{Kinematic Transport Theorem Main Result}$ but replacing the arbitrary point $p$ with the system of particle's center of mass $c$, 
$$\overset{e}{\frac{\partial^{2}}{\partial t^{2}}}[\bar{r}^{o_{e}c}] =
\overset{e}{\frac{\partial^{2}}{\partial t^{2}}}[\bar{R}^{o_{e}o_{b}}]
+ \overset{b}{\frac{\partial^{2}}{\partial t^{2}}}[\bar{r}^{o_{b}c}] 
+ 2{}^{e}\bar{\omega}^{b}\times\overset{b}{\frac{\partial}{\partial t}}[\bar{r}^{o_{b}c}]
+ {}^{e}\bar{\omega}^{b}\times\left\{{}^{e}\bar{\omega}^{b}\times\left[\bar{r}^{o_{b}c}\right]\right\}    
+ \left\{\overset{b}{\frac{\partial}{\partial t}}[{}^{e}\bar{\omega}^{b}]\right\}\times\bar{r}^{o_{b}c}$$
Substituting into Newton's second law (equation $\ref{Newtons second law verbose rigid system intermediate}$),
$$\sum^{n_{2}}_{i = 1}\left[{}^{e}\bar{F}^{p_{i}}_{ext}\right] = M\overset{e}{\frac{\partial^{2}}{\partial t^{2}}}[\bar{r}^{o_{e}c}]$$
$$\sum^{n_{2}}_{i = 1}\left[{}^{e}\bar{F}^{p_{i}}_{ext}\right] = M\left\{ \overset{e}{\frac{\partial^{2}}{\partial t^{2}}}[\bar{R}^{o_{e}o_{b}}]
+ \overset{b}{\frac{\partial^{2}}{\partial t^{2}}}[\bar{r}^{o_{b}c}] 
+ 2{}^{e}\bar{\omega}^{b}\times\overset{b}{\frac{\partial}{\partial t}}[\bar{r}^{o_{b}c}]
+ {}^{e}\bar{\omega}^{b}\times\left\{{}^{e}\bar{\omega}^{b}\times\left[\bar{r}^{o_{b}c}\right]\right\}    
+ \left\{\overset{b}{\frac{\partial}{\partial t}}[{}^{e}\bar{\omega}^{b}]\right\}\times\bar{r}^{o_{b}c} \right\}$$
Expanding the equation to isolate the fictitious forces,
\begin{align*}
\sum^{n_{2}}_{i = 1}\left[{}^{e}\bar{F}^{p_{i}}_{ext}\right] = &  M\overset{e}{\frac{\partial^{2}}{\partial t^{2}}}[\bar{R}^{o_{e}o_{b}}]
+ M\overset{b}{\frac{\partial^{2}}{\partial t^{2}}}[\bar{r}^{o_{b}c}] 
+ 2M\left[{}^{e}\bar{\omega}^{b}\times\overset{b}{\frac{\partial}{\partial t}}(\bar{r}^{o_{b}c})\right] \\ &
+ M\left\{{}^{e}\bar{\omega}^{b}\times\left[{}^{e}\bar{\omega}^{b}\times\left(\bar{r}^{o_{b}c}\right)\right]\right\}
+ M\left[\overset{b}{\frac{\partial}{\partial t}}({}^{e}\bar{\omega}^{b})\right]\times\bar{r}^{o_{b}c} 
\end{align*}
Just as we have already mentioned before, we are now going to move all of the extra forces to the left hand side to form our well-known fictitious forces,
\begin{align}
& \sum^{n_{2}}_{i = 1}\left[{}^{e}\bar{F}^{p_{i}}_{ext}\right] 
 - M\overset{e}{\frac{\partial^{2}}{\partial t^{2}}}[\bar{R}^{o_{e}o_{b}}]
 - 2M\left[{}^{e}\bar{\omega}^{b}\times\overset{b}{\frac{\partial}{\partial t}}(\bar{r}^{o_{b}c})\right] \\ &
 - M\left\{{}^{e}\bar{\omega}^{b}\times\left[{}^{e}\bar{\omega}^{b}\times\left(\bar{r}^{o_{b}c}\right)\right]\right\}
 - M\left[\overset{b}{\frac{\partial}{\partial t}}({}^{e}\bar{\omega}^{b})\right]\times\bar{r}^{o_{b}c}
 = M\overset{b}{\frac{\partial^{2}}{\partial t^{2}}}[\bar{r}^{o_{b}c}]
\label{Newtons Second law System of Particles Non-inertial reference frame}
\end{align}
We have finally arrived at Newton's second law applied to a system of particles in a non-inertial reference frame.

