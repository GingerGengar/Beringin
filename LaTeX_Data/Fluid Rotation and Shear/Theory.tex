
The velocity gradient tensor $e_{ij}$ is defined below,
$$e_{ij} = \frac{\partial v_{i}}{\partial x_{k}}$$
This term appears in the conservation of momentum for fluids, and this velocity gradient tensor has an incredibly important significance.
The velocity gradient tensor is a tensor of rank 2.
We can re-write this velocity gradient tensor in the following form,
$$\frac{\partial v_{i}}{\partial x_{j}} = \frac{\partial v_{i}}{\partial x_{j}} + \frac{1}{2}\frac{\partial v_{j}}{\partial x_{k}} - \frac{1}{2}\frac{\partial v_{j}}{\partial x_{k}}$$
$$\frac{\partial v_{i}}{\partial x_{j}} = \frac{1}{2}\frac{\partial v_{i}}{\partial x_{j}} + \frac{1}{2}\frac{\partial v_{i}}{\partial x_{j}} + \frac{1}{2}\frac{\partial v_{j}}{\partial x_{k}} - \frac{1}{2}\frac{\partial v_{j}}{\partial x_{k}}$$
Re-arranging the terms a little bit,
$$\frac{\partial v_{i}}{\partial x_{j}} = \frac{1}{2}\frac{\partial v_{i}}{\partial x_{j}} + \frac{1}{2}\frac{\partial v_{j}}{\partial x_{k}} + \frac{1}{2}\frac{\partial v_{i}}{\partial x_{j}} - \frac{1}{2}\frac{\partial v_{j}}{\partial x_{k}}$$
Grouping the terms together,
$$\frac{\partial v_{i}}{\partial x_{j}} = \left[\frac{1}{2}\frac{\partial v_{i}}{\partial x_{j}} + \frac{1}{2}\frac{\partial v_{j}}{\partial x_{k}}\right] + \left[\frac{1}{2}\frac{\partial v_{i}}{\partial x_{j}} - \frac{1}{2}\frac{\partial v_{j}}{\partial x_{k}}\right]$$
\begin{equation}\frac{\partial v_{i}}{\partial x_{j}} = \frac{1}{2}\left[\frac{\partial v_{i}}{\partial x_{j}} + \frac{\partial v_{j}}{\partial x_{k}}\right] + \frac{1}{2}\left[\frac{\partial v_{i}}{\partial x_{j}} - \frac{\partial v_{j}}{\partial x_{k}}\right] \label{raw decomposition of velocity gradient tensor Rotation and Shear}\end{equation}
We have essentially re-written the velocity gradient tensor in terms of a symmetric tensor and an anti-symmetric tensor. 
The symmetric tensor (firs term in the right hand side) represents the shear rate of the fluid at a particular point.
The anti-symmetric tensor (second term in the right hand side) represents the rotation rate of the fluid particle at a particular point.
By convention, we rename the symmetric strain rate tensor as $S_{ij}$ and we name the anti-symmetric rotation rate tensor as $\Omega_{ij}$,
$$S_{ij} = \frac{1}{2}\left[\frac{\partial v_{i}}{\partial x_{j}} + \frac{\partial v_{j}}{\partial x_{k}}\right] \quad,\quad \Omega_{ij} = \frac{1}{2}\left[\frac{\partial v_{i}}{\partial x_{j}} - \frac{\partial v_{j}}{\partial x_{k}}\right]$$
which also if we are feeling a bit circular, means that the velocity gradient tensor may be written as the combination of the local shear strain rate $S_{ij}$ and a local rotation rate $\Omega_{ij}$,
$$e_{ij} = S_{ij} + \Omega_{ij}$$

