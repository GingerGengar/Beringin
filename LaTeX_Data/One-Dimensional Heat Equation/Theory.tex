\chapter{Temperature Distribution of a Heated Rod}
\begin{comment} 
Physics Archives
Havent inserted Sturm Liouville orthonormal basis functions
\end{comment}
The function of temperature $u$ at some distance $x$ from the origin of a one-dimensional heated rod of uniform material is governed by the equation below,
$$\frac{\partial u}{\partial t} = k\frac{\partial^2 u}{\partial x^2}$$
wherein $k$ is some constant related to thermal conductivity. Using the substitution, $u(x,t) = f(x)g(t)$ wherein $f(x)$ is a function purely in $x$ and $g(t)$ is a function purely in terms of $t$,
$$\frac{\partial [f(x)g(t)]}{\partial t} = k\frac{\partial^2 [f(x)g(t)]}{\partial x^2}$$
$$f(x)\frac{\partial [g(t)]}{\partial t} = kg(t)\frac{\partial^2 [f(x)]}{\partial x^2}$$
$$f(x)g'(t) = kg(t)f''(x)$$wherein $g'(t)$ represents the first order derivative of $g(t)$ with respect to time $t$, and $f''(x)$ represents second order derivative of $f(x)$ with respect to distance $x$. Further manipulation to yield a left hand side completely in terms of time $t$ and a right hand side completely in terms of displacement $x$,
$$\frac{1}{k}\frac{g'(t)}{g(t)} = \frac{f''(x)}{f(x)} = \lambda$$wherein $\lambda$ is some constant. The reason why $\lambda$ is a constant is because $x$ and $t$ are independent variables, therefore a change in one of the values should not affect the other variable. Since the left hand side is and right hand side are in represented completely in independent variables, $\lambda$ must be a constant if $x$ and $t$ are independent variables. $\lambda$ is the eigenvalue of the problem whose values are often sought and is inferred from the boundary conditions. This is as far as analysis can go without specifying the boundary conditions.
%Seperator
%Seperator
%Seperator
\section{Zero-Endpoint Temperatures}
\begin{comment} 
\end{comment}
Suppose the rod is of length $L$ and that the initial temperature distribution is known. The boundary conditions,
$$u(0,t) = u(L,t) = 0\quad,\quad u(x,0) = m(x)$$The first boundary condition is the zero endpoint condition and the second condition is the initial temperature distribution. Reiterating the first boundary condition and substituting $u(x,t)$ as the product of two single variable functions,
$$u(0,t) = u(L,t) = 0\quad,\quad f(0)g(t) = f(L)g(t) = 0$$
The function $g(t)$ is not trivial, and therefore, $g(t) \neq 0$. Therefore, it follows that,
$$f(0) = f(L) = 0$$Because the endpoint conditions, it is convenient that the function $f(x)$ be made into a trigonometric function. Consider the eigenvalue to be negative,
$$\frac{1}{k}\frac{g'(t)}{g(t)} = \frac{f''(x)}{f(x)} = -\lambda$$
Two ordinary differential equation problems can be obtained from this,
$$\frac{1}{k}\frac{g'(t)}{g(t)} = -\lambda \quad,\quad \frac{f''(x)}{f(x)} = -\lambda$$
$$g'(t) = -\lambda kg(t) \quad,\quad f''(x) = -\lambda f(x)$$                               
The second ordinary differential equation, $\displaystyle{f''(x) = -\lambda f(x)}$ has the solution,
$$f(x) = c_1\cos\left(\sqrt{\lambda}x\right) + c_2\sin\left(\sqrt{\lambda}x\right)$$To satisfy the endpoint condition $\displaystyle{f(0) = f(L) = 0}$, $c_1 = 0$ and $\sqrt{\lambda}L = n\pi$, wherein $n$ are integers starting from zero. Therefore, $\lambda = n^2\pi^2/L^2$. Substituting the eigenvalues and arbitrary constants $c_1$,
$$f(x) = c_2\sin\left(\frac{n\pi}{L} x\right)$$
Substituting the eigenvalue and solving the second ordinary differential equations problem,
$$g'(t) = -\frac{n^2\pi^2k}{L^2}g(t)$$
$$\int\frac{1}{g(t)}dg(t) = -\frac{n^2\pi^2k}{L^2}\int dt$$
$$\ln[g(t)] = -\frac{n^2\pi^2k}{L^2}t + c$$
$$g(t) = Ce^{\displaystyle{-\left(\frac{n^2\pi^2k}{L^2}\right)t}}$$
Substituting the two equations together,
$$u(x,t) = C_ne^{\displaystyle{-\left(\frac{n^2\pi^2k}{L^2}\right)t}}\sin\left(\frac{n\pi}{L} x\right)$$
Due to the partial differential operator being a linear operator, the superposition principle holds true. Therefore, the general solution to the partial differential equation must be the linear combination of its linearly independent solutions,
$$u_g(x,t) = \sum^{\infty}_{n=1}\left[C_ne^{\displaystyle{-\left(\frac{n^2\pi^2k}{L^2}\right)t}}\sin\left(\frac{n\pi}{L} x\right)\right]$$
%Seperator
%Seperator
%Seperator
\section{Insulated Ends}
\begin{comment} 
\end{comment}
With the same length of rod $L$ and known initial temperature distribution $m(x)$, the boundary conditions,
$$\frac{\partial}{\partial x}\left[u(0,t)\right] = \frac{\partial}{\partial x}\left[u(L,t)\right] = 0\quad,\quad u(x,0) = m(x)$$
Substituting the boundary conditions with the definition of $u$ as the product of two single variable functions,
$$f'(0)g(t) = f'(L)g(t) = 0$$
Similarly to the previous case, since $g(t)$ is not the trivial zero function,
$$f'(0) = f'(L) = 0$$
Just in the previous part, it is convenient to choose the eigenvalues to be negative in the two ordinary differential equation problems,
$$\frac{1}{k}\frac{g'(t)}{g(t)} = -\lambda \quad,\quad \frac{f''(x)}{f(x)} = -\lambda$$
This is advantageous because $f$ will take the form of a linear combination of trigonometric functions, at which we can simply choose the cosine series to satisfy the boundary condition above. The general form of $f(x)$,
$$f(x) = c_1\cos\left(\sqrt{\lambda}x\right) + c_2\sin\left(\sqrt{\lambda}x\right)$$
$$f'(x) = -c_1\sin\left(\sqrt{\lambda}x\right) + c_2\cos\left(\sqrt{\lambda}x\right)$$
The only conditions that would satisfy the end-point boundary conditions, $c_2 = 0$, $\sqrt{\lambda}L = n\pi$, $\lambda = n^2\pi^2/L^2$. Substituting the eigenvalues and arbitrary constants would yield,
$$f(x) = c_1\cos\left(\frac{n\pi}{L}x\right)$$
Solving for $g(t)$ yields,
$$g'(t) = -\lambda k g(t)$$
Familiarly,
$$g(t) =  e^{-\lambda k t} = Ce^{\displaystyle{\left(-\frac{n^2\pi^2k}{L^2} t\right)}}$$
Similarly to the previous chapter and by principle of superposition,
$$u_g(x,t) = \sum^{\infty}_{n=1}\left[C_ne^{\displaystyle{-\left(\frac{n^2\pi^2k}{L^2}\right)t}}\cos\left(\frac{n\pi}{L} x\right)\right]$$
