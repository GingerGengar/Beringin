
There are $4$ postulates regarding Newtonian fluids:
\begin{enumerate}
\item When the fluid is not moving, the stresses exerted by the fluid is just the thermodynamic pressure.\label{thermodynamic pressure condition}
\item The stress tensor $\sigma_{ij}$ is going to be directly proportional to the velocity gradient tensor and only that tensor.
This just means physically that if a fluid particle is moving within a fluid, then that fluid particle will "drag" its neighbour to move along with it, and the way the moving fluid particle "drags" its neighbours is by applying stress on neighbouring fluid particles. \label{stress tensor directly proportional to velocity gradient}
\item If the fluid is experiencing a pure rotation, then the fluid will not experience any shear stress. \label{pure rotation no shear stress}
\item The fluid is isotropic, which just means that all basic fluid natural properties are the same in all directions. 
The fluid basically has no preferred direction to flow and it considers all directions to be equal.
Examples of fluids which are non-isotropic are liquid crystals where the fluid properties will change based on the orientation of the liquid crystals.
Air and water are isotropic because both fluids can flow in whichever direction just the same. \label{isotropy condition fluid shear stress}
\end{enumerate}
Based on condition $\ref{thermodynamic pressure condition}$ we can write the stress tensor in the following form
\begin{equation}\sigma_{ij} = -p\delta_{ij} + \tau_{ij} \label{Stress Tensor Constitutive Relation Fluid Momentum}\end{equation}
wherein $p$ represents the thermodynamic pressure in the fluid and $\tau_{ij}$ represents shear stress that is only dependent on fluid flow velocity.
the shear stress $\tau_{ij}$ evaluates to zero when the fluid is completely still.
Based on condition $\ref{stress tensor directly proportional to velocity gradient}$ we believe that the stress tensor is a linear combination of the velocity gradients.
So that's exactly what we did.
The shear stress is a tensor of rank $2$ and so to construct the linear combination of all the elements in a second order tensor, we need a fourth order tensor.
The tensor $F_{ijkl}$ is going to have $81$ elements.
$$\tau_{ij} = F_{ijkl}\frac{\partial u_{k}}{\partial x_{l}}$$
Using equation $\ref{raw decomposition of velocity gradient tensor Rotation and Shear}$ we can re-write the linear combination of velocity gradients as the linear combination of strain rates and rotation rates of the fluid as shown below.
$$\tau_{ij} = \frac{1}{2}G_{ijkl}\left(\frac{\partial v_{k}}{\partial x_{l}} + \frac{\partial v_{l}}{\partial x_{k}}\right) + \frac{1}{2}H_{ijkl}\left(\frac{\partial v_{k}}{\partial x_{l}} - \frac{\partial v_{l}}{\partial x_{k}}\right)$$
Based on the condition that a fluid particle doing a pure rotation will experience no shear stress since it is not experiencing any shear deformation (condition $\ref{pure rotation no shear stress}$), then we can tell that $H_{ijkl} = 0$
$$\tau_{ij} = \frac{1}{2}G_{ijkl}\left(\frac{\partial v_{k}}{\partial x_{l}} + \frac{\partial v_{l}}{\partial x_{k}}\right)$$
There is a unique fourth order tensor that satisfies condition $\ref{isotropy condition fluid shear stress}$ and that fourth order tensor is shown below,
$$G_{ijkl} = \lambda \delta_{ij}\delta_{kl} + \mu(\delta_{ik}\delta_{jl} + \delta_{il}\delta_{jk}) + \gamma(\delta_{ik}\delta_{jl} - \delta_{il}\delta_{jk})$$
Substituting into the expression for the shear stress $\tau_{ij}$,
$$\tau_{ij} = \frac{1}{2}[\lambda \delta_{ij}\delta_{kl} + \mu(\delta_{ik}\delta_{jl} + \delta_{il}\delta_{jk}) + \gamma(\delta_{ik}\delta_{jl} - \delta_{il}\delta_{jk})]\left[\frac{\partial v_{k}}{\partial x_{l}} + \frac{\partial v_{l}}{\partial x_{k}}\right]$$
If $\gamma$ is non-zero then the right hand side of the equation above might have rotation rate terms, and since we need our shear stress $\tau_{ij}$ to purely depend on shear stress and not rotation rate due to condition $\ref{pure rotation no shear stress}$, it means that $\gamma$ must be zero.
$$\tau_{ij} = \frac{1}{2}[\lambda \delta_{ij}\delta_{kl} + \mu(\delta_{ik}\delta_{jl} + \delta_{il}\delta_{jk})]\left[\frac{\partial v_{k}}{\partial x_{l}} + \frac{\partial v_{l}}{\partial x_{k}}\right]$$
Let us expand the equation above,
$$\tau_{ij} = \frac{1}{2}[\lambda \delta_{ij}\delta_{kl} + \mu\delta_{ik}\delta_{jl} + \mu\delta_{il}\delta_{jk}]\left[\frac{\partial v_{k}}{\partial x_{l}} + \frac{\partial v_{l}}{\partial x_{k}}\right]$$
\begin{equation}
\tau_{ij} = \frac{1}{2}\lambda \delta_{ij}\delta_{kl}\left[\frac{\partial v_{k}}{\partial x_{l}} + \frac{\partial v_{l}}{\partial x_{k}}\right]
 + \frac{1}{2}\mu\delta_{ik}\delta_{jl}\left[\frac{\partial v_{k}}{\partial x_{l}} + \frac{\partial v_{l}}{\partial x_{k}}\right]
 + \frac{1}{2}\mu\delta_{il}\delta_{jk}\left[\frac{\partial v_{k}}{\partial x_{l}} + \frac{\partial v_{l}}{\partial x_{k}}\right]
\label{shear stress fluid primitive part 1}
\end{equation}
Let us examine the first term on the right hand side of equation $\ref{shear stress fluid primitive part 1}$ and use the renaming property of the kronecker delta,
$$\frac{1}{2}\lambda \delta_{ij}\delta_{kl}\left[\frac{\partial v_{k}}{\partial x_{l}} + \frac{\partial v_{l}}{\partial x_{k}}\right] = \frac{1}{2}\lambda \delta_{ij}\left[\delta_{kl}\frac{\partial v_{k}}{\partial x_{l}} + \delta_{kl}\frac{\partial v_{l}}{\partial x_{k}}\right]$$
$$\frac{1}{2}\lambda \delta_{ij}\delta_{kl}\left[\frac{\partial v_{k}}{\partial x_{l}} + \frac{\partial v_{l}}{\partial x_{k}}\right] = \frac{1}{2}\lambda \delta_{ij}\left[\frac{\partial v_{k}}{\partial x_{k}} + \frac{\partial v_{k}}{\partial x_{k}}\right]$$
$$\frac{1}{2}\lambda \delta_{ij}\delta_{kl}\left[\frac{\partial v_{k}}{\partial x_{l}} + \frac{\partial v_{l}}{\partial x_{k}}\right] = \lambda \delta_{ij}\frac{\partial v_{k}}{\partial x_{k}}$$
Now let us examine the second term on the right hand side of equation $\ref{shear stress fluid primitive part 1}$ and use the renaming property of the kronecker delta,
$$\frac{1}{2}\mu\delta_{ik}\delta_{jl}\left[\frac{\partial v_{k}}{\partial x_{l}} + \frac{\partial v_{l}}{\partial x_{k}}\right] = \frac{1}{2}\mu\left[\delta_{ik}\delta_{jl}\frac{\partial v_{k}}{\partial x_{l}} + \delta_{ik}\delta_{jl}\frac{\partial v_{l}}{\partial x_{k}}\right]$$
$$\frac{1}{2}\mu\delta_{ik}\delta_{jl}\left[\frac{\partial v_{k}}{\partial x_{l}} + \frac{\partial v_{l}}{\partial x_{k}}\right] = \frac{1}{2}\mu\left[\frac{\partial v_{i}}{\partial x_{j}} + \frac{\partial v_{j}}{\partial x_{i}}\right]$$
Now let us examine the third term on the right hand side of equation $\ref{shear stress fluid primitive part 1}$ and use the renaming property of the kronecker delta,
$$\frac{1}{2}\mu\delta_{il}\delta_{jk}\left[\frac{\partial v_{k}}{\partial x_{l}} + \frac{\partial v_{l}}{\partial x_{k}}\right] = \frac{1}{2}\mu\left[\delta_{il}\delta_{jk}\frac{\partial v_{k}}{\partial x_{l}} + \delta_{il}\delta_{jk}\frac{\partial v_{l}}{\partial x_{k}}\right]$$
$$\frac{1}{2}\mu\delta_{il}\delta_{jk}\left[\frac{\partial v_{k}}{\partial x_{l}} + \frac{\partial v_{l}}{\partial x_{k}}\right] = \frac{1}{2}\mu\left[\frac{\partial v_{j}}{\partial x_{i}} + \frac{\partial v_{i}}{\partial x_{j}}\right]$$
Combining all of these simplification back into equation $\ref{shear stress fluid primitive part 1}$,
$$\tau_{ij} = \lambda \delta_{ij}\frac{\partial v_{k}}{\partial x_{k}} + \frac{1}{2}\mu\left[\frac{\partial v_{i}}{\partial x_{j}} + \frac{\partial v_{j}}{\partial x_{i}}\right] + \frac{1}{2}\mu\left[\frac{\partial v_{j}}{\partial x_{i}} + \frac{\partial v_{i}}{\partial x_{j}}\right]$$
$$\tau_{ij} = \lambda \delta_{ij}\frac{\partial v_{k}}{\partial x_{k}} + \mu\left[\frac{\partial v_{i}}{\partial x_{j}} + \frac{\partial v_{j}}{\partial x_{i}}\right]$$
Now that we have an expression for the shear stress, we can finally put this back into the equation for the stress tensor $\ref{Stress Tensor Constitutive Relation Fluid Momentum}$
$$\sigma_{ij} = -p\delta_{ij} + \tau_{ij}$$
$$\sigma_{ij} = -p\delta_{ij} + \lambda \delta_{ij}\frac{\partial v_{k}}{\partial x_{k}} + \mu\left[\frac{\partial v_{i}}{\partial x_{j}} + \frac{\partial v_{j}}{\partial x_{i}}\right]$$

