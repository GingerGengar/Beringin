
%seperator
%seperator
%seperator
%seperator
%seperator
%seperator
\chapter{Kinematic Transport Theorem}
\begin{comment}
\end{comment}
The Kinematic Transport Theorem (KTT) specifies the relationship between vector time derivatives of different reference frames. 
Kinematics is the study of motion without cause. 
The consequences of the derivations below can only tell how some vector's derivative with respect to time is 'perceived' when viewed in a different way (reference frame). 
The derivations below tell nothing of how some object should move. 
The derivation below only claims, if this is a particular type of movement, this can be 'described' differently in another different reference frame when that different reference frame is also moving.
%Seperator
\\~\\Reference frames are typically defined with an origin, and a set of basis vectors. 
Reference frames in $3$-dimensions typically have $3$ basis vectors to allow general vectors to be expressed. 
Vectors are typically defined as linear combinations of reference frame basis vectors. 
Suppose the $e$ reference frame has $3$ basis vectors, $\hat{e_{1}}$, $\hat{e_{2}}$, and $\hat{e_{3}}$. 
Let the position of some particle $p$ relative to the origin $o$ of the $e$ reference frame be denoted as $\left(\bar{r}^{op}\right)$.
We can express this position vector as a linear combination of the basis vectors ($\hat{e_{1}}$, $\hat{e_{2}}$, $\hat{e_{3}}$) of the $e$ reference frame, 
$$\bar{r}^{op} = a\hat{e_{1}} + b\hat{e_{2}} + c\hat{e_{3}}$$
wherein $a$, $b$, and $c$, are scalar coefficients. 
In typical matrix notation representations, the vector $\bar{r}^{op}$ could be expressed as,
$$\bar{r}^{op} = \begin{bmatrix}
a && b && c
\end{bmatrix}^{T}$$
Since reference frames are arbitrarily constructed, let reference frame $b$ have its origin in the same position as reference frame $e$. 
However, reference frame $b$ has different basis vectors than reference frame $e$. 
The basis vectors for reference frame $b$ is defined to be $\hat{b_{1}}$, $\hat{b_{2}}$, and $\hat{b_{3}}$. 
The vector $\bar{r}^{op}$ can now be defined as,
$$\bar{r}^{op} = d\hat{b_{1}} + f\hat{b_{2}} + g\hat{b_{3}}$$
wherein $d$, $f$, and $g$, are different scalar coefficients than $a$, $b$, and $c$. 
Just as in the previous case, using the notation rule described earlier, $\bar{r}^{op}$ can be expressed as,
$$\bar{r}^{op} = \begin{bmatrix}
d && f && g
\end{bmatrix}^{T}$$
This is very confusing. 
To distinguish the basis vectors that are used to express $\bar{r}^{op}$, the $C$ specifier is used to differentiate basis vector sets. 
Hence, we can safely express $\bar{r}^{op}$ below,
$${}^{}_{e}\bar{r}^{op} = \begin{bmatrix}
a && b && c
\end{bmatrix}^{T} \quad,\quad {}^{}_{f}\bar{r}^{op} = \begin{bmatrix}
d && f && g
\end{bmatrix}^{T}$$
Throughout the workings, the attribute 'inertial' and 'non-inertial' are often used. 
Inertial reference frames are reference frames that are either still in free space or are moving at some constant velocity in a particular direction in free space. 
Any rotation, and acceleration of the reference frame origins are not allowed in inertial reference frames. 
If a particular reference frame exhibits rotation in free space or acceleration of its origin, then the reference frame must be non-inertial.
%seperator
%seperator
%seperator
%seperator
%seperator
\section{Frame Derivative}
\begin{comment}
\end{comment}
Vectors in $R^{n}$ can be expressed by an infinite combination of $n$ linearly independent basis vectors. 
Since the basis vectors used is somewhat arbitrary, then there are infinite ways to express a vector in $R^{n}$ generally, and more specifically in $R^{3}$. 
The time derivative of a vector is a representation of how the scalar coefficients of the basis vectors change with respect to time. 
Since the scalar coefficients used to represent a vector is dependent on the basis vectors used, which in turn, is dependent on a declared reference frame, then it is only appropriate to specify time derivative for vectors to a specific reference frame. 
Taking the time derivative of a vector 'v' with respect to reference frame 'M' is to determine how the scalar coefficients representing vector 'v'as a linear combination of basis vectors for reference frame 'M' changes with respect to time.
%Seperator
\\~\\Taking the time derivative of a vector, 'with respect to a frame' is defined as a new operation: Frame derivative. 
The $LHS$ of the equation below represents the frame derivative and the $RHS$ shows the implementation of the frame derivative. 
We use the same vector $\bar{r}^{op}$ as before.
$$\overset{e}{\frac{\partial}{\partial t}}[\bar{{r}^{op}}] = \hat{e_{1}}\frac{d}{d t}[a] + \hat{e_{2}}\frac{d}{d t}[b] + \hat{e_{3}}\frac{d}{d t}[c] \quad,\quad \overset{f}{\frac{\partial}{\partial t}}[\bar{r}^{op}] = \hat{b_{1}}\frac{d}{d t}[d] + \hat{b_{2}}\frac{d}{d t}[f] + \hat{b_{3}}\frac{d}{d t}[g]$$
Note that the time derivative of the scalar quantity must match with the appropriate basis vector and must match with the appropriate refrence frame. 
More generally,
$$\overset{e}{\frac{\partial^n}{\partial t^n}}[\bar{{r}^{op}}] = \hat{e_{1}}\frac{d^n}{d t^n}[a] + \hat{e_{2}}\frac{d^n}{d t^n}[b] + \hat{e_{3}}\frac{d^n}{d t^n}[c] \quad,\quad \overset{f}{\frac{\partial^n}{\partial t^n}}[\bar{r}^{op}] = \hat{b_{1}}\frac{d^n}{d t^n}[d] + \hat{b_{2}}\frac{d^n}{d t^n}[f] + \hat{b_{3}}\frac{d^n}{d t^n}[g]$$


%Insert some wrong examples below:


%seperator
%seperator
%seperator
%seperator
%seperator
\section{$1^{st}$ Order Derivative}
\begin{comment}
\end{comment}
%Enter Some proof as to why this is true
Let $e$ be an inertial reference frame and  $b$ be a non-inertial reference frame. 
For a vector $\bar{v}$ that starts at the origin of reference frame $b$,
\begin{equation}
\overset{e}{\frac{\partial}{\partial t}}[\bar{v}] = \overset{b}{\frac{\partial}{\partial t}}[\bar{v}] + {}^{e}\bar{\omega}^{b}\times\bar{v}\label{KTT-1st-form}
\end{equation}
%Seperator
\begin{figure}[H]\centering
\def\svgwidth{330px}
\subimport{Images/}{Frame_Derivative_Overview.pdf_tex}
\caption{Consider the simple case where a non-inertial reference frame is rotating simple about the inertial reference frame's third axis. We have drawn an arbitrary vector $\bar{v}$}
\label{Frame Derivative Overview first order}
\end{figure}
%Seperator
\begin{figure}[H]\centering
\def\svgwidth{500px}
\subimport{Images/}{Frame_Derivative_Specifics.pdf_tex}
\caption{Here we show that how this arbitrary vector is "perceived" (in terms of linear combination of basis vectors in the non-inertial reference frame) changes due to the rotation of the non-inertial reference frame. So don't be surprised of equation $\ref{KTT-1st-form}$}
\label{Frame Derivative Specifics first order}
\end{figure}
%KTT of velocity
Let $o_{e}$ represent the origin of reference frame $e$ and $o_{b}$ represent the origin of reference frame $b$. 
Let the position vector $\bar{R}^{o_{e}o_{b}}$ represent $o_{b}$ relative to $o_{e}$. 
Let $\bar{r}^{o_{b}p}$ represent the position of point $p$ with respect to $o_{b}$. 
Then, it must follow,
$$\bar{r}^{o_{e}p} = \bar{R}^{o_{e}o_{b}} + \bar{r}^{o_{b}p}$$
%Seperator
\begin{figure}[H]\centering
\def\svgwidth{500px}
\subimport{Images/}{General_Motion_Two_Reference_Frames.pdf_tex}
\caption{Illustration of problem setup. Dashed lines are supposed to represent position vectors.}
\label{General Motion Two Reference Frames}
\end{figure}
%Seperator
Taking the time derivative with respect to the inertial frame $e$
$$\overset{e}{\frac{\partial}{\partial t}}[\bar{r}^{o_{e}p}] = \overset{e}{\frac{\partial}{\partial t}}[\bar{R}^{o_{e}o_{b}}] + \overset{e}{\frac{\partial}{\partial t}}[\bar{r}^{o_{b}p}]$$
Based on equation \ref{KTT-1st-form},
$$\overset{e}{\frac{\partial}{\partial t}}[\bar{r}^{o_{b}p}] = \overset{b}{\frac{\partial}{\partial t}}[\bar{r}^{o_{b}p}] + {}^{e}\bar{\omega}^{b}\times\bar{r}^{o_{b}p}$$
Substituting,
$$\overset{e}{\frac{\partial}{\partial t}}[\bar{r}^{o_{e}p}] = \overset{e}{\frac{\partial}{\partial t}}[\bar{R}^{o_{e}o_{b}}] + \overset{b}{\frac{\partial}{\partial t}}[\bar{r}^{o_{b}p}] + {}^{e}\bar{\omega}^{b}\times\bar{r}^{o_{b}p}$$
This is an important equation. 
$LHS$ represents the first order time derivative with respect to an inertial reference frame of $\displaystyle \bar{r}^{o_{e}p}$. 
If vector $\displaystyle \bar{r}^{o_{e}p}$ represents position vector of some point, then $LHS$ represents the velocity of that point according to the inertial reference frame $e$. 
The term $\displaystyle \overset{e}{\frac{\partial}{\partial t}}[\bar{R}^{o_{e}o_{b}}]$ represents the velocity of non-inertial reference frame $b$ origin relative relative to the origin of inertial reference frame $e$ viewed in the $e$ frame. 
The term $\displaystyle \overset{b}{\frac{\partial}{\partial t}}[\bar{r}^{o_{b}p}]$ represents the velocity of point $p$ relative to origin of reference frame $b$ viewed in non-inertial reference frame $b$. 
The term $\displaystyle {}^{e}\bar{\omega}^{b}\times\bar{r}^{o_{b}p}$ represents additional velocity perceived in the inertial reference frame $e$ point $p$ exhibits due to the rotation of reference frame $b$ relative to reference frame $e$.


%seperator
%seperator
%seperator
%seperator
%seperator
\section{$2^{nd}$ Order Derivative} \label{Centrifugal, Coriolis, Azimuthal Introduction}
\begin{comment}
\end{comment}
Taking the frame derivative of the velocity expression with respect to the inertial frame $e$,
\begin{equation}
\overset{e}{\frac{\partial}{\partial t}}\left\{\overset{e}{\frac{\partial}{\partial t}}[\bar{r}^{o_{e}p}]\right\}      =    \overset{e}{\frac{\partial}{\partial t}}\left\{\overset{e}{\frac{\partial}{\partial t}}[\bar{R}^{o_{e}o_{b}}]\right\}     +  \overset{e}{\frac{\partial}{\partial t}}\left\{\overset{b}{\frac{\partial}{\partial t}}[\bar{r}^{o_{b}p}]\right\}      +   \overset{e}{\frac{\partial}{\partial t}}\left\{{}^{e}\bar{\omega}^{b}\times\bar{r}^{o_{b}p}\right\}
\label{intermediate-2}
\end{equation}
Simplifying to second order frame derivatives,
$$\overset{e}{\frac{\partial}{\partial t}}\left\{\overset{e}{\frac{\partial}{\partial t}}[\bar{r}^{o_{e}p}]\right\} = \overset{e}{\frac{\partial^{2}}{\partial t^{2}}}[\bar{r}^{o_{e}p}]\quad,\quad \overset{e}{\frac{\partial}{\partial t}}\left\{\overset{e}{\frac{\partial}{\partial t}}[\bar{R}^{o_{e}o_{b}}]\right\} =  \overset{e}{\frac{\partial^{2}}{\partial t^{2}}}[\bar{R}^{o_{e}o_{b}}]$$

Susbtituting into equation $\ref{intermediate-2}$,
\begin{equation}\overset{e}{\frac{\partial^{2}}{\partial t^{2}}}[\bar{r}^{o_{e}p}]    =    \overset{e}{\frac{\partial^{2}}{\partial t^{2}}}[\bar{R}^{o_{e}o_{b}}]     +  \overset{e}{\frac{\partial}{\partial t}}\left\{\overset{b}{\frac{\partial}{\partial t}}[\bar{r}^{o_{b}p}]\right\}      +   \overset{e}{\frac{\partial}{\partial t}}\left\{{}^{e}\bar{\omega}^{b}\times\bar{r}^{o_{b}p}\right\}\label{inter-4}\end{equation}

Applying equation $\ref{KTT-1st-form}$ to one of the terms in equation $\ref{intermediate-2}$,
$$\overset{e}{\frac{\partial}{\partial t}}\left\{\overset{b}{\frac{\partial}{\partial t}}[\bar{r}^{o_{b}p}]\right\} = \overset{b}{\frac{\partial}{\partial t}}\left\{\overset{b}{\frac{\partial}{\partial t}}[\bar{r}^{o_{b}p}]\right\} + {}^{e}\bar{\omega}^{b}\times\overset{b}{\frac{\partial}{\partial t}}[\bar{r}^{o_{b}p}]$$
Simplifying to second order frame derivatives,
\begin{equation}\overset{e}{\frac{\partial}{\partial t}}\left\{\overset{b}{\frac{\partial}{\partial t}}[\bar{r}^{o_{b}p}]\right\} = \overset{b}{\frac{\partial^{2}}{\partial t^{2}}}[\bar{r}^{o_{b}p}] + {}^{e}\bar{\omega}^{b}\times\overset{b}{\frac{\partial}{\partial t}}[\bar{r}^{o_{b}p}]\label{whammu-1}\end{equation}

Using product rule for the last term in equation $\ref{intermediate-2}$, 
\begin{equation}\overset{e}{\frac{\partial}{\partial t}}\left\{{}^{e}\bar{\omega}^{b}\times\bar{r}^{o_{b}p}\right\} = {}^{e}\bar{\omega}^{b}\times\overset{e}{\frac{\partial}{\partial t}}\left\{\bar{r}^{o_{b}p}\right\}  +  \overset{e}{\frac{\partial}{\partial t}}\left\{{}^{e}\bar{\omega}^{b}\right\}\times\bar{r}^{o_{b}p}\label{Inter-3}\end{equation}

By conjecture,
$$\overset{e}{\frac{\partial}{\partial t}}\left\{{}^{e}\bar{\omega}^{b}\right\} = \overset{b}{\frac{\partial}{\partial t}}\left\{{}^{e}\bar{\omega}^{b}\right\}$$
This claim is rather simple to prove. 
Simply apply equation $\ref{KTT-1st-form}$ to the claim, 
$$\overset{e}{\frac{\partial}{\partial t}}\left\{{}^{e}\bar{\omega}^{b}\right\} = \overset{b}{\frac{\partial}{\partial t}}[{}^{e}\bar{\omega}^{b}] + {}^{e}\bar{\omega}^{b}\times{}^{e}\bar{\omega}^{b}$$
The cross product of a vector with itself is zero, $\displaystyle {}^{e}\bar{\omega}^{b}\times{}^{e}\bar{\omega}^{b} = 0$ . 
Hence, we can recover our claim,
$$\overset{e}{\frac{\partial}{\partial t}}\left\{{}^{e}\bar{\omega}^{b}\right\} = \overset{b}{\frac{\partial}{\partial t}}[{}^{e}\bar{\omega}^{b}]$$

Applying equation $\ref{KTT-1st-form}$, to $\displaystyle \overset{e}{\frac{\partial}{\partial t}}\left\{\bar{r}^{o_{b}p}\right\}$,
$$\overset{e}{\frac{\partial}{\partial t}}\left\{\bar{r}^{o_{b}p}\right\} = \overset{b}{\frac{\partial}{\partial t}}\left[\bar{r}^{o_{b}p}\right] + {}^{e}\bar{\omega}^{b}\times\left[\bar{r}^{o_{b}p}\right]$$

Susbtituting into equation $\ref{Inter-3}$,
$$\overset{e}{\frac{\partial}{\partial t}}\left\{{}^{e}\bar{\omega}^{b}\times\bar{r}^{o_{b}p}\right\} = {}^{e}\bar{\omega}^{b}\times\left\{\overset{b}{\frac{\partial}{\partial t}}\left[\bar{r}^{o_{b}p}\right] + {}^{e}\bar{\omega}^{b}\times\left[\bar{r}^{o_{b}p}\right]\right\}    + \left\{\overset{b}{\frac{\partial}{\partial t}}[{}^{e}\bar{\omega}^{b}]\right\}\times\bar{r}^{o_{b}p}$$
Expanding,
\begin{equation}\overset{e}{\frac{\partial}{\partial t}}\left\{{}^{e}\bar{\omega}^{b}\times\bar{r}^{o_{b}p}\right\} = {}^{e}\bar{\omega}^{b}\times\left\{\overset{b}{\frac{\partial}{\partial t}}\left[\bar{r}^{o_{b}p}\right]\right\}   +   {}^{e}\bar{\omega}^{b}\times\left\{{}^{e}\bar{\omega}^{b}\times\left[\bar{r}^{o_{b}p}\right]\right\}    +    \left\{\overset{b}{\frac{\partial}{\partial t}}[{}^{e}\bar{\omega}^{b}]\right\}\times\bar{r}^{o_{b}p}\label{whammu-2}\end{equation}

Substituting equation $\ref{whammu-1}$ and $\ref{whammu-2}$ into equation $\ref{inter-4}$,
\begin{align*}\overset{e}{\frac{\partial^{2}}{\partial t^{2}}}[\bar{r}^{o_{e}p}]    =&    \overset{e}{\frac{\partial^{2}}{\partial t^{2}}}[\bar{R}^{o_{e}o_{b}}]     +   \overset{b}{\frac{\partial^{2}}{\partial t^{2}}}[\bar{r}^{o_{b}p}] + {}^{e}\bar{\omega}^{b}\times\overset{b}{\frac{\partial}{\partial t}}[\bar{r}^{o_{b}p}]     +   {}^{e}\bar{\omega}^{b}\times\left\{\overset{b}{\frac{\partial}{\partial t}}\left[\bar{r}^{o_{b}p}\right]\right\}   \\&+   {}^{e}\bar{\omega}^{b}\times\left\{{}^{e}\bar{\omega}^{f}\times\left[\bar{r}^{o_{b}p}\right]\right\}    +    \left\{\overset{b}{\frac{\partial}{\partial t}}[{}^{e}\bar{\omega}^{b}]\right\}\times\bar{r}^{o_{b}p}\end{align*}
Simplifying,
\begin{equation}
\overset{e}{\frac{\partial^{2}}{\partial t^{2}}}[\bar{r}^{o_{e}p}] =
\overset{e}{\frac{\partial^{2}}{\partial t^{2}}}[\bar{R}^{o_{e}o_{b}}]     
+ \overset{b}{\frac{\partial^{2}}{\partial t^{2}}}[\bar{r}^{o_{b}p}] 
+ 2{}^{e}\bar{\omega}^{b}\times\overset{b}{\frac{\partial}{\partial t}}[\bar{r}^{o_{b}p}]
+ {}^{e}\bar{\omega}^{b}\times\left\{{}^{e}\bar{\omega}^{b}\times\left[\bar{r}^{o_{b}p}\right]\right\}    
+ \left\{\overset{b}{\frac{\partial}{\partial t}}[{}^{e}\bar{\omega}^{b}]\right\}\times\bar{r}^{o_{b}p}
\label{Kinematic Transport Theorem Main Result}
\end{equation}

The expression above is a purely kinematic relation. 
If $\displaystyle \bar{r}^{o_{e}p}$ is considered to be the position vector of a particular point $p$, then $\displaystyle \overset{e}{\frac{\partial^{2}}{\partial t^{2}}}[\bar{r}^{o_{e}p}]$ would represent acceleration of point $p$ relative to the origin of reference frame $e$ and perceived in the inertial reference frame $e$. 
$\displaystyle \overset{e}{\frac{\partial^{2}}{\partial t^{2}}}[\bar{R}^{o_{e}o_{b}}]$ represents the acceleration of the origin of reference frame $b$ relative to the origin of reference frame $e$ perceived in the inertial reference frame $e$. 
$\displaystyle \overset{b}{\frac{\partial^{2}}{\partial t^{2}}}[\bar{r}^{o_{b}p}]$ represents the acceleration perceived in non-inertial frame $b$ of point $p$ relative to the origin of frame $b$. 
The next few terms have specific formal names.
%Seperator
\\~\\Equation $\ref{Kinematic Transport Theorem Main Result}$ expresses the acceleration perceived in the inertial $e$ frame of point $p$ relative to origin of frame $e$. 
It is possible to rewrite the equation $\ref{Kinematic Transport Theorem Main Result}$ to instead express the acceleration perceived in the inertial $b$ frame of point $p$ relative to origin of frame $b$,
\begin{equation}\overset{e}{\frac{\partial^{2}}{\partial t^{2}}}[\bar{r}^{o_{e}p}]  -  \overset{e}{\frac{\partial^{2}}{\partial t^{2}}}[\bar{R}^{o_{e}o_{b}}]  -  2{}^{e}\bar{\omega}^{b}\times\overset{b}{\frac{\partial}{\partial t}}[\bar{r}^{o_{b}p}]    -   {}^{e}\bar{\omega}^{b}\times\left\{{}^{e}\bar{\omega}^{b}\times\left[\bar{r}^{o_{b}p}\right]\right\}  -    \left\{\overset{b}{\frac{\partial}{\partial t}}[{}^{e}\bar{\omega}^{b}]\right\}\times\bar{r}^{o_{b}p} =  \overset{b}{\frac{\partial^{2}}{\partial t^{2}}}[\bar{r}^{o_{b}p}]    \label{esidisi2}\end{equation}
%Seperator
\begin{itemize}
\item $\displaystyle   -  \overset{e}{\frac{\partial^{2}}{\partial t^{2}}}[\bar{R}^{o_{e}o_{b}}]$ is often referred to as the fictitious acceleration term. 
Imagine we are in a train and suddenly the train start moving forward and we can feel some kind of "pull" towards the back of the train.
In another example, imagine we are in an elevator and suddenly the elevator starts moving upwards and we feel a strong "pull" towards the ground than usual.
This term appears when the reference frame we are in (either fixed to the elevator or fixed to the train) suddenly starts moving and the origin of our reference frame starts to accelerate.
This is one of the most common fictitious forces we can experience.
%Seperator
\item $\displaystyle   -  2{}^{e}\bar{\omega}^{b}\times\overset{b}{\frac{\partial}{\partial t}}[\bar{r}^{o_{b}p}]$ term is known as the coriolis acceleration.
This is one of the most unintuitive fictitious acceleration to feel. 
If we stand on the outer edge of a moving carousel, face towards the center of the carousel and walk inwards towards the center of the carousel, we would feel some kind of force pushing us sideways.
This is the coriolis force and it is actually used to ensure that angular momentum is conserved (though we have not yet introduced angular momentum, so it's fine if you don't understand this point for now). 
%Seperator
\item $\displaystyle     -   {}^{e}\bar{\omega}^{b}\times\left\{{}^{e}\bar{\omega}^{b}\times\left[\bar{r}^{o_{b}p}\right]\right\}$ term is known as centrifugal acceleration.
This is quite intuitive for anyone who has been in a high speed car ride before. 
When the car turns in a highway while travelling at high enough velocities, we might feel ourselves being pushed "out of the center of rotation", so if our car was turning left, we would be pushed out to the right, meanwhile if our car was turning right, we would be pushed out to the left. 
The centrifugal force is also often times used in the chemical field to separate various liquids by spinning them extremely quickly in a kind of drum.
The "heavier" liquid will tend to stay at the outer parts of the drum meanwhile the lighter liquid will accumulate at the inner parts of the drum.
Unintuitively, the earth's rotation also counteracts out weight somewhat because the centrifugal force has a tendency to push us "up" towards space. 
So in an alternate reality yes, it is possible to have an earth that spins fast enough that we would be "weightless" even at the surface of the earth.
%Seperator
\item $\displaystyle   -    \left\{\overset{b}{\frac{\partial}{\partial t}}[{}^{e}\bar{\omega}^{b}]\right\}\times\bar{r}^{o_{b}p}$ term is known as the azimuthal acceleration. 
Imagine we were in a carousel that was initially not rotating at all.
Suddenly the carousel starts accelerating to its normal rotation rate. 
As the carousel starts accelerating and spinning faster, we would feel a "pull" towards the back of our seats. 
That pull is an example of azimuthal acceleration.
This one is also quite intuitive.
\end{itemize}
%Seperator
The fictitious acceleration is a consequence of the rotational and translational motion of reference frame $b$. 
These fictitious accelerations are not a consequence of physical interactions, but rather a consequence of perception, hence their names. 
If reference frame $e$ and reference frame $b$ are both inertial reference frames, 
$$\overset{e}{\frac{\partial^{2}}{\partial t^{2}}}[\bar{r}^{o_{e}p}] =  \overset{b}{\frac{\partial^{2}}{\partial t^{2}}}[\bar{r}^{o_{b}p}]$$
This would require $2$ things to be simultaneously true,
$$\overset{e}{\frac{\partial^{2}}{\partial t^{2}}}[\bar{R}^{o_{e}o_{b}}] = 0 \quad,\quad {}^{e}\bar{\omega}^{b} = 0$$
Another definition for inertial reference frames are reference frames that exhibit no fictitious acceleration. 
Non-inertial frames are allowed to exhibit fictitious acceleration but inertial reference frames cannot. 
Therefore, the constraint at the beginning that inertial reference frames are allowed to move translationally relative to one another at some constant velocity and are not allowed any rotation with respect to one another.

