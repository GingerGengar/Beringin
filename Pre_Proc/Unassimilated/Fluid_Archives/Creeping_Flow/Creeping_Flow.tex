%Type of Document
\documentclass[a4paper, 12pt]{report}

%Load Pre-ambles
\usepackage{../Environment/Packages}
\usepackage{../Environment/Conventions}
\usepackage{../Environment/Hyahoos}

\begin{document}

\begin{center}
\chapter{Creeping Flows}
\begin{comment}
\end{comment}
Creeping Flows are the opposite of Potential Flows. The effects of viscosity are important but the inertial effects of the fluid is negligible. The degeneracy from the general Navier Stokes equation is shown below,
%Seperator
%Seperator
%Seperator
%Seperator
%Seperator
\section{Cartesian Parallel Flows}
\begin{comment}
\end{comment}
The continuum governing equation for incompressible fluids in cartesian coordinates are shown below,
$$0 = \nabla\d\b{v_f} = \f{\p}{\p x}[\b{v_f}] + \f{\p}{\p y}[\b{v_f}] + \f{\p}{\p z}[\b{v_f}]$$
Ignoring the $y$-direction degenerates this problem to a $2$-dimensional case and evaluates the term $\dst{\f{\p}{\p y}[\b{v_f}]}$ to zero. It is known that this case is a parallel flow case, therefore, the velocity in the $z$-direction is also non-existent. Therefore, the term $\dst{\f{\p}{\p z}[\b{v_f}]}$ evaulates to zero as well.  
$$0 = \f{\p}{\p x}[\b{v_f}]$$
Since this is a $2$-dimensional problem, then the velocity is also independent in the thickness $y$-direction. This also means that the fluid velocity is identical to just the horizontal velocity $\dst{\b{v_f} = \b{v_x}}$. Therefore,
$$0 = \f{\p}{\p x}[\b{v_f}] = \f{\p}{\p x}[\b{v_x}]$$
This equation shows that the velocity is independent in the $x$-direction. 
An alternate version of the Navier-Stokes momentum equation,
$$\rho\l[\f{\p\b{v_f}}{\p t} + (\b{v_f}\d\nabla)\b{v_f}\r] = \rho \b{g} - \nabla P_r + \mu\nabla^2\b{v_f}$$
wherein $\dst{\nabla^2 \b{v_f} = (\nabla \d\nabla)\b{v_f}}$ in this case. Evaluating the Navier-Stokes momentum equation in the $x$-direction,
$$\rho\l[\f{\p\b{v_x}}{\p t} + \b{v_x}\f{\p}{\p x}[\b{v_x}] + \b{v_y}\f{\p}{\p y}[\b{v_x}] + \b{v_z}\f{\p}{\p z}[\b{v_x}]\r] = - \f{\p}{\p x}[P_r] + \mu\l[\f{\p^2}{\p x^2}[\b{v_x}] + \f{\p^2}{\p y^2}[\b{v_x}] + \f{\p^2}{\p z^2}[\b{v_x}]\r]$$
If analysis is performed on a steady state fluid flow, the term $\dst{\f{\p\b{v_x}}{\p t}}$ evaluates to zero. Using the previous finding after applying the continuity governing equation, the term $\dst{\b{v_x}\f{\p}{\p x}[\b{v_x}]}$ evaluates to zero. Due to the $2$-dimensional nature of the problem, the term $\dst{\b{v_y}\f{\p}{\p y}[\b{v_x}]}$ evaluates to zero. Since there is no velocity in the veritcal $z$-direction due to the parallel nature of the flow, the term $\dst{\b{v_z}\f{\p}{\p z}[\b{v_x}]}$ evaluates to zero. Consistent with the previous findings, since the velocity is not a function of horizontal displacement $x$, the term $\dst{\f{\p^2}{\p x^2}[\b{v_x}]}$ evaluates to zero. Due to the $2$-dimensional nature of the problem, the term $\dst{\f{\p^2}{\p y^2}[\b{v_x}]}$ evaluates to zero. Therefore,
$$0 = - \f{\p}{\p x}[P_r] + \mu\f{\p^2}{\p z^2}[\b{v_x}]$$
$$\f{\p}{\p x}[P_r] =  \mu\f{\p^2}{\p z^2}[\b{v_x}]$$
Evaluating the Navier-Stokes momentum equation in the $z$-direction,
$$\rho\l[\f{\p\b{v_z}}{\p t} + \b{v_x}\f{\p}{\p x}[\b{v_z}] + \b{v_y}\f{\p}{\p y}[\b{v_z}] + \b{v_z}\f{\p}{\p z}[\b{v_z}]\r] = -\rho g - \f{\p}{\p z}[P_r] + \mu\l[\f{\p^2}{\p x^2}[\b{v_z}] + \f{\p^2}{\p y^2}[\b{v_z}] + \f{\p^2}{\p z^2}[\b{v_z}]\r]$$
Since the the vertical velocities are zero, then,
$$\f{\p\b{v_z}}{\p t} = \b{v_x}\f{\p}{\p x}[\b{v_z}] = \b{v_y}\f{\p}{\p y}[\b{v_z}] = \b{v_z}\f{\p}{\p z}[\b{v_z}] = \f{\p^2}{\p x^2}[\b{v_z}] = \f{\p^2}{\p y^2}[\b{v_z}] = \f{\p^2}{\p z^2}[\b{v_z}] = 0$$
Therefore,
$$0 = -\rho g - \f{\p}{\p z}[P_r]$$
Suppose one were to attempt to find the pressure function,
$$-\rho g \int\,dz = \int \,dP_r$$
$$-\rho g z + f(x) = P_r$$
wherein $f(x)$ is a function purely in terms of $x$. Taking the second conclusion we obtained from applying the Navier-Stokes momentum equation in the $x$-direction,
$$\f{\p}{\p x}[-\rho g z + f(x)] =  \mu\f{\p^2}{\p z^2}[\b{v_x}]$$
$$f'(x) =  \mu\f{\p^2}{\p z^2}[\b{v_x}]$$
The velocity $\b{v_x}$ is only dependent on the $z$-direction. Therefore, the second order partial derivative term, $\dst{\mu\f{\p^2}{\p z^2}[\b{v_x}]}$ must also be fully dependent on the $z$-direction. This means that the function $f'(x)$ is a constant function $f'(x) = c_1$, therefore, $f(x)$. Firstly this computes the pressure function,
$$P_r = -\rho g z + c_1 x + \alpha$$ 
wherein both $c_1$ and $\alpha$ are arbitrary constsants. Secondly, the implications for the velocity profile,
$$f'(x) = c_1 = \mu\f{\p^2}{\p z^2}[\b{v_x}]$$
$$\f{1}{\mu}\int c_1\,dz = \f{c_1}{\mu}z + c_2 = \f{\p}{\p z}[\b{v_x}]$$
$$\f{c_1}{\mu}\int z\,dz + \int c_2\,dz = \b{v_x}$$
$$\f{c_1}{2\mu} z^2 + c_2z +c_3 = \b{v_x}$$
%Seperator
%Seperator
%Seperator
%Seperator
%Seperator
\section{Radial Parallel Flows}
\begin{comment}
\end{comment}
The continuum governing equation is shown below,
$$0 = \nabla\d\b{v_f} = \f{\p u}{\p x} + \f{\p v}{\p y} + \f{\p w}{\p z} = \f{\p v_r}{\p r} + \f{1}{r}\f{\p v_\t}{\p\t} + \f{\p v_z}{\p z}$$
The Navier-Stokes Equation is shown below,
$$\rho\l[\f{\p\b{v_f}}{\p t} + (\b{v_f}\d\nabla)\b{v_f}\r] = \rho \b{g} - \nabla P + \mu\nabla^2\b{v_f}$$
%Seperator
%Seperator
%Seperator
%Seperator
\subsection{Velocity Profile}
\begin{comment}
\end{comment}
To find the velocity profile of the fluid in the radial direction, the continuum governing equation is applied first in polar coordiantes.
$$0 = \nabla\d\b{v_f} = \f{\p v_r}{\p r} + \f{1}{r}\f{\p v_\t}{\p\t} + \f{\p v_z}{\p z}$$
Ignoring the $z$-axis,
$$0 = \f{\p v_r}{\p r} + \f{1}{r}\f{\p v_\t}{\p\t}$$
By inspection it could be seen that the radial velocity is zero, and $\dst{v_r = 0}$. Therefore,
$$0 = \f{\p v_\t}{\p\t}\quad,\quad 0 = \f{\p^2 v_\t}{\p\t^2}$$
This shows that the tangential velocity is purely a function of radius and not a function of $\t$. It is Assumed that gravity acts on the $z$-direction and is largely ignored. 
Evaluating the expression in the radial direction,
$$\rho\l[\f{dv_r}{dt} + v_r\f{\p v_r}{\p r} + v_\t\f{1}{r}\f{\p v_r}{\p\t} - \f{v_\t^2}{r}\r] = - \f{\p P}{\p r} + \mu\l[\f{1}{r}\f{\p}{\p r}\l(r\f{\p v_r}{\p r}\r) + \f{1}{r^2}\f{\p^2 v_r}{\p \t^2} - \f{v_r}{r^2} - \f{2}{r^2}\f{\p v_\t}{\p\t}\r]$$
Since radial velocity is always zero, $\dst{\f{dv_r}{dt} = v_r\f{\p v_r}{\p r} = v_\t\f{1}{r}\f{\p v_r}{\p\t} = \f{1}{r^2}\f{\p^2 v_r}{\p \t^2} = \f{v_r}{r^2} = 0}$. Since tangential velocity is not a function of $\t$, $\dst{\f{2}{r^2}\f{\p v_\t}{\p\t} = 0}$
$$- \f{\rho v_\t^2}{r} = - \f{\p P}{\p r}$$
$$\f{\rho v_\t^2}{r} = \f{\p P}{\p r}$$
Evaluating the expression in the tangential direction,
$$\rho\l[\f{dv_\t}{dt} + v_r\f{\p v_\t}{\p r} + v_\t\f{1}{r}\f{\p v_\t}{\p\t} + \f{v_rv_\t}{r}\r] = - \f{1}{r}\f{\p P}{\p\t} + \mu\l[\f{1}{r}\f{\p}{\p r}\l(r\f{\p v_\t}{\p r}\r) + \f{1}{r^2}\f{\p^2 v_\t}{\p \t^2} - \f{v_\t}{r^2} + \f{2}{r^2}\f{\p v_r}{\p\t}\r]$$
$\dst{\f{dv_\t}{dt} = 0}$ because it is assumed that the system is already in a steady state. $\dst{v_r\f{\p v_\t}{\p r} = \f{2}{r^2}\f{\p v_r}{\p\t}  = \f{v_rv_\t}{r}= 0}$ because $v_r = 0$ as the previous assumption. Since tangential velocity is purely a function of radius, $\dst{\f{1}{r^2}\f{\p^2 v_\t}{\p \t^2} = v_\t\f{1}{r}\f{\p v_\t}{\p\t} = 0}$. Therefore, the equation degenerates into,
$$0 = - \f{1}{r}\f{\p P}{\p\t} + \mu\l[\f{1}{r}\f{\p}{\p r}\l(r\f{\p v_\t}{\p r}\r) - \f{v_\t}{r^2}\r]$$
By a seperate mathematical proof,
$$\f{1}{r}\f{\p}{\p r}\l(r\f{\p v_\t}{\p r}\r) - \f{v_\t}{r^2} = \f{\p}{\p r}\l[\f{1}{r}\f{\p}{\p r}(rv_\t)\r]$$
Therefore,
$$0 = - \f{1}{r}\f{\p P}{\p\t} + \mu\f{\p}{\p r}\l[\f{1}{r}\f{\p}{\p r}(rv_\t)\r]$$
By the relation found by applying the Incompressible Navier-Stokes on the radial direction, it could be seen that pressure $\dst{P}$ is dependent on tangential velocity $v_\t$ and $r$. However, since tangential velocity is only dependent on $r$ from the relation found by applying continuity, then it follows that pressure must only be dependent on $r$. Therefore, $\dst{\f{\p P}{\p r} = 0}$. Therefore, the Navier-Stokes applied on the tangential direction further degenerates into,
$$0 = \f{\p}{\p r}\l[\f{1}{r}\f{\p}{\p r}(rv_\t)\r]$$
Solving for tangential velocity as a function of radius,
$$c_\a = \f{1}{r}\f{\p}{\p r}(rv_\t)$$
$$\int^{r}_{r=R_1}rc_\a\,dr = \int^{r,v}_{r=R_1,v_\t = \w_1R_1}drv_\t$$
$$\f{1}{2}c_\a\l[r^2\r]^{r}_{r=R_1} = \l[rv_\t\r]^{r,v_\t}_{r=R_1,v_\t = \w_1R_1}$$
$$\f{1}{2}c_\a\l[r^2 - R_1^2\r] = \l[rv_\t - \w_1R_1^2\r]$$
When $r = R_2$, $v_\t = \w_2R_2$. Therefore,
$$\f{1}{2}c_\a\l[R_2^2 - R_1^2\r] = \l[R_2\w_2R_2 - \w_1R_1^2\r]$$
$$c_\a\l[R_2^2 - R_1^2\r] = 2\l[\w_2R_2^2 - \w_1R_1^2\r]$$
$$c_\a = \f{2\l[\w_2R_2^2 - \w_1R_1^2\r]}{\l[R_2^2 - R_1^2\r]}$$
By substituting the newly found definition for the constant $c_\a$,
$$\f{\l[\w_2R_2^2 - \w_1R_1^2\r]}{\l[R_2^2 - R_1^2\r]}  \l[r^2 - R_1^2\r] = rv_\t - \w_1R_1^2$$
$$\f{\l[\w_2R_2^2 - \w_1R_1^2\r]}{\l[R_2^2 - R_1^2\r]}  \l[r - \f{R_1^2}{r}\r] + \f{\w_1R_1^2}{r} = v_\t$$
$$\f{\l[\w_2R_2^2 - \w_1R_1^2\r]}{\l[R_2^2 - R_1^2\r]}  \l[r - \f{R_1^2}{r}\r] + \f{\w_1R_1^2\l[R_2^2 - R_1^2\r]}{r\l[R_2^2 - R_1^2\r]} = v_\t$$
$$\f{\l[\w_2R_2^2 - \w_1R_1^2\r]}{\l[R_2^2 - R_1^2\r]}  \l[r - \f{R_1^2}{r}\r] + \f{\l[\w_1R_1^2R_2^2 - \w_1R_1^4\r]}{r\l[R_2^2 - R_1^2\r]} = v_\t$$
$$\f{1}{\l[R_2^2 - R_1^2\r]}  \l[r\l(\w_2R_2^2 - \w_1R_1^2\r) - \f{R_1^2\l(\w_2R_2^2 - \w_1R_1^2\r) - \l[\w_1R_1^2R_2^2 - \w_1R_1^4\r]}{r}\r] = v_\t$$
$$\f{1}{\l[R_2^2 - R_1^2\r]}  \l[r\l(\w_2R_2^2 - \w_1R_1^2\r) - \f{\w_2R_1^2R_2^2 - \w_1R_1^4 - \w_1R_1^2R_2^2 + \w_1R_1^4}{r}\r] = v_\t$$
$$v_\t = \f{1}{\l[R_2^2 - R_1^2\r]}  \l[r\l(\w_2R_2^2 - \w_1R_1^2\r) - \f{R_1^2R_2^2\l[\w_2 - \w_1\r]}{r}\r]$$
%Seperator
%Seperator
%Seperator
%Seperator
\subsection{Scalar Pressure Field}
\begin{comment}
\end{comment}
To find the pressure field of the fluid, the two relations that are obtained from applying the incompressible Navier-Stokes equation on the radial and tangential direction is used,
$$\f{\rho v_\t^2}{r} = \f{\p P}{\p r}\quad,\quad0 = \f{\p}{\p r}\l[\f{1}{r}\f{\p}{\p r}(rv_\t)\r]$$
The second relation when further operated on gives the function of tangential velocity $v_\t$ with respect to radius $r$. Integrating the first relation with respect to $r$ and applying the boundary conditions yields,
\begin{equation*}
\begin{split}
P =& P_1 + \f{\rho}{(R_2^2-R_1^2)^2}\l[(\w_2R_2^2 - \w_1R_1^2)^2\l(\f{r^2 - R_1^2}{2}\r)\r.{}\\
- &\l.{} 2R_1^2R_2^2(\w_2-\w_1)(\w_2R_2^2 - \w_1R_1^2)\ln\f{r}{R_1} - \f{R_1^4R_2^4}{2}(\w_2-\w_1)^2\l(\f{1}{r^2} - \f{1}{R_1^2}\r)\r]
\end{split}
\end{equation*}
%Seperator
%Seperator
%Seperator
%Seperator
\subsection{Induced Torque}
\begin{comment}
\end{comment}
Suppose the outer cylinder is rotating and the inner cylinder is held at rest, the equation for torque exerted on the inner cylinder must be
$$\Gamma = 2\pi\mu\w_2R_1^2\l[\f{R_2H}{R_2-R_1} + \f{R_1^2}{4b}\r]$$
wherein $b$ represents the gap between the two cylindrical surfaces and $\Gamma$ is the torque exerted on the inner cylinder. Since this is typically ignored,
$$\Gamma = 2\pi\mu\w_2R_1^2\l[\f{R_2H}{R_2-R_1}\r]$$
The torque per unit width length,
$$\Gamma_H = 2\pi\mu\w_2R_1^2\l[\f{R_2}{R_2-R_1}\r]$$
This is very useful to test for the viscosity of a specific fluid.
\end{center}

\end{document}
