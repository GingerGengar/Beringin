%Type of Document
\documentclass[a4paper, 12pt]{report}

%Load Pre-ambles
\usepackage{../Environment/Packages}
\usepackage{../Environment/Conventions}
\usepackage{../Environment/Hyahoos}

\begin{document}

\begin{center}
%Seperator
%Seperator
%Seperator
%Seperator
%Seperator
%Seperator
\chapter{Reynold's Transport Theorem}
\begin{comment}
Liebnitz Theorem

In the middle of additions:


\end{comment}
One variation of Liebniz Rule applicable for volumetric integrals is shown below. for the variable $T$ wherein $T$ may represent a time dependent scalar, vector, or tensor.
$$\f{d}{dt}\iiint_{R(t)}T\,dV_o = \iiint_{R(t)}\f{\p}{\p t}[T]dV_o + \iint_{S(t)}T\b{v_s}\b{n}dS$$
wherein $R(t)$ represents an arbitray region of space, $V_o$ represents volume, $S(t)$ represents the surface of the region defined by $R(t)$, $\b{v_s}$ represents the velocity of the moving surface, $\b{n}$ represents normal vector of the surface. Depending on the variable type $T$, the operation $\dst{T\b{v_s}\b{n}}$ would depend on a case to case basis.

Using a change of variables,
$$\frac{d}{dt}\iiint^{}_{V_{s}(t)}\phi \,dV_{o} = \frac{d}{dt}\iiint^{}_{\Gamma} \phi J \,dV_{o,i}$$
wherein $V_{s}(t)$ represents a control mass region, $\phi$ may represent some time changing scalar variable, but in general, could represent the elements of an arbitrary tensor as well. $dV_{o}$ represents infinitesmial volume. Since the vector field is evolving with time, all the points inside $V_{s}(t)$ is at some place initially at time $t = 0$. The region that contains all the points inside $V_{s}(t)$ at time $t = 0$ is considered to be $\Gamma$. Since we are considering the general case wherein volume may expand or contract, we declare $dV_{o,i} $ to represent infinitesmial volume initially at time $t = 0$. The relationship between infinitesmial volume at the present time $dV_{o}$ and infinitesmial volume initially,
$$dV_{o} = J\,dV_{o,i}$$
wherein $J$ represents the Jacobian, which is the determinant of the velocity gradient tensor (more on this later). From all these information, the change of variables could be performed as shown above. $\epsilon$ is a region that is not varying with time $t$. Since the bounds of integration is now unchanging with time, the time derivative operation is now commutative with the volumetric integral. Therefore,
$$\frac{d}{dt}\iiint^{}_{V_{s}(t)}\phi \,dV_{o} = \iiint^{}_{\Gamma}\frac{d}{dt}[\phi J] \,dV_{o,i}$$
Without loss of generality, assuming that $\phi$ changes with coordinates $x_{i}$ and time, the time derivative is equivalent to the substantive or material derivative. Therefore,
$$\frac{d}{dt}\iiint^{}_{V_{s}(t)}\phi \,dV_{o} = \iiint^{}_{\Gamma}\frac{D}{Dt}[\phi J] \,dV_{o,i}$$
Using product rule,
$$\frac{d}{dt}\iiint^{}_{V_{s}(t)}\phi \,dV_{o} = \iiint^{}_{\Gamma}\phi\frac{D}{Dt}[J] + J\frac{D}{Dt}[\phi] \,dV_{o,i}$$
By a tedious mathematical proof,
$$\frac{D}{Dt}[J] = (\nabla \cdot\bar{v}_{s})J$$
wherein $\bar{v_{s}}$ represents the velocity of the moving surface. $\bar{v}_{s}$ is not to be confused with $V_{s}$. $V_{s}$ represents the control mass region earlier meanwhile $\bar{v}_{s}$ represents the velocity of the moving boundaries of $V_{s}$. Substituting,
$$\frac{d}{dt}\iiint^{}_{V_{s}(t)}\phi \,dV_{o} = \iiint^{}_{\Gamma}\phi(\nabla \cdot\bar{v}_{s})J + J\frac{D}{Dt}[\phi] \,dV_{o,i}$$
Making a change of variables once again to revert back to the region $V_{s}(t)$ from the initial positions $\Gamma$,
$$\frac{d}{dt}\iiint^{}_{V_{s}(t)}\phi \,dV_{o} = \iiint^{}_{\Gamma}\left\{\phi(\nabla \cdot\bar{v}_{s}) + \frac{D}{Dt}[\phi]\right\} \,JdV_{o,i}$$
$$\frac{d}{dt}\iiint^{}_{V_{s}(t)}\phi \,dV_{o} = \iiint^{}_{V_{s}(t)} \phi(\nabla \cdot\bar{v}_{s}) + \frac{D}{Dt}[\phi]  \,dV_{o}$$
Expanding the substantive derivative of $\phi$ as $\displaystyle \frac{D}{Dt}[\phi] = \frac{\partial}{\partial t}[\phi] + \bar{v_{s}}\cdot\nabla\phi$,
$$\frac{d}{dt}\iiint^{}_{V_{s}(t)}\phi \,dV_{o} = \iiint^{}_{V_{s}(t)} \phi(\nabla \cdot\bar{v}_{s}) + \frac{\partial}{\partial t}[\phi] + \bar{v_{s}}\cdot\nabla\phi \,dV_{o}$$
Using the divergence of scalar vector product identity,
$$\nabla \cdot(\phi\bar{v}_{s}) = \phi(\nabla \cdot\bar{v}_{s}) + \bar{v_{s}}\cdot\nabla\phi$$
Substituting,
$$\frac{d}{dt}\iiint^{}_{V_{s}(t)}\phi \,dV_{o} = \iiint^{}_{V_{s}(t)} \frac{\partial}{\partial t}[\phi] + \nabla \cdot(\phi\bar{v}_{s}) \,dV_{o}$$
Parsing out the integral,
$$\frac{d}{dt}\iiint^{}_{V_{s}(t)}\phi \,dV_{o} = \iiint^{}_{V_{s}(t)} \frac{\partial}{\partial t}[\phi] \,dV_{o}  +  \iiint^{}_{V_{s}(t)}\nabla \cdot(\phi\bar{v}_{s}) \,dV_{o}$$
Using divergence theorem, $\displaystyle \iiint^{}_{V_{s}(t)}\nabla \cdot(\phi\bar{v}_{s}) \,dV_{o} = \iint^{}_{S_{s}(t)}\phi\bar{v}_{s} \cdot\,\hat{n}dS$. Substituting,
%Shouldnt we be defining a new generalized inner product operator? What happens if phi is a tensor, then phi v_s is not even defined, neither is phi v_s dot n dS.
$$\frac{d}{dt}\iiint^{}_{V_{s}(t)}\phi \,dV_{o} = \iiint^{}_{V_{s}(t)} \frac{\partial}{\partial t}[\phi] \,dV_{o}  +  \iint^{}_{S_{s}(t)}\phi\bar{v}_{s} \cdot\,\hat{n}dS$$
%Seperator
%Seperator
%Seperator
%Seperator
%Seperator
\section{Substantive Derivative}
\begin{comment}
Substantive Derivative
Needs Serious revamping
\end{comment}
Suppose a quantity $b$ is dependent on the the variable time $t$ and the typical cartesian coordinates $x$, $y$, $z$. Taking the derivative of variable $b$ with respect to time yields the following based on chain rule,
$$\f{d}{dt}[b] = \f{\p}{\p t}[b] + \f{\p}{\p x}[b]\times\f{\p}{\p t}[x] + \f{\p}{\p y}[b]\times\f{\p}{\p t}[y] + \f{\p}{\p z}[b]\times\f{\p}{\p t}[z]$$
Taking note that the partial derivatives of the cartesian coordinates defines velocity in the cartesian coordinates. Therefore, 
$$\f{\p}{\p t}[x] = u\quad,\quad \f{\p}{\p t}[y] = v\quad,\quad \f{\p}{\p t}[z] = w$$
wherein $u$, $v$, and $w$ typically represents velocity in the $x$, $y$, and $z$ directions respectively. Therefore, the derivative of $y$ with respect to time $t$ would take the form,
$$\f{d}{dt}[b] = \f{\p}{\p t}[b] + u\f{\p}{\p x}[b] + v\f{\p}{\p y}[b] + w\f{\p}{\p z}[b]$$
If the $\nabla$ operator is defined as
$$\nabla = \begin{pmatrix} \dst{\f{\p}{\p x}}& \dst{\f{\p}{\p y}}& \dst{\f{\p}{\p z}} \end{pmatrix}^T$$
Therefore, the derivative of $y$ with respect to time $t$ would take the form
$$\f{d}{dt}[b] = \f{\p}{\p t}[b] + u\f{\p}{\p x}[b] + v\f{\p}{\p y}[b] + w\f{\p}{\p z}[b]$$
Let the velocity vector be defined as 
$$\b{v} = \begin{pmatrix}u\\v\\w\\ \end{pmatrix}$$
It follows that the derivative of $b$ with respect to time $t$ would take the form
$$\f{d}{dt}[b] = \f{\p}{\p t}[b] + \b{v}\d\nabla b$$
%Seperator
%Seperator
%Seperator
%Seperator
%Seperator
\section{Divergence Theorem}
\begin{comment}
Divergence Theorem
\end{comment}
The Divergence Theorem is stated below. The variable $\b{F}$ must represent a vector in $R^3$
$$\iiint_{R(t)}  \nabla \d \b{F} \,dV_o = \iint_{S(t)}\b{F}\d \b{n}dS$$
Alternately,
$$\iiint_{R(t)}  \f{\p}{\p x}[\b{F}] + \f{\p}{\p y}[\b{F}] + \f{\p}{\p z}[\b{F}] \,dV_o = \iint_{S(t)}\b{F}\d \b{n}dS$$
wherein $dV_o$ represents an infinitesmially small volume. $S(t)$ is the surface encapsulating the region $R(t)$. $\b{n}$ is the normal vector of the control volume, and $dS$ is an infinitesmial area of surface $S(t)$.
\end{center}

\end{document}
