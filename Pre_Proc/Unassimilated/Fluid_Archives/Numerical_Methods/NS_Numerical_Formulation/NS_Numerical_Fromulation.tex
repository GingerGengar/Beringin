%Type of Document
\documentclass[a4paper, 12pt]{report}

%Load Pre-ambles
\usepackage{../../Environment/Packages}
\usepackage{../../Environment/Conventions}
\usepackage{../../Environment/Hyahoos}

\begin{document}

\begin{center}

%Seperator
%Seperator
%Seperator
%Seperator
%Seperator
\section{}
\begin{comment}
\end{comment}

%Seperator
%Seperator
%Seperator
%Seperator
\subsection{}
\begin{comment}
\end{comment}

%Seperator
%Seperator
%Seperator
%Seperator
%Seperator
\section{Numerical Formulation of Navier Stokes}
\begin{comment}
\end{comment}

%Seperator
%Seperator
%Seperator
%Seperator
\subsection{Conservation Equations}
\begin{comment}
\end{comment}
Conservation of mass is shown below,
$$\frac{\partial}{\partial t}[\rho] + \nabla \cdot(\rho \bar{v_{f}}) = 0$$
Conservation of linear momentum,
$$\frac{\partial}{\partial t}(\rho \bar{v_{f}}) + \nabla \cdot(\rho \bar{v_{f}}\bar{v_{f}}) = -\nabla P_{r} + \nabla \cdot \bar{\bar{\tau}} + \bar{f}$$
Conservation of energy,
$$\frac{\partial}{\partial t}\left[\rho\left(e+\frac{1}{2}|v_{f}|^{2}\right)\right] + \nabla \cdot\left[\rho \bar{v_{f}}\left(e + \frac{1}{2}|\bar{v_{f}}|^{2}\right)\right] = -\nabla \cdot(\rho\bar{v_{f}}) + \nabla \cdot(\bar{\bar{\tau}}\cdot \bar{v_{f}}) - \nabla \cdot\bar{q} + \bar{f}\cdot\bar{v_{f}} + Q$$

%Seperator
%Seperator
%Seperator
%Seperator
\subsection{Viscous Stress Relations}
\begin{comment}
\end{comment}
The viscous stress tensor is symmetric. Using the Stokes' hypothesis, the viscous stress tensor,
$$\tau_{ij} = \mu\left(\frac{\partial v_{i}}{\partial x_{j}} + \frac{\partial v_{j}}{\partial x_{i}}\right) - \frac{2}{3}\mu\frac{\partial v_{k}}{\partial x_{k}}\delta_{ij}$$
wherein dynamic viscosity $\mu$ is largely a function of temperature,
$$\mu = \mu(T)$$

%Seperator
%Seperator
%Seperator
%Seperator
\subsection{Empirical Assumptions}
\begin{comment}
\end{comment}
Fourier's heat conduction law,
$$q_{i} = -k_{f}\frac{\partial T}{\partial x_{i}}$$
wherein the conductivity of the fluid $k_{f}$ may be a function of temperature,
$$k_{f} = k_{f}(T)$$
The last relation needed is the ideal gas law,
$$P_{r} = (\gamma-1)\rho e\quad,\quad T = (\gamma-1)e/R$$

%Seperator
%Seperator
%Seperator
%Seperator
\subsection{Vector form}
\begin{comment}
\end{comment}
In $3$-dimensions, there are $5$ equations for the Navier-stokes, $1$ continuity, $3$ for momentum in each direction $x_{1}$, $x_{2}$ and $x_{3}$ and the last $1$ for conservation of energy. The Navier-Stokes equation in vector form can be rewritten as follows,
$$\frac{\partial U}{\partial t} + \frac{\partial E}{\partial x} + \frac{\partial F}{\partial y} + \frac{\partial G}{\partial z} = \frac{\partial E_{v}}{\partial x} + \frac{\partial F_{v}}{\partial y} + \frac{\partial G_{v}}{\partial z} + S$$
wherein the solution vector $U$ is shown below,
$$U = \begin{bmatrix}
\rho \\ \rho v_{1} \\ \rho v_{2} \\ \rho v_{3} \\ E_{t}
\end{bmatrix}$$
wherein $E_{t}$ is defined as,
$$E_{t} = \rho e + \frac{1}{2}\rho(v_{1}^{2}+v_{2}^{2}+v_{3}^{2})$$
To obtain the velocities in $3$ dimensions, 
$$v_{1} = \frac{\rho v_{1}}{\rho} \quad,\quad v_{2} = \frac{\rho v_{2}}{\rho} \quad,\quad v_{3} = \frac{\rho v_{3}}{\rho}$$
Susbtituting in terms of the elements of the solution vector,
$$v_{1} = \frac{U_{2}}{U_{1}} \quad,\quad v_{2} = \frac{U_{3}}{U_{1}} \quad,\quad v_{3} = \frac{U_{4}}{U_{1}}$$
The vectors $E$, $F$ and $G$ are shown below,
$$E = \begin{bmatrix}
\rho v_{1} \\ \rho v_{1}^{2} + P_{r} \\ \rho v_{1}v_{2} \\ \rho v_{1}v_{3} \\ (E_{t}+P_{r})v_{1}
\end{bmatrix}\quad,\quad F = \begin{bmatrix}
\rho v_{2} \\ \rho v_{1}v_{2} \\ \rho v_{2}^{2} + P_{r} \\ \rho v_{2}v_{3} \\ (E_{t}+P_{r})v_{2}
\end{bmatrix}\quad,\quad G = \begin{bmatrix}
\rho v_{3} \\ \rho v_{1}v_{3} \\ \rho v_{2}v_{3} \\ \rho v_{3}^{2} + P_{r} \\ (E_{t} + P_{r})v_{3}
\end{bmatrix}$$
The vectors $E_{v}$, $F_{v}$ and $G_{v}$ are defined below,
$$E_{v} = \begin{bmatrix}
0 \\ \tau_{11} \\ \tau_{21} \\ \tau_{31} \\ 
v_{1}\tau_{11} + v_{2}\tau_{12} + v_{3}\tau_{13} + g_{1}
\end{bmatrix} \quad,\quad F_{v} = \begin{bmatrix}
0 \\ \tau_{12} \\ \tau_{22} \\ \tau_{32} \\
v_{1}\tau_{21} + v_{2}\tau_{22} + v_{3}\tau_{23} + g_{2}
\end{bmatrix}$$
$$G_{v} = \begin{bmatrix}
0 \\ \tau_{13} \\ \tau_{23} \\ \tau_{33} \\
v_{1}\tau_{31} + v_{2}\tau_{32} + v_{3}\tau_{33} + g_{3}
\end{bmatrix}$$
$$S = 0$$

%Seperator
%Seperator
%Seperator
%Seperator
\subsection{}
\begin{comment}
\end{comment}

%Seperator
%Seperator
%Seperator
%Seperator
%Seperator
\end{center}

\end{document}
