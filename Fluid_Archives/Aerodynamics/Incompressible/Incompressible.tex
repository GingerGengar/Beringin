%Type of Document
\documentclass[class=report, 12pt, crop=false]{standalone}

%Load Pre-ambles
\usepackage{../Environment/Packages}
\usepackage{../Environment/Conventions}
\usepackage{../Environment/Hyahoos}

\begin{document}

\begin{center}
\chapter{Inviscid Incompressible Flow}
\begin{comment}
\end{comment}
%Seperator
%Seperator
%Seperator
%Seperator
%Seperator
\stdsection{Thin Airfoil Theory}
\begin{comment}
\end{comment}
An infintely long wing could be considered a $2$-dimensional problem. The thin airfoil theory also incorporates all vortices generated by the viscous boundary layer as a line of free-stream vortices on the mean chamber line of the airfoil. The vortices extend parallel in the spanwise direction. The thin airfoil theory differentiates between different geometries of an airfoil.
%Seperator
%Seperator
%Seperator
%Seperator
\subsection{Derivation and Theorem-Specific Coefficients}
\begin{comment}
\end{comment}
A coordinate change is used in this theorem to express $x$ in terms of $\t_d$ wherein $x$ represents the horizontal distance from the leading edge of the airfoil and $\t_d$ is the change of coordinates variable that is used to represnt $x$. Here, $c$ represents the chord length of the airfoil. The relationship between $x$ and $\t_d$ is shown below,
$$x = \f{c}{2}[1-\cos(\t_d)]\quad,\quad 0\leq \t_d \leq \pi$$
Conversely $\t_d$ in terms of $x$,
$$\f{2x}{c} = 1-\cos(\t_d)$$
$$\cos(\t_d)= 1-\f{2x}{c}$$
$$\t_d= \arccos\l[1-\f{2x}{c}\r]\quad,\quad 0\leq x \leq c$$
The coefficients $A_0$ and $A_n$ are shown below,
$$A_0 = \a -\f{1}{\pi}\int^{\pi}_{0}\f{dz}{dx}d\t_d\quad,\quad A_n = \f{2}{\pi}\int^{\pi}_{0}\f{dz}{dx}\cos(n\t_d)\,d\t_d$$
wherein $\a$ represents angle of attack of the specific airfoil.
%Seperator
%Seperator
%Seperator
%Seperator
\subsection{Lift}
\begin{comment}
\end{comment}
The equation for the coefficient of lift $C_L$ is shown below,
$$C_L = 2\pi\l(A_0 + \f{A_1}{2}\r) = 2\pi\l(\a-\a_0\r)$$
wherein $\a_0$ represents the angle of attack for the airfoil when lift is zero.
%Seperator
%Seperator
%Seperator
%Seperator
\subsection{Moments}
\begin{comment}
\end{comment}
The moments about the aerodynamic center of the airfoil,
$$C_{M,ac} = C_{M,c/4} = \f{\pi}{4}(A_2 - A_1)$$
wherein $C_M$ represent the moments coefficient and the subscript $ac$ represents at the aerodynamic center. Since the aerodynamic center is assumed to be quarter chord of the airfoil, the coefficients of moments at the aerodynamic center is identical to the coefficient of moments at the quarter chord of the airfoil. The coefficients $A_1$ and $a_2$ are obtainable via the equations above.
%Seperator
%Seperator
%Seperator
%Seperator
\subsection{Limitations}
\begin{comment}
\end{comment}
The thin airfoil theori fails at thicker airfols, also, could not predict drag. The theory also fails to predict stall since the airflow is assumed to be inviscid. The theorem can only be used to find lift and pitching coefficient.
%Seperator
%Seperator
%Seperator
%Seperator
\subsection{Other Implications}
\begin{comment}
\end{comment}
%Seperator
%Seperator
%Seperator
%Seperator
%Seperator
\section{Lifting Line Theory}
\begin{comment}
\end{comment}
An early attempt at describing the behaviours of a finite wing. Uniquely the lifting line theory incorporates free-stream vortices in a 'u'-shape. Firstly in the configuration of a horse-shoe vortex , and in more advanced interpretation,s as a bunch of horse-shoe vortices superimposed on top of each other along the 'lifting-line'. The lifting line theory differentiates could only take into account the total effect of airfoil twist, geometry and perhaps deployed flaps instead of each individual effect. The total factor of those effects affect the free-stream vortex distribution along the 'lifting-line'. 
%Seperator
%Seperator
%Seperator
%Seperator
\subsection{Lift}
\begin{comment}
\end{comment}
Based on the Kutta-Joukowski theorem,
$$\f{dL}{dy} = \rho_\infty v_\infty \Gamma(y)$$
wherein $\rho_\infty$ represents the free stream density of the airflow, $v_\infty$ represents free stream velocity and $\Gamma(y)$ represents the vortex strength at some point along the wing. Integrating the expression above,
$$L = \rho_\infty v_\infty\int^{b/2}_{-b/2}\Gamma(y)\,dy$$
wherein $b$ represents the full wingspan. The coordinate as per usual is nested on the root of the wing at the leading edge of the main wing. The $x$-direction is prallel to the chord line, the $y$-direction is parallel ot the span length, and $z$-direction is vertical.
%Seperator
%Seperator
%Seperator
%Seperator
\subsection{Drag}
\begin{comment}
\end{comment}
Since the fluid is inviscid, there is no drag due to viscosity, or flow seperation. In the lifting line theory, then the drag is purely induced drag. Induced drag is caused by the "downwash" of the wing causing a slightly altered angle of attack. This causes the lift force to tilt backwards slightly, contributing to drag. Since this phenomenon only exists due to the existence of downwash which in turn is due to lift generation, induced drag is also known as lift-induced drag. The formula for induced drag is below,
$$D_i = \rho_\infty v_\infty\int^{b/2}_{-b/2}\Gamma(y)\a_i(y)\,dy$$
wherein the induced angle of attack $\a_i$ is determined by the downwash and free stream velocity. The downash in turn is determined by integrating the vortex distribution via biot-savart law.
$$\a_i(y) = \arctan\l[\f{-w(y)}{v_\infty}\r]$$
Due to the small angle approximation or taking the first order taylor approximation, $\dst{\lim_{x\to0}[\tan(x)] = x}$. By the small angle, approximation, then,
$$\a_i(y) \approx -\f{w(y)}{v_\infty}$$
Substituting for the definition of downwash $w(y)$,
$$w(y) = -\f{1}{4\pi}\int^{b/2}_{-b/2}\f{1}{y-y_d}\f{d\Gamma}{dy}\, dy_d$$
wherein $y_d$ represents a dummy variable that is used for integration purposes only.
$$\a_i(y) \approx \f{1}{4\pi v_\infty}\int^{b/2}_{-b/2}\f{1}{y-y_d}\f{d\Gamma}{dy}\, dy_d$$
%Seperator
%Seperator
%Seperator
%Seperator
\subsection{Elliptical Vortex Distribution Wing}
\begin{comment}
\end{comment}
By using the vortex distribution,
$$\Gamma(y) = \Gamma_0\sqrt{\dst{1-\l(\f{2y}{b}\r)^2}}$$
The results of performing the steps above and performing the relevant substitutions,
$$\a_i = \f{\Gamma_0}{2bv_\infty} \quad,\quad C_L = \a_i \pi A_R \quad,\quad C_{Di} = \f{C_L^2}{\pi A_R}$$
wherein $A_R$ represents aspect ratio, and $b$ represents total wingspan.
%Seperator
%Seperator
%Seperator
%Seperator
%Seperator
\section{Static Longitudinal Stability}
\begin{comment}
\end{comment}
The coefficient of moments about the center of mass $C_{M,cm}$ is shown below,
$$C_{M,cm} = C_{M,0} + C_{M,\a}\a_a$$
wherein $C_{M,0}$ represents the coefficeint of moments about the center of mass when the absolute angle of attack is zero. $\a_a$ represents the absolute angle of attack, which is the angle of attack of the airfoil starting at zero from the airfoil producing zero lift. $C_{M,\a}$ represents the derivative of $C_{M,cm}$ with respect to the absolute angle of attack $\a_a$. There are only 2 conditions for static longitudinal stability:
\begin{enumerate}
\centering
\item $C_{M,0}$ must be a positive value 
\item $C_{M,\a}$ must be a negative value
\end{enumerate}
\end{center}

\end{document}
