%Type of Document
\documentclass[a4paper, 12pt]{report}

%Load Pre-ambles
\usepackage{../../Environment/Packages}
\usepackage{../../Environment/Conventions}
\usepackage{../../Environment/Hyahoos}


\begin{document}
\tableofcontents

\begin{center}



The famous Galuret's integral is shown below,
\begin{equation}
\int^{\pi}_{0} \frac{\cos(n\theta_{0})}{\cos(\theta_{0}) - \cos(\theta)} \,d\theta_{0} = \frac{\pi \sin(n\theta)}{\sin(\theta)} \quad,\quad n = 0,1,2,3,\dots
\label{Galuret's Integral}
\end{equation}


%Seperator
%Seperator
%Seperator
%Seperator
%Seperator
\section{Biot-Savart Law}
\begin{comment}
\end{comment}

%Seperator
%Seperator
%Seperator
%Seperator
\subsection{Variable List}
\begin{comment}
\end{comment}
The variable definitions are reiterated below for reference,
\begin{itemize}
\item $\bar{v_{i}}$ represents the induced velocity at a particular point of interest due to a vortex line
\item $\Gamma_{v}$ represents the constant strength of a vortex. For full generality, the strength of the vortex can change along the vortex line, but for our purposes, this is asssumed to be constant within a single vortex line
\item $\bar{l_{l}}$ represents some vector tanget to the vortex line
\item $\bar{r_{r}}$ represents the relative position vector of the point of interest to some point on the vortex line
\item $\beta_{v}$ represents the angle between the vector $\bar{r_{r}}$ and the vortex line
\item $\hat{k}$ represents a unit vector into the page
\item $\delta_{r}$ represents a scalar which represents the perpendicular distance between the point of interest and the vortex line.
\item 
\item 
\item 
\end{itemize}

%Seperator
%Seperator
%Seperator
%Seperator
\subsection{General Form}
\begin{comment}
\end{comment}
The general form of the Biot-Savart law is shown below,
$$d\bar{v_{i}} = \frac{\Gamma_{v}}{4\pi}\frac{d \bar{l_{l}}\times \bar{r_{r}}}{|\bar{r_{r}}|^{3}}$$
wherein $\bar{v_{i}}$ represents the induced velocity at a particular point of interest due to a vortex line, $\Gamma_{v}$ represents the constant strength of a vortex, $\bar{l_{l}}$ represents some vector tanget to the vortex line, and $\bar{r_{r}}$ represents the relative position vector of the point of interest to some point on the vortex line. Switching the order of the cross product,
$$d\bar{v_{i}} = -\frac{\Gamma_{v}}{4\pi}\frac{\bar{r_{r}}\times d\bar{l_{l}}}{|\bar{r_{r}}|^{3}}$$
\begin{equation}
d\bar{v_{i}} = -\frac{\Gamma_{v}}{4\pi}\frac{\bar{r_{r}}}{|\bar{r_{r}}|^{3}}\times d\bar{l_{l}}
\label{Biot-Savart Law}
\end{equation}



%Seperator
%Seperator
%Seperator
%Seperator
\subsection{Straight Vortex Line}
\begin{comment}
\end{comment}
\begin{figure}[H]\centering
%\includegraphics[scale=\handsize]{Straight_Vortex_Line}
\caption{Schematic of the Vortex Line and the Point of Interest}
\label{StraightVortexFigure}
\end{figure}
Reiterating equation $\ref{Biot-Savart Law}$,
$$d\bar{v_{i}} = -\frac{\Gamma_{v}}{4\pi}\frac{\bar{r_{r}}}{|\bar{r_{r}}|^{3}}\times d\bar{l_{l}}$$
Re-writing the equation above in terms of the vector magnitudes,
$$d\bar{v_{i}} = -\frac{\Gamma_{v}}{4\pi}\frac{|\bar{r_{r}}|\sin(\beta_{v})}{|\bar{r_{r}}|^{3}}d|\bar{l_{l}}|\hat{k}$$
wherein $\beta_{v}$ represents the angle between the vector $\bar{r_{r}}$ and the vortex line and $\hat{k}$ represents a unit vector into the page. Simplifying,
\begin{equation}
d\bar{v_{i}} = -\frac{\Gamma_{v}}{4\pi}\frac{\sin(\beta_{v})}{|\bar{r_{r}}|^{2}}d|\bar{l_{l}}|\hat{k}
\label{Straight Biot-Savart Magnitudes}
\end{equation}
Examining Figure $\ref{StraightVortexFigure}$,
\begin{equation}
\delta_{r} = |\bar{r_{r}}|\sin(\beta_{v})
\label{trigonometry 1 straight vortex line}
\end{equation}
\begin{equation}
\tan(\beta_{v}) = \frac{\delta_{r}}{|\bar{l_{l}}|}
\label{trigonometry 2 straight vortex line}
\end{equation}
wherein $\delta_{r}$ is a scalar which represents the perpendicular distance between the point of interest and the vortex line. Manipulating equation $\ref{trigonometry 1 straight vortex line}$ to obtain a transformation between $|\bar{r_{r}}|$ to $\beta_{v}$,
$$\delta_{r} = |\bar{r_{r}}|\sin(\beta_{v})$$
\begin{equation}
\frac{\delta_{r}}{\sin(\beta_{v})} = |\bar{r_{r}}|
\label{transformation of veclength 2 perp distance}
\end{equation}
Manipulating expression $\ref{trigonometry 2 straight vortex line}$ to obtain a relation of $d\beta_{v}$ to $d|\bar{l_{l}}|$,
$$\tan(\beta_{v}) = \frac{\delta_{r}}{|\bar{l_{l}}|}$$
$$|\bar{l_{l}}| = \frac{\delta_{r}}{\tan(\beta_{v})}$$
Taking derivative with respect to angle $\beta_{v}$,
$$\frac{d}{d \beta_{v}}\left[|\bar{l_{l}}|\right] = \frac{d}{d \beta_{v}}\left[\frac{\delta_{r}}{\tan(\beta_{v})}\right]$$
$$\frac{d}{d \beta_{v}}\left[|\bar{l_{l}}|\right] = \delta_{r}\frac{d}{d \beta_{v}}\left[\frac{1}{\tan(\beta_{v})}\right]$$
$$\frac{d}{d \beta_{v}}\left[|\bar{l_{l}}|\right] = \delta_{r}\frac{d}{d \beta_{v}}\left[\cot(\beta_{v})\right]$$
$$\frac{d}{d \beta_{v}}\left[|\bar{l_{l}}|\right] = \delta_{r}\times -\frac{1}{\sin^{2}(\beta_{v})}$$
$$\frac{d|\bar{l_{l}}|}{d\beta_{v}} = -\delta_{r}\frac{1}{\sin^{2}(\beta_{v})}$$
\begin{equation}
d|\bar{l_{l}}| = -\delta_{r}\frac{1}{\sin^{2}(\beta_{v})}\,d\beta_{v}
\label{differential relations straight vort line}
\end{equation}
Reiterating equation $\ref{Straight Biot-Savart Magnitudes}$
$$d\bar{v_{i}} = -\frac{\Gamma_{v}}{4\pi}\frac{\sin(\beta_{v})}{|\bar{r_{r}}|^{2}}d|\bar{l_{l}}|\hat{k}$$
$$d\bar{v_{i}} = -\frac{\Gamma_{v}}{4\pi}\sin(\beta_{v})\frac{1}{|\bar{r_{r}}|^{2}}d|\bar{l_{l}}|\hat{k}$$
Susbtituting equation $\ref{transformation of veclength 2 perp distance}$ into the expression above,
$$\frac{\delta_{r}}{\sin(\beta_{v})} = |\bar{r_{r}}|$$
$$\frac{\sin(\beta_{v})}{\delta_{r}} = \frac{1}{|\bar{r_{r}}|}$$
$$\frac{\sin^{2}(\beta_{v})}{\delta_{r}^{2}} = \frac{1}{|\bar{r_{r}}|^{2}}$$
$$d\bar{v_{i}} = -\frac{\Gamma_{v}}{4\pi}\sin(\beta_{v})\frac{\sin^{2}(\beta_{v})}{\delta_{r}^{2}}\, d|\bar{l_{l}}|\hat{k}$$
Substituting equation $\ref{differential relations straight vort line}$ for the differentials,
$$d|\bar{l_{l}}| = -\delta_{r}\frac{1}{\sin^{2}(\beta_{v})}\,d\beta_{v}$$
$$d\bar{v_{i}} = -\frac{\Gamma_{v}}{4\pi}\sin(\beta_{v})\frac{\sin^{2}(\beta_{v})}{\delta_{r}^{2}}\times -\delta_{r}\frac{1}{\sin^{2}(\beta_{v})}\,d\beta_{v}\hat{k}$$
$$d\bar{v_{i}} = \frac{\Gamma_{v}}{4\pi}\sin(\beta_{v})\frac{\sin^{2}(\beta_{v})}{\delta_{r}}\frac{1}{\sin^{2}(\beta_{v})}\,d\beta_{v}\hat{k}$$
$$d\bar{v_{i}} = \frac{\Gamma_{v}}{4\pi}\frac{1}{\delta_{r}}\sin(\beta_{v})\, d\beta_{v}\hat{k}$$
$$d\bar{v_{i}} = \frac{\Gamma_{v}}{4\pi\delta_{r}}\sin(\beta_{v})\, d\beta_{v}\hat{k}$$
Integrating the expression above,
$$\int d\bar{v_{i}} = \frac{\Gamma_{v}}{4\pi\delta_{r}}\int^{\beta_{2}}_{\beta_{1}} \sin(\beta_{v})\, d\beta_{v}\hat{k}$$
$$\bar{v_{i}} = \frac{\Gamma_{v}}{4\pi\delta_{r}} \times -\left[\cos(\beta_{v})\right]^{\beta_{2}}_{\beta_{1}}\hat{k}$$
$$\bar{v_{i}} = -\frac{\Gamma_{v}}{4\pi\delta_{r}}\left[\cos(\beta_{v})\right]^{\beta_{2}}_{\beta_{1}}\hat{k}$$
$$\bar{v_{i}} = -\frac{\Gamma_{v}}{4\pi\delta_{r}}\left[\cos(\beta_{2}) - \cos(\beta_{1})\right]\hat{k}$$
\begin{equation}
\bar{v_{i}} = \frac{\Gamma_{v}}{4\pi\delta_{r}}\left[\cos(\beta_{1}) - \cos(\beta_{2})\right]\hat{k}
\label{Induced Velocity Straight Vortex Line Unspecified}
\end{equation}





%Seperator
%Seperator
%Seperator
\subsubsection{Infinite Vortex Line}
\begin{comment}
\end{comment}
Reiterating equation $\ref{Induced Velocity Straight Vortex Line Unspecified}$,
$$\bar{v_{i}} = \frac{\Gamma_{v}}{4\pi\delta_{r}}\left[\cos(\beta_{1}) - \cos(\beta_{2})\right]\hat{k}$$
For the infinite vortex line case, $\beta_{1} = 0$ and $\beta_{2} = \pi$
$$\bar{v_{i}} = \frac{\Gamma_{v}}{4\pi\delta_{r}}\left[1+1\right]\hat{k}$$
$$\bar{v_{i}} = \frac{2\Gamma_{v}}{4\pi\delta_{r}}\hat{k}$$
\begin{equation}
\bar{v_{i}} = \frac{\Gamma_{v}}{2\pi\delta_{r}}\hat{k}
\label{Induced Velocity Infinite Vortex Line Result}
\end{equation}


%Seperator
%Seperator
%Seperator
\subsubsection{Semi-Infinite Vortex Line}
\begin{comment}
\end{comment}
Reiterating equation $\ref{Induced Velocity Straight Vortex Line Unspecified}$,
$$\bar{v_{i}} = \frac{\Gamma_{v}}{4\pi\delta_{r}}\left[\cos(\beta_{1}) - \cos(\beta_{2})\right]\hat{k}$$
For the semi-infinite vortex line, $\beta_{1} = \pi/2$ and $\beta_{2} = \pi$. Therefore,
$$\bar{v_{i}} = \frac{\Gamma_{v}}{4\pi\delta_{r}}\left[0 + 1\right]\hat{k}$$
\begin{equation}
\bar{v_{i}} = \frac{\Gamma_{v}}{4\pi\delta_{r}}\hat{k}
\label{Induced Velocity Semi-Infinite Vortex Line Result}
\end{equation}

%Seperator
%Seperator
%Seperator
%Seperator
%Seperator
\section{Lifting Line Theory}
\begin{comment}
\end{comment}
\begin{figure}[H]\centering
%\includegraphics[scale=\handsize]{}
\caption{Schematic of Continuous Vortex Distribution in Lifting Line Theory}
\label{Lifting Line Vortex Distribution Schematic}
\end{figure}

%Seperator
%Seperator
%Seperator
%Seperator
\subsection{Variable List}
\begin{comment}
\end{comment}
The variable definitions are reiterated below for reference,
\begin{itemize}
\item $L_{a}$ represents the lift force experienced by a $2$-dimensional airfoil
\item $\rho_{\infty}$ represents the free-stream density of the incoming fluid
\item $v_{\infty}$ represents the free-stream velocity of the incoming fluid
\item $\Gamma_{a}$ represents the circulation for a $2$-dimensional airfoil
\item $L_{w}$ represents the total lift experienced by the whole generalized wing
\item $\Gamma_{w}$ represents the total circulation generated by the wing
\item $\bar{w_{i}}$ is a vector representing the "downwash" velocity
\item $y_{0}$ represents a particular point of interest at the wing
\item $D_{i}$ represents the total induced drag experienced by the wing
\item $\alpha_{eff}$ represents the effective angle of attack that the airfoil sees with the altered flow
\item $\alpha_{Geo}$ represents the geometric angle of attack of the airfoil with respect to the free-stream very far away from the airfoil
\item $C_{L,2d}$ represents the coefficient of lift for a $2$-dimensional airfoil
\item $\alpha_{0}$ represents the zero lift angle of attack for a $2$-dimensional airfoil
\item $c_{w}(y_{0})$ represents the chord of the wing at some spanwise location along the wing
\item $b$ represents the total span of the wing
\item $A_{w}$ represents the area of the wing
\item 
\end{itemize}


%Seperator
%Seperator
%Seperator
%Seperator
\subsection{Lift}
\begin{comment}
\end{comment}
Based on the Kutta-Joukowski theorem, which can be proven using $2$-dimensional conformal mapping,
$$L_{a} = \rho_{\infty}v_{\infty}\Gamma_{a}$$
wherein $L_{a}$ represents the lift force experienced by a $2$-dimensional airfoil, $\rho_{\infty}$ represents the free-stream density of the incoming fluid, $v_{\infty}$ represents the free-stream velocity of the incoming fluid, and $\Gamma_{a}$ represents the circulation for a $2$-dimensional airfoil. For a generalized wing, wherein there is airfoil geometry variation in the spanwise direction, it seems reasonable to consider the $2$-dimensional airfoil properties to hold true for a particular point in the generalized wing. Therefore, 
$$L_{a} = dL_{w}$$
wherein $L_{w}$ represents the total lift experienced by the whole generalized wing.
$$\Gamma_{a} = \Gamma_{w}dy$$
wherein $\Gamma_{w}$ represents the total circulation generated by the wing. Substituting into the Kutta-Joukowski theorem,
\begin{equation}
dL_{w} = \rho_{\infty}v_{\infty}\Gamma_{w}\,dy
\label{differential lift in lifting line}
\end{equation}
Integrating for the entire wing,
\begin{equation}
L_{w} = \rho_{\infty}v_{\infty}\int^{b/2}_{-b/2}\Gamma_{w}\,dy
\label{Lift in Lifting Line Theory}
\end{equation}


%Seperator
%Seperator
%Seperator
%Seperator
\subsection{Induced Drag}
\begin{comment}
\end{comment}
Induced drag is specific only to $3$-dimensional flows around $3$-dimensional wings. Induced drag does not exist for $2$-dimensional flows around $2$-dimensional wings. Based on the lifting line model, the "legs" of the vortex distribution induces a downward velocity on the wing. THis downward fluid velocity shifts the lift vector slightly and produces a horizontal component of force in the flow direction. This force is the induced drag.

%Seperator
%Seperator
%Seperator
\subsubsection{Downwash}
\begin{comment}
\end{comment}
Downwash is the vertical component of induced velocity due to the "legs" of the vortex distribution stretching to $\infty$. Reiterating the induced velocity due to application of the Biot-Savart Law for a semi-infinite vortex line (equation $\ref{Induced Velocity Semi-Infinite Vortex Line Result}$)
$$\bar{v_{i}} = \frac{\Gamma_{v}}{4\pi\delta_{r}}\hat{k}$$
$$\bar{v_{i}} = \frac{1}{4\pi\delta_{r}}\Gamma_{v}\hat{k}$$
The strenght of the vortex line, $\Gamma_{v}$ is an infinitesmially small portion of the total vortex strenght of the wing. Therefore,
$$\Gamma_{v} = d\Gamma_{w}$$
Substituting, and then multiplying by $1$,
$$\bar{v_{i}} = \frac{1}{4\pi\delta_{r}}d\Gamma_{w}\hat{k}$$
$$\bar{v_{i}} = \frac{1}{4\pi\delta_{r}}\frac{d\Gamma_{w}}{dy} \,dy\hat{k}$$
The induced velocity when applied to this problem $\bar{v_{i}}$ takes downwards to be the positive direction. Typically downwash takes "upwards" as positive direction. Therefore,
$$\bar{v_{i}} = -d\bar{w_{i}}$$
wherein $\bar{w_{i}}$ is a vector representing the "downwash" velocity. Substituting,
$$-d\bar{w_{i}} = \frac{1}{4\pi\delta_{r}}\frac{d\Gamma_{w}}{dy} \,dy\hat{z}$$
$$d\bar{w_{i}} = -\frac{1}{4\pi\delta_{r}}\frac{d\Gamma_{w}}{dy} \,dy\hat{z}$$
The distance $\delta_{r}$ is taken to be,
$$\delta_{r} = y_{0}-y$$
wherein $y_{0}$ represents a particular point of interest at the wing. Substituting,
$$d\bar{w_{i}} = -\frac{1}{4\pi}\frac{1}{y_{0}-y}\frac{d\Gamma_{w}}{dy} \,dy\hat{z}$$
Integrating for the entire wing,
\begin{equation}
\bar{w_{i}} = -\frac{1}{4\pi} \int^{b/2}_{-b/2}\frac{1}{y_{0}-y}\frac{d\Gamma_{w}}{dy} \,dy\hat{z}
\label{Downwash in Lifting Line Theory}
\end{equation}

%Seperator
%Seperator
%Seperator
\subsubsection{Induced Angle of Attack}
\begin{comment}
\end{comment}
The induced angle of attack by convention is a positive quantity when the downwash vector is pointing down. Therefore,
$$\alpha_{i} = \arctan\left(\frac{|\bar{w_{i}}|}{v_{\infty}}\right)$$
By the small angle approximation,
$$\alpha_{i} \approx \frac{|\bar{w_{i}}|}{v_{\infty}}$$
Substituting the expression for downwash velocity vector (equation $\ref{Downwash in Lifting Line Theory}$)
$$\bar{w_{i}} = -\frac{1}{4\pi} \int^{b/2}_{-b/2}\frac{1}{y_{0}-y}\frac{d\Gamma_{w}}{dy} \,dy\hat{z}$$
$$|\bar{w_{i}}| = \frac{1}{4\pi} \int^{b/2}_{-b/2}\frac{1}{y_{0}-y}\frac{d\Gamma_{w}}{dy} \,dy$$
$$\alpha_{i}(y_{0}) = \frac{1}{v_{\infty}}\frac{1}{4\pi} \int^{b/2}_{-b/2}\frac{1}{y_{0}-y}\frac{d\Gamma_{w}}{dy} \,dy$$
\begin{equation}
\alpha_{i}(y_{0}) = \frac{1}{4\pi v_{\infty}} \int^{b/2}_{-b/2}\frac{1}{y_{0}-y}\frac{d\Gamma_{w}}{dy} \,dy
\label{induced angle of attack Lifting Line Theory}
\end{equation}

%Seperator
%Seperator
%Seperator
\subsubsection{Drag Expression}
\begin{comment}
\end{comment}
Examining the $2$ dimensional airfoil, with a shifted incoming fluid velocity,
$$dD_{i} = dL_{w}\sin(\alpha_{i})$$
wherein $D_{i}$ represents the total induced drag experienced by the wing. Using the small angle approximation,
$$\sin(\alpha_{i}) \approx \alpha_{i}$$
Therefore,
$$dD_{i} = \alpha_{i}dL_{w}$$
Using equation $\ref{differential lift in lifting line}$
$$dL_{w} = \rho_{\infty}v_{\infty}\Gamma_{w}\,dy$$
$$dD_{i} = \alpha_{i}\rho_{\infty}v_{\infty}\Gamma_{w}\,dy$$
Integrating for the whole wing,
$$D_{i} = \int^{b/2}_{-b/2}\alpha_{i}\rho_{\infty}v_{\infty}\Gamma_{w}\,dy$$
\begin{equation}
D_{i} = \rho_{\infty}v_{\infty}\int^{b/2}_{-b/2}\alpha_{i}\Gamma_{w}\,dy
\label{Induced Drag Lifting Line Theory}
\end{equation}


%Seperator
%Seperator
%Seperator
%Seperator
\subsection{Vortex Distribution}
\begin{comment}
\end{comment}

%Seperator
%Seperator
%Seperator
\subsubsection{Angle of Attack Relations}
\begin{comment}
\end{comment}
$$\alpha_{eff}(y_{0}) = \alpha_{Geo}(y_{0}) - \alpha_{i}(y_{0})$$
wherein $\alpha_{eff}$ represents the effective angle of attack that the airfoil sees with the altered flow and $\alpha_{Geo}$ represents the geometric angle of attack of the airfoil with respect to the free-stream very far away from the airfoil. These various angles of attack are functions of spanwise distance $y_{0}$. Simple algebraic manipulation of the expression above yields,
\begin{equation}
\alpha_{eff} + \alpha_{i} = \alpha_{Geo}
\label{Various Angles of Attack}
\end{equation}

%Seperator
%Seperator
%Seperator
\subsubsection{Lifting Line Fundamental Equation}
\begin{comment}
\end{comment}
Based on results of conformal mapping for $2$-dimensional airfoils, the coefficient of lift for a $2$-dimensional airfoil,
$$C_{L,2d} = 2\pi(\alpha_{eff}-\alpha_{0})$$
wherein $C_{L,2d}$ represents the coefficient of lift for a $2$-dimensional airfoil and $\alpha_{0}$ represents the zero lift angle of attack for a $2$-dimensional airfoil. The total lift experienced by the wing can be thought of as an integral of the infinitesmially small amount of lift generated by a series of $2$-dimensional airfoil cross-sections in the spanwise direction. Therefore,
$$\frac{dL_{w}}{dy} = \frac{1}{2}\rho_{\infty}v_{\infty}^{2}c_{w}(y_{0}) C_{L,2d}$$
wherein $c_{w}(y_{0})$ represents the chord of the wing at some spanwise location along the wing.
Reiterating equation $\ref{differential lift in lifting line}$
$$dL_{w} = \rho_{\infty}v_{\infty}\Gamma_{w}\,dy$$
$$\frac{dL_{w}}{dy} = \rho_{\infty}v_{\infty}\Gamma_{w}$$
Substituting into the definition of the $2$-dimensional coefficient of lift,
$$\frac{dL_{w}}{dy} = \frac{1}{2}\rho_{\infty}v_{\infty}^{2}c_{w}(y_{0}) C_{L,2d} = \rho_{\infty}v_{\infty}\Gamma_{w}$$
$$\frac{1}{2}v_{\infty}^{2}c_{w}(y_{0}) C_{L,2d} = v_{\infty}\Gamma_{w}$$
$$\frac{1}{2}v_{\infty}c_{w}(y_{0}) C_{L,2d} = \Gamma_{w}$$
$$C_{L,2d} = \frac{2\Gamma_{w}}{v_{\infty}c_{w}}$$
Subsituting into the expression for the $2$-dimensional lift coefficient based on angles of attack,
$$C_{L,2d} = \frac{2\Gamma_{w}}{v_{\infty}c_{w}} = 2\pi(\alpha_{eff}-\alpha_{0})$$
$$\frac{\Gamma_{w}}{v_{\infty}c_{w}} = \pi(\alpha_{eff}-\alpha_{0})$$
$$\frac{\Gamma_{w}}{\pi v_{\infty}c_{w}} = \alpha_{eff}-\alpha_{0}$$
$$\frac{\Gamma_{w}}{\pi v_{\infty}c_{w}} + \alpha_{0} = \alpha_{eff}$$
Substituting the expression above into equation $\ref{Various Angles of Attack}$,
$$\frac{\Gamma_{w}}{\pi v_{\infty}c_{w}} + \alpha_{0} + \alpha_{i} = \alpha_{Geo}$$
Subsituting the definition for induced angle of attack (equation $\ref{induced angle of attack Lifting Line Theory}$)
$$\alpha_{i}(y_{0}) = \frac{1}{4\pi v_{\infty}} \int^{b/2}_{-b/2}\frac{1}{y_{0}-y}\frac{d\Gamma_{w}}{dy} \,dy$$
\begin{equation}
\frac{\Gamma_{w}}{\pi v_{\infty}c_{w}} + \alpha_{0} + \frac{1}{4\pi v_{\infty}} \int^{b/2}_{-b/2}\frac{1}{y_{0}-y}\frac{d\Gamma_{w}}{dy} \,dy = \alpha_{Geo}
\label{Fundamental Equation of Lifting Line Theory}
\end{equation}
The equation shown above is the famous "Fundamental Equation of Prandtl's Lifting Line Theory".

%Seperator
%Seperator
%Seperator
\subsubsection{Fourier Method to Solving Circulation}
\begin{comment}
\end{comment}
Using the coordinate transformation, 
\begin{equation}
y = -\frac{b}{2}\cos(\theta)
\label{Recurring Coordinate Transformation Lifting Line}
\end{equation}
An arbitrary vortex distribution can be expressed as an infinite sum of sine waves as shown below,
\begin{equation}
\Gamma_{w} = 2bv_{\infty}\sum^{\infty}_{k = 1}\left[A_{k}\sin(k\theta)\right]
\label{Fourier Sine Circulation Distribution}
\end{equation}
wherein $b$ represents the total span of the wing. The constant $2bv_{\infty}$ was chosen purely for scaling purposes. The Fourier Sine infinite series shown above could be substituted into the "Fundamental Equation of Prandtl's Lifting Line Theory" to obtain an expression of the constants $A_{j}$ in terms of known quantities of the wing. Solving the resulting equation would yield coefficients $A_{j}$ which can then be used to solve for lift and drag of the wing. Determining the derivative of equation $\ref{Fourier Sine Circulation Distribution}$ with respect to $y$,
$$\frac{d}{d y}\left[\Gamma_{w}\right] = \frac{d}{d y}\left\{2bv_{\infty}\sum^{\infty}_{k = 1}\left[A_{k}\sin(k\theta)\right]\right\}$$
\begin{equation}
\frac{d}{d y}\left[\Gamma_{w}\right] =  2bv_{\infty}\sum^{\infty}_{k = 1}\left[A_{k}\frac{d}{d y}\left\{\sin(k\theta)\right\}\right]
\label{Fourier Sine Derivative Lifting Line}
\end{equation}
$$\frac{d}{d y}\left\{\sin(k\theta)\right\} = \frac{d}{d k\theta}\left\{\sin(k\theta)\right\}\times\frac{d k\theta}{dy}$$
$$\frac{d}{d y}\left\{\sin(k\theta)\right\} =  \cos(k\theta)  \times k\frac{d \theta}{dy}$$
$$\frac{d}{d y}\left\{\sin(k\theta)\right\} =  k\cos(k\theta) \times \frac{d \theta}{dy}$$
Reiterating the coordinate transformation used,
$$y = -\frac{b}{2}\cos(\theta)$$
Taking derivative of the coordinate transformation with respect to $\theta$,
\begin{equation}
\frac{dy}{d\theta} = \frac{b}{2}\sin(\theta)
\label{dydth Common Coordinate Transformation}
\end{equation}
$$\frac{d\theta}{dy} = \frac{2}{b}\frac{1}{\sin(\theta)}$$
Substituting,
$$\frac{d}{d y}\left\{\sin(k\theta)\right\} =  k\cos(k\theta) \times \frac{2}{b}\frac{1}{\sin(\theta)}$$
$$\frac{d}{d y}\left\{\sin(k\theta)\right\} =  \frac{2k}{b}\frac{\cos(k\theta)}{\sin(\theta)}$$
Susbtituting into equation $\ref{Fourier Sine Derivative Lifting Line}$
$$\frac{d}{d y}\left[\Gamma_{w}\right] =  2bv_{\infty}\sum^{\infty}_{k = 1}\left[A_{k} \frac{2k}{b}\frac{\cos(k\theta)}{\sin(\theta)} \right]$$
$$\frac{d}{d y}\left[\Gamma_{w}\right] =  2bv_{\infty}\sum^{\infty}_{k = 1}\left[\frac{2A_{k}k}{b}\frac{\cos(k\theta)}{\sin(\theta)}\right]$$
Reiterating equation $\ref{dydth Common Coordinate Transformation}$ 
$$\frac{dy}{d\theta} = \frac{b}{2}\sin(\theta)$$
Manipulating the expression,
$$dy = \frac{b}{2}\sin(\theta)d\theta$$
Therefore,
$$\frac{d}{d y}\left[\Gamma_{w}\right]\,dy =  2bv_{\infty}\sum^{\infty}_{k = 1}\left[\frac{2A_{k}k}{b}\frac{\cos(k\theta)}{\sin(\theta)}\right]\times \frac{b}{2}\sin(\theta)\,d\theta$$
$$\frac{d}{d y}\left[\Gamma_{w}\right]\,dy =  2bv_{\infty}\sum^{\infty}_{k = 1}\left[\frac{2A_{k}k}{b}\frac{b}{2} \cos(k\theta)\right] \,d\theta$$
$$\frac{d}{d y}\left[\Gamma_{w}\right]\,dy =  2bv_{\infty}\sum^{\infty}_{k = 1}\left[A_{k}k \cos(k\theta)\right] \,d\theta$$
Reiterating the coordinate transformation $\ref{Recurring Coordinate Transformation Lifting Line}$,
$$y = -\frac{b}{2}\cos(\theta)$$
Applying the coordinate transformation for some specific point along the wing $y = y_{0}$,
$$y_{0} = -\frac{b}{2}\cos(\theta_{0})$$
Therefore,
$$y_{0}-y = -\frac{b}{2}\cos(\theta_{0})+\frac{b}{2}\cos(\theta)$$
$$y_{0}-y = \frac{b}{2}\left[\cos(\theta)-\cos(\theta_{0})\right]$$
$$\frac{1}{y_{0}-y} = \frac{2}{b}\frac{1}{\cos(\theta)-\cos(\theta_{0})}$$
Therefore,
$$\frac{1}{y_{0}-y}\frac{d\Gamma_{w}}{d y}\,dy =  2bv_{\infty}\sum^{\infty}_{k = 1}\left[\frac{2}{b}\frac{1}{\cos(\theta)-\cos(\theta_{0})}\times A_{k}k \cos(k\theta)\right] \,d\theta$$
$$\frac{1}{y_{0}-y}\frac{d\Gamma_{w}}{d y}\,dy =  2bv_{\infty}\sum^{\infty}_{k = 1}\left[\frac{2}{b}A_{k}k\frac{\cos(k\theta)}{\cos(\theta)-\cos(\theta_{0})}\right] \,d\theta$$
$$\frac{1}{y_{0}-y}\frac{d\Gamma_{w}}{d y}\,dy =  4v_{\infty}\sum^{\infty}_{k = 1}\left[A_{k}k\frac{\cos(k\theta)}{\cos(\theta)-\cos(\theta_{0})}\right] \,d\theta$$
When $y = b/2$,
$$\frac{b}{2} = -\frac{b}{2}\cos(\theta)$$
$$-1 = \cos(\theta)$$
$$\theta = \pi$$
When $y = -b/2$,
$$-\frac{b}{2} = -\frac{b}{2}\cos(\theta)$$
$$1 = \cos(\theta)$$
$$\theta = 0$$
Integrating for the whole wing with the bounds determined above,
$$\int^{b/2}_{-b/2}\frac{1}{y_{0}-y}\frac{d\Gamma_{w}}{d y}\,dy =  4v_{\infty}\sum^{\infty}_{k = 1}\left[A_{k}k \int^{\pi}_{0}\frac{\cos(k\theta)}{\cos(\theta)-\cos(\theta_{0})}\,d\theta\right] $$
The last term in the expression above is Glauret's integral. Reiterating Glauret's integral (equation $\ref{Galuret's Integral}$)
$$\int^{\pi}_{0} \frac{\cos(n\theta_{0})}{\cos(\theta_{0}) - \cos(\theta)} \,d\theta_{0} = \frac{\pi \sin(n\theta)}{\sin(\theta)} \quad,\quad n = 0,1,2,3,\dots$$
$$\int^{b/2}_{-b/2}\frac{1}{y_{0}-y}\frac{d\Gamma_{w}}{d y}\,dy =  4v_{\infty}\sum^{\infty}_{k = 1}\left[A_{k}k \frac{\pi \sin(k\theta_{0})}{\sin(\theta_{0})}\right] $$
\begin{equation}
\int^{b/2}_{-b/2}\frac{1}{y_{0}-y}\frac{d\Gamma_{w}}{d y}\,dy =  4v_{\infty}\pi\sum^{\infty}_{k = 1}\left[A_{k}k\frac{\sin(k\theta_{0})}{\sin(\theta_{0})}\right] 
\label{second term generalized lifting line theory}
\end{equation}
Substituting equation $\ref{second term generalized lifting line theory}$ and equation $\ref{Fourier Sine Circulation Distribution}$ into "Fundamental Lifting Line Theory" (equation $\ref{Fundamental Equation of Lifting Line Theory}$),
$$\frac{\Gamma_{w}}{\pi v_{\infty}c_{w}} + \alpha_{0} + \frac{1}{4\pi v_{\infty}} \int^{b/2}_{-b/2}\frac{1}{y_{0}-y}\frac{d\Gamma_{w}}{dy} \,dy = \alpha_{Geo}$$
$$\frac{\Gamma_{w}}{\pi v_{\infty}c_{w}} + \alpha_{0} + \frac{1}{4\pi v_{\infty}}\times 4v_{\infty}\pi\sum^{\infty}_{k = 1}\left[A_{k}k\frac{\sin(k\theta_{0})}{\sin(\theta_{0})}\right] = \alpha_{Geo}$$
$$\frac{\Gamma_{w}}{\pi v_{\infty}c_{w}} + \alpha_{0} + \frac{v_{\infty}}{ v_{\infty}} \sum^{\infty}_{k = 1}\left[A_{k}k\frac{\sin(k\theta_{0})}{\sin(\theta_{0})}\right] = \alpha_{Geo}$$
$$\frac{\Gamma_{w}}{\pi v_{\infty}c_{w}} + \alpha_{0} + \sum^{\infty}_{k = 1}\left[A_{k}k\frac{\sin(k\theta_{0})}{\sin(\theta_{0})}\right] = \alpha_{Geo}$$
$$\frac{1}{\pi v_{\infty}c_{w}}\left\{2bv_{\infty}\sum^{\infty}_{k = 1}\left[A_{k}\sin(k\theta_{0})\right]\right\} + \alpha_{0} + \sum^{\infty}_{k = 1}\left[A_{k}k\frac{\sin(k\theta_{0})}{\sin(\theta_{0})}\right] = \alpha_{Geo}$$
$$\alpha_{0} + \frac{2b}{\pi c_{w}}\left\{\sum^{\infty}_{k = 1}\left[A_{k}\sin(k\theta_{0})\right]\right\} + \sum^{\infty}_{k = 1}\left[A_{k}k\frac{\sin(k\theta_{0})}{\sin(\theta_{0})}\right] = \alpha_{Geo}$$
$$\alpha_{0} + \sum^{\infty}_{k = 1}\left[A_{k}\frac{2b}{\pi c_{w}}\sin(k\theta_{0})\right] + \sum^{\infty}_{k = 1}\left[A_{k}k\frac{\sin(k\theta_{0})}{\sin(\theta_{0})}\right] = \alpha_{Geo}$$
$$\alpha_{0} + \sum^{\infty}_{k = 1}\left[A_{k}\frac{2b\sin(k\theta_{0})}{\pi c_{w}}\right] + \sum^{\infty}_{k = 1}\left[A_{k}\frac{k\sin(k\theta_{0})}{\sin(\theta_{0})}\right] = \alpha_{Geo}$$
$$\alpha_{0} + \sum^{\infty}_{k = 1}\left[A_{k}\frac{2b\sin(k\theta_{0})}{\pi c_{w}} + A_{k}\frac{k\sin(k\theta_{0})}{\sin(\theta_{0})}\right] = \alpha_{Geo}$$
$$\alpha_{0} + \sum^{\infty}_{k = 1}\left\{A_{k}\left[\frac{2b\sin(k\theta_{0})}{\pi c_{w}} + \frac{k\sin(k\theta_{0})}{\sin(\theta_{0})}\right]\right\} = \alpha_{Geo}$$
$$\alpha_{0} + \sum^{\infty}_{k = 1}\left\{A_{k}\left[\left(\frac{2b\sin(k\theta_{0})}{\pi c_{w}} + \frac{k\sin(k\theta_{0})}{\sin(\theta_{0})}\right)\right]\right\} = \alpha_{Geo}$$
\begin{equation}
\alpha_{0} + \sum^{\infty}_{k = 1}\left\{A_{k}\left[\left(\frac{2b}{\pi c_{w}} + \frac{k}{\sin(\theta_{0})}\right)\sin(k\theta_{0})\right]\right\} = \alpha_{Geo}
\label{System of Equations Lifting Line Theory General}
\end{equation}


%Seperator
%Seperator
%Seperator
\subsubsection{Solution Algorithm}
\begin{comment}
\end{comment}


%Seperator
%Seperator
%Seperator
%Seperator
\subsection{Special Case: Elliptical Vortex Distribution}
\begin{comment}
\end{comment}
A wing with an elliptical circulation distribution has a circulation distribution as denoted below,
$$\Gamma_{w} = \Gamma_{0}\sqrt{1-\left(\frac{2}{b}y\right)^{2}}$$
wherein $\Gamma_{0}$ is some constant. Using the coordinate transformation, 
$$y = -\frac{b}{2}\cos(\theta)$$
$$\frac{2}{b}y = -\frac{2}{b}\frac{b}{2}\cos(\theta)$$
$$\frac{2}{b}y = -\cos(\theta)$$
$$\left(\frac{2}{b}y\right)^{2} = \cos^{2}(\theta)$$
$$\Gamma_{0}\sqrt{1-\left(\frac{2}{b}y\right)^{2}} = \Gamma_{0}\sqrt{1-\cos^{2}(\theta)}$$
A famous trigonometric identity,
$$\sin^{2}(\alpha) + \cos^{2}(\alpha) = 1$$
$$\sin^{2}(\alpha) = 1 - \cos^{2}(\alpha)$$
$$\Gamma_{0}\sqrt{1-\left(\frac{2}{b}y\right)^{2}} = \Gamma_{0}\sqrt{\sin^{2}(\theta)}$$
\begin{equation}
\Gamma_{0}\sqrt{1-\left(\frac{2}{b}y\right)^{2}} = \Gamma_{0}\sin(\theta)
\label{elliptical circulation distribution th coordinates}
\end{equation}
From a previous algebraic manipulation, When $y = b/2$,
$$\frac{b}{2} = -\frac{b}{2}\cos(\theta)$$
$$-1 = \cos(\theta)$$
$$\theta = \pi$$
When $y = -b/2$,
$$-\frac{b}{2} = -\frac{b}{2}\cos(\theta)$$
$$1 = \cos(\theta)$$
$$\theta = 0$$
Taking derivative of the coordinate transformation,
$$\frac{dy}{d\theta} = \frac{b}{2}\sin(\theta)$$
\begin{equation}
dy = \frac{b}{2}\sin(\theta)\,d\theta
\label{dy in terms of th lifting line coordinates}
\end{equation}


%Seperator
%Seperator
%Seperator
\subsubsection{Lift}
\begin{comment}
\end{comment}
Reiterating the expression for lift experienced by a wing, $\ref{Lift in Lifting Line Theory}$
$$L_{w} = \rho_{\infty}v_{\infty}\int^{b/2}_{-b/2}\Gamma_{w}\,dy$$
Substituting the circulation distribution (equation $\ref{elliptical circulation distribution th coordinates}$), the bounds and the differential $dy$ (equation $\ref{dy in terms of th lifting line coordinates}$),
$$L_{w} = \rho_{\infty}v_{\infty}\int^{\pi}_{0} \Gamma_{0}\sin(\theta) \frac{b}{2}\sin(\theta)\,d\theta$$
$$L_{w} = \frac{1}{2}\rho_{\infty}v_{\infty}b \int^{\pi}_{0} \Gamma_{0}\sin(\theta) \sin(\theta)\,d\theta$$
$$L_{w} = \frac{1}{2}\rho_{\infty}v_{\infty}\Gamma_{0}b \int^{\pi}_{0} \sin^{2}(\theta) \,d\theta$$
Using the double angle trigonometric identity,
$$\cos(2\theta) = 1 - 2\sin^{2}(\theta)$$
$$2\sin^{2}(\theta) = 1 - \cos(2\theta)$$
$$\sin^{2}(\theta) = \frac{1}{2}\left[1 - \cos(2\theta)\right]$$
$$L_{w} = \frac{1}{2}\rho_{\infty}v_{\infty}\Gamma_{0}b \frac{1}{2}\int^{\pi}_{0}  1 - \cos(2\theta)  \,d\theta$$
$$L_{w} = \frac{1}{4}\rho_{\infty}v_{\infty}\Gamma_{0}b \left[\theta - \frac{1}{2}\sin(2\theta)\right]^{\pi}_{0} $$
\begin{equation}
L_{w} = \frac{1}{4}\rho_{\infty}v_{\infty}\Gamma_{0}b\pi
\label{Elliptical Distribution Lift Force}
\end{equation}
\begin{comment}
Using the definition for coefficient of lift,
$$C_{L} = \frac{L_{w}}{\displaystyle \frac{1}{2}\rho_{\infty}v_{\infty}^{2}A_{w}}$$
wherein $A_{w}$ represents the area of the wing. Manipulating the expression,
$$\frac{1}{2}\rho_{\infty}v_{\infty}^{2}A_{w} C_{L} = L_{w}$$
Equating the lift force produced by the wing to equation $\ref{Elliptical Distribution Lift Force}$,
$$\frac{1}{2}\rho_{\infty}v_{\infty}^{2}A_{w} C_{L} = L_{w} = \frac{1}{4}\rho_{\infty}v_{\infty}\Gamma_{0}b\pi$$
$$\frac{1}{2}v_{\infty}A_{w} C_{L} = \frac{1}{4}\Gamma_{0}b\pi$$
$$\frac{4}{2}v_{\infty}A_{w} C_{L} = \Gamma_{0}b\pi$$
$$2 v_{\infty}A_{w} C_{L} = \Gamma_{0}b\pi$$
$$\frac{2 v_{\infty}A_{w} C_{L}}{b\pi} = \Gamma_{0}$$
Susbtituting the constant $\Gamma_{0}$ out to equation $\ref{Elliptical Distribution Lift Force}$ to eliminate dependency on unknown factor $\Gamma_{0}$,
$$L_{w} = \frac{1}{4}\rho_{\infty}v_{\infty}b\pi\times \frac{2 v_{\infty}A_{w} C_{L}}{b\pi}$$
$$L_{w} = \frac{1}{4}\rho_{\infty}v_{\infty}\times \frac{2 v_{\infty}A_{w} C_{L}}{}$$
\end{comment}


%Seperator
%Seperator
%Seperator
\subsubsection{Downwash}
\begin{comment}
\end{comment}
Reiterating the experssion for downwash produced by a wing, $\ref{Downwash in Lifting Line Theory}$
$$\bar{w_{i}} = -\frac{1}{4\pi} \int^{b/2}_{-b/2}\frac{1}{y_{0}-y}\frac{d\Gamma_{w}}{dy} \,dy\hat{z}$$

%Seperator
%Seperator
%Seperator
\subsubsection{Induced Drag}
\begin{comment}
\end{comment}
Reiterating the expression for induced angle of attack produced by the trailing vortices, 
$$\alpha_{i} \approx \frac{|\bar{w_{i}}|}{v_{\infty}}$$
Reiterating the expression for induced drag experienced by a wing, $\ref{Induced Drag Lifting Line Theory}$
$$D_{i} = \rho_{\infty}v_{\infty}\int^{b/2}_{-b/2}\alpha_{i}\Gamma_{w}\,dy$$

%Seperator
%Seperator
%Seperator
%Seperator
\subsection{Elliptical Distribution Proofs}
\begin{comment}
\end{comment}





%Seperator
%Seperator
%Seperator
%Seperator
%Seperator
\end{center}

\end{document}
