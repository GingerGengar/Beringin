%Type of Document
\documentclass[a4paper, 12pt]{report}

%Load Pre-ambles
\usepackage{../../Environment/Packages}
\usepackage{../../Environment/Conventions}
\usepackage{../../Environment/Hyahoos}

\begin{document}

\begin{center}

%Seperator
%Seperator
%Seperator
%Seperator
%Seperator
\section{Grid Transformations}
\begin{comment}
\end{comment}

%Seperator
%Seperator
%Seperator
%Seperator
\subsection{Transformation Definitions}
\begin{comment}
\end{comment}
Let the grid points in physical space have its positions be expressed in $x$, $y$, and $z$. Let the grid points transformed into a uniform mesh have its positione expressed in $\xi$, $\eta$, and $\zeta$ coordinates. The transformation is assumed to be smooth and invertible. Invertibility implies that each grid point in $x$, $y$, and $z$ have a single corresponding point in $\xi$, $\eta$, and $\zeta$ and vice versa. Under the grid transformation, 
$$x = x(\xi,\eta,\zeta) \quad,\quad y = y(\xi,\eta,\zeta) \quad,\quad z = z(\xi,\eta,\zeta)$$
Due to invertibility, the reverse would be,
$$\xi = \xi(x,y,z) \quad,\quad \eta = \eta(x,y,z) \quad,\quad \zeta = \zeta(x,y,z)$$ 

%Seperator
%Seperator
%Seperator
%Seperator
\subsection{Transformation Properties}
\begin{comment}
Not Complete, Crammer's rule for all of the matrices are not writte down
\end{comment}
Based on chain rule for the curvilinear coordinates,
$$dx = x_{\xi}d\xi + x_{\eta}d\eta + x_{\zeta}d\zeta \quad,\quad dy = y_{\xi}d\xi + y_{\eta}d\eta + y_{\zeta}d\zeta \quad,\quad dz = z_{\xi}d\xi + z_{\eta}d\eta + z_{\zeta}d\zeta$$
Putting into matrix form,
\begin{equation}\begin{bmatrix}
dx \\ dy \\ dz
\end{bmatrix} = \begin{bmatrix}
x_{\xi} && x_{\eta} && x_{\zeta} \\
y_{\xi} && y_{\eta} && y_{\zeta} \\
z_{\xi} && z_{\eta} && z_{\zeta} \\
\end{bmatrix}\begin{bmatrix}
d\xi \\ d\eta \\ d\zeta
\end{bmatrix}
\label{curvilinear coordinate matrix}
\end{equation}
Based on chain rule for the uniform coordinates,
$$d\xi = \xi_{x}dx + \xi_{y}dy + \xi_{z}dz \quad,\quad d\eta = \eta_{x}dx + \eta_{y}dy + \eta_{z}dz \quad,\quad d\zeta = \zeta_{x}dx + \zeta_{y}dy + \zeta_{z}dz$$
Putting into matrix form,
\begin{equation}\begin{bmatrix}
d\xi \\ d\eta \\ d\zeta
\end{bmatrix} = \begin{bmatrix}
\xi_{x} && \xi_{y} && \xi_{z} \\
\eta_{x} && \eta_{y} && \eta_{z} \\
\zeta_{x} && \zeta_{y} && \zeta_{z} \\
\end{bmatrix}\begin{bmatrix}
dx \\ dy \\ dz
\end{bmatrix}
\label{uniform coordinate matrix}
\end{equation}
Substituting equation $\ref{uniform coordinate matrix}$ into equation $\ref{curvilinear coordinate matrix}$,
$$\begin{bmatrix}
dx \\ dy \\ dz
\end{bmatrix} = \begin{bmatrix}
x_{\xi} && x_{\eta} && x_{\zeta} \\
y_{\xi} && y_{\eta} && y_{\zeta} \\
z_{\xi} && z_{\eta} && z_{\zeta} \\
\end{bmatrix}\begin{bmatrix}
\xi_{x} && \xi_{y} && \xi_{z} \\
\eta_{x} && \eta_{y} && \eta_{z} \\
\zeta_{x} && \zeta_{y} && \zeta_{z} \\
\end{bmatrix}\begin{bmatrix}
dx \\ dy \\ dz
\end{bmatrix}$$
The only way the expression above is true is if
$$\begin{bmatrix}
x_{\xi} && x_{\eta} && x_{\zeta} \\
y_{\xi} && y_{\eta} && y_{\zeta} \\
z_{\xi} && z_{\eta} && z_{\zeta} 
\end{bmatrix}\begin{bmatrix}
\xi_{x} && \xi_{y} && \xi_{z} \\
\eta_{x} && \eta_{y} && \eta_{z} \\
\zeta_{x} && \zeta_{y} && \zeta_{z}
\end{bmatrix} = \begin{bmatrix}
1 && 0 && 0 \\
0 && 1 && 0 \\
0 && 0 && 1
\end{bmatrix}$$
Let 
$$A_{1} = \begin{bmatrix}
x_{\xi} && x_{\eta} && x_{\zeta} \\
y_{\xi} && y_{\eta} && y_{\zeta} \\
z_{\xi} && z_{\eta} && z_{\zeta} 
\end{bmatrix} \quad,\quad A_{2} = \begin{bmatrix}
\xi_{x} && \xi_{y} && \xi_{z} \\
\eta_{x} && \eta_{y} && \eta_{z} \\
\zeta_{x} && \zeta_{y} && \zeta_{z}
\end{bmatrix}$$
The determinant of the identity matrix is $1$. Therefore, 
$$det(A_{1})\times det(A_{2}) = 1$$
$$det(A_{1}) = \frac{1}{det(A_{2})}$$
Let the jacobian $J$ be defined as,
$$J = det(A_{2})$$
Therefore,
$$det(A_{1}) = \frac{1}{J}$$
$$\frac{1}{det(A_{1})} = J$$
The matrix equation above is actually $3$ systems of $3\times3$ linear systems. Parsing out for the partial derivatives in $x$,
$$\begin{bmatrix}
x_{\xi} && x_{\eta} && x_{\zeta} \\
y_{\xi} && y_{\eta} && y_{\zeta} \\
z_{\xi} && z_{\eta} && z_{\zeta} \\
\end{bmatrix}\begin{bmatrix}
\xi_{x} \\
\eta_{x} \\
\zeta_{x} \\
\end{bmatrix} = \begin{bmatrix}
1 \\
0 \\
0 
\end{bmatrix}$$
%Look at this procedure
Crammer rule could be used to solve the matrix above,
$$\xi_{x} = \frac{\displaystyle \begin{vmatrix}
1 && x_{\eta} && x_{\zeta} \\
0 && y_{\eta} && y_{\zeta} \\
0 && z_{\eta} && z_{\zeta} \\
\end{vmatrix}}{\displaystyle \begin{vmatrix}
x_{\xi} && x_{\eta} && x_{\zeta} \\
y_{\xi} && y_{\eta} && y_{\zeta} \\
z_{\xi} && z_{\eta} && z_{\zeta} \\
\end{vmatrix}} = \frac{1}{\displaystyle \begin{vmatrix}
x_{\xi} && x_{\eta} && x_{\zeta} \\
y_{\xi} && y_{\eta} && y_{\zeta} \\
z_{\xi} && z_{\eta} && z_{\zeta} \\
\end{vmatrix}}\begin{vmatrix}
1 && x_{\eta} && x_{\zeta} \\
0 && y_{\eta} && y_{\zeta} \\
0 && z_{\eta} && z_{\zeta} \\
\end{vmatrix} = \frac{1}{det(A_{1})}\begin{vmatrix}
1 && x_{\eta} && x_{\zeta} \\
0 && y_{\eta} && y_{\zeta} \\
0 && z_{\eta} && z_{\zeta} \\
\end{vmatrix}$$
$$\xi_{x} = J(y_{\eta}z_{\zeta} - y_{\zeta}z_{\eta})$$
%Do the procedure above for the rest below
Parsing out for the partial derivatives in $y$,
$$\begin{bmatrix}
x_{\xi} && x_{\eta} && x_{\zeta} \\
y_{\xi} && y_{\eta} && y_{\zeta} \\
z_{\xi} && z_{\eta} && z_{\zeta} \\
\end{bmatrix}\begin{bmatrix}
\xi_{y} \\
\eta_{y} \\
\zeta_{y} \\
\end{bmatrix} = \begin{bmatrix}
0 \\
1 \\
0
\end{bmatrix}$$
Parsing out for the partial derivatives in $z$,
$$\begin{bmatrix}
x_{\xi} && x_{\eta} && x_{\zeta} \\
y_{\xi} && y_{\eta} && y_{\zeta} \\
z_{\xi} && z_{\eta} && z_{\zeta} \\
\end{bmatrix}\begin{bmatrix}
\xi_{z} \\
\eta_{z} \\
\zeta_{z} 
\end{bmatrix} = \begin{bmatrix}
0 \\
0 \\
1
\end{bmatrix}$$


%Seperator
%Seperator
%Seperator
%Seperator
\subsection{Curvilinear to Uniform Transformation}
\begin{comment}
\end{comment}
The Navier-Stokes equations can be expressed in the following vector form,
\begin{equation}
Q = \frac{\partial D}{\partial t} + \frac{\partial B}{\partial x} + \frac{\partial H}{\partial y} + \frac{\partial M}{\partial z}
\label{simplified Navier Stokes vector form}
\end{equation}
wherein $D$, $B$, $H$ and $M$ are all column vectors with the same number of elements inside them. To solve the Navier Stokes equation in the uniform grid space $\xi$, $\eta$, and $\zeta$, the derivatives in $x$, $y$, and $z$ must be transformed. Using chain rule for derivative in $x$,
$$\frac{\partial}{\partial x} = \frac{\partial}{\partial \xi}\times \frac{\partial\xi}{\partial x} + \frac{\partial}{\partial \eta}\times\frac{\partial\eta}{\partial x} + \frac{\partial}{\partial \zeta}\times\frac{\partial \zeta}{\partial x}$$
Since the basic multiplication operation is commutative, the expression above can be rewritten as,
$$\frac{\partial}{\partial x} =  \frac{\partial\xi}{\partial x}\frac{\partial}{\partial \xi} + \frac{\partial\eta}{\partial x}\frac{\partial}{\partial \eta} + \frac{\partial \zeta}{\partial x}\frac{\partial}{\partial \zeta}$$
Using subscripts to imply partial derivative operations,
$$\frac{\partial}{\partial x} =  \xi_{x}\frac{\partial}{\partial \xi} + \eta_{x}\frac{\partial}{\partial \eta} + \zeta_{x}\frac{\partial}{\partial \zeta}$$
Using chain rule for derivative in $y$, and applying the same treatments as for derivative in $x$,
$$\frac{\partial}{\partial y} = \frac{\partial}{\partial \xi}\frac{\partial\xi}{\partial y} + \frac{\partial}{\partial \eta}\frac{\partial \eta}{\partial y} + \frac{\partial}{\partial \zeta}\frac{\partial\zeta}{\partial y}$$
$$\frac{\partial}{\partial y} = \frac{\partial\xi}{\partial y}\frac{\partial}{\partial \xi} + \frac{\partial \eta}{\partial y}\frac{\partial}{\partial \eta} + \frac{\partial\zeta}{\partial y}\frac{\partial}{\partial \zeta}$$
$$\frac{\partial}{\partial y} = \xi_{y}\frac{\partial}{\partial \xi} + \eta_{y}\frac{\partial}{\partial \eta} + \zeta_{y}\frac{\partial}{\partial \zeta}$$
Using chain rule for derivative in $z$, and applying the same treatments as for derivative in $x$,
$$\frac{\partial}{\partial z} = \frac{\partial}{\partial\xi}\frac{\partial\xi}{\partial z} + \frac{\partial}{\partial\eta}\frac{\partial\eta}{\partial z} + \frac{\partial}{\partial\zeta}\frac{\partial\zeta}{\partial z}$$
$$\frac{\partial}{\partial z} = \frac{\partial\xi}{\partial z}\frac{\partial}{\partial\xi} + \frac{\partial\eta}{\partial z}\frac{\partial}{\partial\eta} + \frac{\partial\zeta}{\partial z}\frac{\partial}{\partial\zeta}$$
$$\frac{\partial}{\partial z} = \xi_{z}\frac{\partial}{\partial\xi} + \eta_{z}\frac{\partial}{\partial\eta} + \zeta_{z}\frac{\partial}{\partial\zeta}$$
Substituting all of the partial derivatives into equation $\ref{simplified Navier Stokes vector form}$
\begin{align*}
Q = \frac{\partial D}{\partial t} + \frac{\partial B}{\partial x} + \frac{\partial H}{\partial y} + \frac{\partial M}{\partial z}
\end{align*}
\begin{align*}
Q = \frac{\partial D}{\partial t} & + \xi_{x}\frac{\partial B}{\partial \xi} + \eta_{x}\frac{\partial B}{\partial \eta} + \zeta_{x}\frac{\partial B}{\partial \zeta}
\\ & + \xi_{y}\frac{\partial H}{\partial \xi} + \eta_{y}\frac{\partial H}{\partial \eta} + \zeta_{y}\frac{\partial H}{\partial \zeta}
\\ & + \xi_{z}\frac{\partial M}{\partial\xi} + \eta_{z}\frac{\partial M}{\partial\eta} + \zeta_{z}\frac{\partial M}{\partial\zeta}
\end{align*}
\begin{align*}
Q = \frac{\partial D}{\partial t} & + \xi_{x}\frac{\partial B}{\partial \xi} + \xi_{y}\frac{\partial H}{\partial \xi}
\\ & + \eta_{x}\frac{\partial B}{\partial \eta} + \eta_{y}\frac{\partial H}{\partial \eta} + \eta_{z}\frac{\partial M}{\partial\eta} + \zeta_{y}\frac{\partial H}{\partial \zeta}
\\ & + \xi_{z}\frac{\partial M}{\partial\xi} + \zeta_{z}\frac{\partial M}{\partial\zeta} + \zeta_{x}\frac{\partial B}{\partial \zeta}
\end{align*}
\begin{align*}
Q = \frac{\partial D}{\partial t} & + \xi_{x}\frac{\partial B}{\partial \xi} + \xi_{y}\frac{\partial H}{\partial \xi} + \xi_{z}\frac{\partial M}{\partial\xi}
\\ & + \eta_{x}\frac{\partial B}{\partial \eta} + \eta_{y}\frac{\partial H}{\partial \eta} + \eta_{z}\frac{\partial M}{\partial\eta}
\\ & + \zeta_{y}\frac{\partial H}{\partial \zeta} + \zeta_{z}\frac{\partial M}{\partial\zeta} + \zeta_{x}\frac{\partial B}{\partial \zeta}
\end{align*}
\begin{align*}
Q = \frac{\partial D}{\partial t} & + \xi_{x}\frac{\partial B}{\partial \xi} + \xi_{y}\frac{\partial H}{\partial \xi} + \xi_{z}\frac{\partial M}{\partial\xi}
\\ & + \eta_{x}\frac{\partial B}{\partial \eta} + \eta_{y}\frac{\partial H}{\partial \eta} + \eta_{z}\frac{\partial M}{\partial\eta}
\\ & + \zeta_{x}\frac{\partial B}{\partial \zeta} + \zeta_{y}\frac{\partial H}{\partial \zeta} + \zeta_{z}\frac{\partial M}{\partial\zeta}
\end{align*}
Since $\xi$, $\eta$, and $\zeta$ are all in terms of $x$, $y$, and $z$, then the partial derivatives $\xi_{x}$, $\eta_{y}$ and other similar terms must also be purely in terms of $x$, $y$, and $z$. Therefore,
\begin{align*}
Q = \frac{\partial D}{\partial t} & + \frac{\partial}{\partial \xi}\left[\xi_{x}B + \xi_{y}H + \xi_{z}M\right]
\\ & + \frac{\partial}{\partial \eta}\left[\eta_{x}B + \eta_{y}H + \eta_{z}M\right]
\\ & + \frac{\partial}{\partial \zeta}\left[\zeta_{x}B + \zeta_{y}H + \zeta_{z}M\right]
\end{align*}
The Jacobian $J$ can be defined in terms of $x$, $y$, and $z$ or using $\xi$, $\eta$, and $\zeta$ depending on whether the Jacobian is expressed using matrix $A_{1}$ or matrix $A_{2}$. Let  $J_{ax}$ be the Jacobian expressed in terms of $x$, $y$, and $z$.
\begin{align*}
\frac{Q}{J_{ax}} = \frac{\partial }{\partial t}\left[\frac{D}{J_{ax}}\right] & + \frac{\partial}{\partial \xi}\left[\frac{1}{J_{ax}}\left(\xi_{x}B + \xi_{y}H + \xi_{z}M\right)\right]
\\ & + \frac{\partial}{\partial \eta}\left[\frac{1}{J_{ax}}\left(\eta_{x}B + \eta_{y}H + \eta_{z}M\right)\right]
\\ & + \frac{\partial}{\partial \zeta}\left[\frac{1}{J_{ax}}\left(\zeta_{x}B + \zeta_{y}H + \zeta_{z}M\right)\right]
\end{align*}
The jacobian $J_{ax}$ expressed in terms of $x$, $y$, and $z$, can "move" in and out of the partial derivative operations because the partial derivative operations are purely in terms of $\xi$, $\eta$ and $\zeta$.  Let a change of variables be defined as shown below,
\begin{equation}
\hat{Q} = \frac{Q}{J_{ax}} \quad,\quad \hat{D} = \frac{D}{J_{ax}} \quad,\quad \hat{B} = \frac{1}{J_{ax}}\left(\xi_{x}B + \xi_{y}H + \xi_{z}M\right) 
\label{QDBHM transformation def 1}
\end{equation}
\begin{equation}
\hat{H} = \frac{1}{J_{ax}}\left(\eta_{x}B + \eta_{y}H + \eta_{z}M\right) \quad,\quad \hat{M} = \frac{1}{J_{ax}}\left(\zeta_{x}B + \zeta_{y}H + \zeta_{z}M\right)
\label{QDBHM transformation def 2}
\end{equation}
Substituting,
\begin{align*}
\hat{Q} = \frac{\partial }{\partial t}\left[\hat{D}\right] + \frac{\partial}{\partial \xi}\left[\hat{B}\right] + \frac{\partial}{\partial \eta}\left[\hat{H}\right] + \frac{\partial}{\partial \zeta}\left[\hat{M}\right]
\end{align*}
Therefore, the Navier-Stokes equation in vector form on the uniform grid can be rewritten as shown below,
\begin{equation}
\hat{Q} = \frac{\partial \hat{D}}{\partial t} + \frac{\partial \hat{B}}{\partial \xi} + \frac{\partial \hat{H}}{\partial \eta} + \frac{\partial \hat{M}}{\partial \zeta}
\label{Uniform grid Navier Stokes equation vector form}
\end{equation}
wherein the transformations necessary to convert equation $\ref{simplified Navier Stokes vector form}$ into equation $\ref{Uniform grid Navier Stokes equation vector form}$ is equation $\ref{QDBHM transformation def 1}$ and equation $\ref{QDBHM transformation def 2}$.

%Seperator
%Seperator
%Seperator
%Seperator
\subsection{Uniform to Curvilinear Transformation}
\begin{comment}
Crammer's rule not complete yet
\end{comment}
Reiterating the definition for the transformed variables $\hat{Q}$, $\hat{D}$, $\hat{B}$, $\hat{H}$, $\hat{M}$,
$$\hat{Q} = \frac{Q}{J_{ax}} \quad,\quad \hat{D} = \frac{D}{J_{ax}} \quad,\quad \hat{B} = \frac{1}{J_{ax}}\left(\xi_{x}B + \xi_{y}H + \xi_{z}M\right) $$
$$\hat{H} = \frac{1}{J_{ax}}\left(\eta_{x}B + \eta_{y}H + \eta_{z}M\right) \quad,\quad \hat{M} = \frac{1}{J_{ax}}\left(\zeta_{x}B + \zeta_{y}H + \zeta_{z}M\right)$$
Transforming from uniform grid quantity $\hat{Q}$ and $\hat{D}$ is simple,
$$\hat{Q} = \frac{Q}{J_{ax}} \quad,\quad \hat{D} = \frac{D}{J_{ax}} $$
\begin{equation}
J_{ax}\hat{Q} = Q \quad,\quad J_{ax}\hat{D} = D
\label{QDhat -> QD normal}
\end{equation}
The other quantities can be re-arranged into a matrix equation shown below,
$$\begin{bmatrix}
\hat{B} \\ \hat{H} \\ \hat{M}
\end{bmatrix} = \frac{1}{J_{ax}}\begin{bmatrix}
\xi_{x} && \xi_{y} && \xi_{z} \\
\eta_{x} && \eta_{y} && \eta_{z} \\
\zeta_{x} && \zeta_{y} && \zeta_{z} \\
\end{bmatrix}\begin{bmatrix}
B \\ H \\ M
\end{bmatrix}$$
The $3\times3$ system could be solved using Crammer's rule to expess $B$, $H$, and $M$ in terms of $\hat{B}$, $\hat{H}$ and $\hat{M}$.



%Seperator
%Seperator
%Seperator
%Seperator
%Seperator
\end{center}

\end{document}
