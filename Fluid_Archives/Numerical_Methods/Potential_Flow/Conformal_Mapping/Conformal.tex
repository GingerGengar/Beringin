%Type of Document
\documentclass[class=report, 12pt, crop=false]{standalone}

%Load Pre-ambles
\usepackage{../Environment/Packages}
\usepackage{../Environment/Conventions}
\usepackage{../Environment/Hyahoos}

\begin{document}

\begin{center}


%Seperator
%Seperator
%Seperator
\section{Problem 1}
\begin{comment}
\end{comment}
%Seperator
%Seperator
\subsection{Part a}
The Joukowski transformation is shown below,
$$z = \zeta + \f{c^2}{\zeta}$$
wherein $\zeta$ represents complex numbers. Analyzing the complex numbers purely in terms of of its real and imaginary components, let
$$\zeta = x + iy$$
Substituting $\zeta$,
$$z = x + iy + \f{c^2}{x + iy} = x + iy + \f{c^2}{x + iy}\times\f{x-iy}{x-iy} = x + iy + \f{c^2(x-iy)}{(x + iy)(x-iy)}$$
$$z = x + iy + \f{c^2x-ic^2y}{x^2-(iy)^2} = x + iy + \f{c^2x-ic^2y}{x^2+y^2} = x + iy + \f{c^2}{x^2+y^2}x - \f{c^2}{x^2+y^2}iy$$
$$z = x + \f{c^2}{x^2+y^2}x + iy - \f{c^2}{x^2+y^2}iy = \l[1+\f{c^2}{x^2+y^2}\r]x + \l[1-\f{c^2}{x^2+y^2}\r]yi$$
wherein $x$ and $y$ represents the real and imaginary components respectively of the input to the Joukowski transformation. Manipulating the Joukowski transformation to non-dimensionalize $x$ and $y$,
$$z = \l[1+\f{1}{(x/c)^2+(y/c)^2}\r]x + \l[1-\f{1}{(x/c)^2+(y/c)^2}\r]yi$$
$$z/c = \l[1+\f{1}{(x/c)^2+(y/c)^2}\r]x/c + \l[1-\f{1}{(x/c)^2+(y/c)^2}\r]iy/c$$


The circle in the $\zeta$ plane is defined below,
$$\l(\f{x}{c}\r)^2 + \l(\f{y}{c}-\f{y_s}{c}\r)^2 = \l(\f{r}{c}\r)^2$$
wherein $\dst{\f{y_s}{c}}$ and $\dst{\f{r}{c}}$ is defined as, $\dst{\f{y_s}{c} = 0.08}$ and $\dst{\f{r}{c} = 1.1}$
The relation between $x$ and $y$ in the $\zeta$ plane are expressed in non-dimensionalized length scales with respect to $c$. Since the points in the circle of $\zeta$ plane correspond to the inputs to the Joukowski transformation to produce the airfoil in the $z$ plane, the non-dimensional lengths $x/c$ and $y/c$ in the expression for the circle in the $\zeta$ plane is identical to the non-dimensional lengths in the Joukowski transformation expression. Let the change of notation be denoted below,
$$x_c = \f{x}{c}\quad,\quad y_c = \f{y}{c}\quad,\quad z_c = \f{z}{c}$$


The leading and trailing edges of the airfoil in the $z$-plane is defined to be the where the airfoil crosses the real axis in the $z$-plane. To find the leading and trailing edge of the airfoil, the imaginary component in the Joukowski transformation is set to zero.
$$Im[z_c] = \l[1-\f{1}{(x/c)^2+(y/c)^2}\r]y/c = 0$$
Either both of these statements hence must be true,
$$0 = 1-\f{1}{(x/c)^2+(y/c)^2} \quad,\quad 0 = y/c$$
Based on the problem definition, 
$$0 = y_c$$

Reiterating the circle in the $\zeta$ plane,
$$x_c^2 + \l(y_c-y_{sc}\r)^2 = r_c^2$$
To find the $x_c$ in the $\zeta$ plane corresponding to the leading and trailing edge of the airfoil in the $z$ plane, the value for $y_c$ is substituted into the circle expression,
$$x_c^2 + \l(-y_{sc}\r)^2 = r_c^2$$
$$x_c^2 = r_c^2 - \l(y_{sc}\r)^2$$
$$x_c = \pm\sqrt{\dst{r_c^2 - y_{sc}^2}}$$

On the trailing and leading edge, substituting $0 = y_c$ into the expression for the Joukowski transformation leads the imaginary component of $z_c$ to be zero. The substituted expression is shown below,
$$z/c = \l[1+\f{1}{(x/c)^2+(y/c)^2}\r]x/c$$
$$z/c = \l[1+\f{1}{(x/c)^2}\r]x/c$$
$$z_c = \l[1+\f{1}{(x_c)^2}\r]x_c = \l[x_c+\f{x_c}{(x_c)^2}\r]$$
$$z_c = x_c+\f{1}{x_c}$$
Let $z_{cl}$ represent the complex number at the leading edge and $z_{ct}$ represent the complex number at the trailing edge. Since both $z_{cl}$ and $z_{ct}$ are non-dimensionalized by $c$, the non-dimensional chord length could be expressed as,
$$l_c = z_{ct} - z_{cl}$$

For the leading edge,
$$x_{cl} = -\sqrt{\dst{r_c^2 - y_{sc}^2}}$$
$$z_{cl} = x_{cl}+\f{1}{x_{cl}}$$
Substituting $x_{cl}$,
$$z_{cl} = -\sqrt{\dst{r_c^2 - y_{sc}^2}}-\f{1}{\sqrt{\dst{r_c^2 - y_{sc}^2}}}$$

For the trailing edge,
$$x_{ct} = \sqrt{\dst{r_c^2 - y_{sc}^2}}$$
$$z_{ct} = x_{ct}+\f{1}{x_{ct}}$$
Substituting $x_{ct}$,
$$z_{ct} = \sqrt{\dst{r_c^2 - y_{sc}^2}}+\f{1}{\sqrt{\dst{r_c^2 - y_{sc}^2}}}$$

Susbtituting to obtain $l_c$,
$$l_c = z_{ct} - z_{cl} = \sqrt{\dst{r_c^2 - y_{sc}^2}}+\f{1}{\sqrt{\dst{r_c^2 - y_{sc}^2}}}-\l[-\sqrt{\dst{r_c^2 - y_{sc}^2}}-\f{1}{\sqrt{\dst{r_c^2 - y_{sc}^2}}}\r]$$
$$l_c = \sqrt{\dst{r_c^2 - y_{sc}^2}}+\f{1}{\sqrt{\dst{r_c^2 - y_{sc}^2}}} + \sqrt{\dst{r_c^2 - y_{sc}^2}} + \f{1}{\sqrt{\dst{r_c^2 - y_{sc}^2}}} = 2\sqrt{\dst{r_c^2 - y_{sc}^2}} + \f{2}{\sqrt{\dst{r_c^2 - y_{sc}^2}}}$$
$$l_c = 2\l[\sqrt{\dst{r_c^2 - y_{sc}^2}} + \f{1}{\sqrt{\dst{r_c^2 - y_{sc}^2}}}\r]$$
Substituting for values $r_c = 1.1$ and $y_{sc} = 0.08$,
$$l_c = 4.01718349638$$
%Seperator
%Seperator
\subsection{Part b}
It is important to note that in the previous part, assuming that the leading edge and the trailing edge is defined as the point where the airfoil crosses the real axis of the $z$-plane, the corresponding points in the $\zeta$ plane is the point where the circle crosses the real axis.The diagram below shows potential flow over the cylinder in the complex $\zeta$ plane. 
%\\~\\\includegraphics[scale=\sizem]{Pecker.jpg}
\\~\\The free-stream velocity is $u_\infty$ and comes at the cylinder at an angle of attack $\a$. The leading and trailing edges of the airfoil in the $z$ plane  correspond to the points labelled $x_{c1}$ and $x_{c2}$ respectively. The points $a_1$ and $b_1$ represent the stagnation points on the circle when there is no circulation about the center of the circle. The points $a_2$ and $b_2$ represent the stagnation points on the circle when there is circulation about the center of the circle.
\\~\\The circulation strength is set such that one of the stagnation points on the circle would coincide with point $x_{c2}$. The point $x_{c2}$ in the $\zeta$ complex plane corresponds to the trailing edge of the airfoil in the $z$-plane. By setting the circulation such that a stagnation point would occur at the point of the trailing edge adjusts the potential flow to satisfy the Kutta condition. By observing the diagram, one can find the definition of $\be$ in terms of $\a$ and $\g$,

%Include definition for angle beta
$$\be = \a + \g$$
We can then express $\g$ in terms of the known variables $r_c$ and $y_sc$,
$$r_c\sin\g = y_{sc}$$
Manipulating to make $\g$ subject of the expression,
$$\sin\g = \f{y_{sc}}{r_c}$$
$$\g = \arcsin\l(\f{y_{sc}}{r_c}\r)$$
Substituting $\g$ into the expression for $\be$,
$$\be = \a + \g = \a + \arcsin\l(\f{y_{sc}}{r_c}\r)$$

%Include sin beta = whatever
Applying the Milne-Thompson circle theorem on angled uniform free-stream flow, and then adding a free-stream vortex to the origin of the circle to set the stagnation points away from their original positions yield the following expression,
$$\sin\be = \f{\Gamma_a}{4\pi r U_\infty}$$
wherein $\Gamma_a$ represents the strength of the free-stream vortex and $r$ represents the radius of the circle when applying the Milne-Thompson circle theorem. The strength of the free-stream vortex will be important in determining lift. Manipulating the expression above to make $\Gamma_a$ subject of the expression,
$$4\pi r U_\infty\sin\be = \Gamma_a$$

%Include definition for lift
According to the Kutta-Joukowski theorem, the lift per unit span $L_s$ of an airfoil is given by the following expression,
$$L_s = \rho U_\infty \Gamma_a$$

%Include definition for c_L
The $2$-dimensional lift coefficient is defined as,
$$c_L = \f{L_s}{\dst{\f{1}{2}\rho U_\infty^2 l}}$$
wherein $l$ represents the chord length of the airfoil. Substituting for $L_s$ in terms of the strength of the free-stream vortex $\Gamma_a$,
$$c_L = \f{\rho U_\infty \Gamma_a}{\dst{\f{1}{2}\rho U_\infty^2 l}} = \f{ \Gamma_a}{\dst{\f{1}{2} U_\infty l}} = \f{2\Gamma_a}{\dst{U_\infty l}}$$

%Substitute all that stuff to form the function c_L = f(\a)
Substituting the circulation of the free-stream vortex in terms of $\sin\be$,
$$c_L = \f{2\Gamma_a}{\dst{U_\infty l}} = \f{8\pi r U_\infty\sin\be}{\dst{U_\infty l}} = \f{8\pi r \sin\be}{\dst{ l}}$$
Non-dimensionalizing the radius of the cylinder and the airfoil to our previous work compatible,
$$c_L = \f{8\pi (r/c) \sin\be}{\dst{(l/c)}} = \f{8\pi r_c }{\dst{l_c}}\sin\be$$

Susbtituting the expression for $\be$ in terms of the location of the non-dimensional circle center $y_{sc}$ and non-dimensional circle radius $r_c$ in the $\zeta$ plane,
$$c_L = \f{8\pi r_c }{\dst{l_c}}\sin\l[\a + \arcsin\l(\f{y_{sc}}{r_c}\r)\r]$$
wherein $l_c$ was the non-dimensional chord length which was determined earlier. Substituting numerical values, $r_c=1.1$, $l_c=4.01718349638$, and $y_{sc}=0.08$,
$$c_L = (6.8819398906)\sin\l[\a + 0.072791538003\r]$$

%Seperator
%Seperator
\subsection{Part c}
The coefficient of lift is zero when,
$$c_L = 0 = \sin(0)$$
Therefore,
$$0 = \a_{zero} + \arcsin\l(\f{y_{sc}}{r_c}\r)$$
$$\a_{zero} =  -\arcsin\l(\f{y_{sc}}{r_c}\r)$$
Substituting for $y_{sc}$ and $r_c$,
$$\a_{zero} = -0.072791538003\,rad$$

%Seperator
%Seperator
\subsection{Part d}

%Use the definition for beta
Reiterating the definition for the angle $\be$,
$$\be = \a + \arcsin\l(\f{y_{sc}}{r_c}\r)$$

%Use definition of stagnation point in angled-free stream around cylinder
Let $\zeta_{s1}$ represent the front stagnation point around the cylinder and $\zeta_{s2}$ represent the back stagnation point around the cylinder. When there is no free-stream vortex nested at the center of the circle, the stagnation point $\zeta_{s1}$,
$$\zeta_{s1} = r_ce^{i(\pi+\a)} + iy_{sc}$$
When there is a free-stream vortex at the center of the circle of strength $\Gamma_a$, the stagnation point $\zeta_{s1}$ has shifted by an angle $\be$ in the counter-clockwise direction,
$$\zeta_{s1} = r_ce^{i(\pi+\a+\be)} + iy_{sc}$$
Using the polar definition of complex numbers, it is simple to prove the following is true,
$$e^{\pi i} = -1$$
Substituting,
$$\zeta_{s1} = r_ce^{\pi i}e^{i(\a+\be)} + iy_{sc} = -r_ce^{i(\a+\be)} + iy_{sc}$$
Expressing $\zeta_{s1}$ in the cartesian definition of complex numbers,
$$\zeta_{s1} = -r_c\cos\l[\a+\be\r] - r_ci\sin\l[\a+\be\r] + iy_{sc}$$
$$\zeta_{s1} = -r_c\cos\l[\a+\be\r] + i\l\{y_{sc} - r_c\sin\l[\a+\be\r]\r\}$$
Substituting for the definition of $\be$,
$$\zeta_{s1} = -r_c\cos\l[\a+\a + \arcsin\l(\f{y_{sc}}{r_c}\r)\r] + i\l\{y_{sc} - r_c\sin\l[\a+\a + \arcsin\l(\f{y_{sc}}{r_c}\r)\r]\r\}$$
$$\zeta_{s1} = -r_c\cos\l[2\a + \arcsin\l(\f{y_{sc}}{r_c}\r)\r] + i\l\{y_{sc} - r_c\sin\l[2\a + \arcsin\l(\f{y_{sc}}{r_c}\r)\r]\r\}$$
Substituting the for the angle $\a=0.0872664626\,rad$, and the other known variables, $r_c=1.1$, $l_c=4.01718349638$, and $y_{sc}=0.08$,
$$\zeta_{s1} = -1.06652798048 - 0.189291787589i$$

%Use Joukowski to then find the corresponding point in z plane
Applying the Kutta Joukowski transformation to $\zeta_{s1}$,
$$z_{cs1} = \l[1+\f{1}{(x/c)^2+(y/c)^2}\r]x/c + \l[1-\f{1}{(x/c)^2+(y/c)^2}\r]iy/c$$
$$z_{cs1} = -1.97551620021 - 0.02796080691i$$

%Seperator
%Seperator
%Seperator
\section{Problem 2}
\begin{comment}
\end{comment}
%Seperator
%Seperator
\subsection{Part a}
The airfoil in the $z$-plane is shown below,
%\\~\\\includegraphics[scale=\size]{Airfoil_Geometry.png}
%Seperator
%Seperator
\subsection{Part b}
The definition for the angle $\be$ for this case,
$$\be = \a + \arctan\l(\f{\eta_0-\eta_t}{\xi_t-\xi_0}\r)$$
The definition for circulation given angle $\be$ from the previous case is shown below,
$$\Gamma_a = 4\pi r U_\infty\sin\be$$
Substituting the angle $\be$,
$$\Gamma_a = 4\pi r U_\infty\sin\l[\a + \arctan\l(\f{\eta_0-\eta_t}{\xi_t-\xi_0}\r)\r]$$
$$\f{\Gamma_a}{U_\infty} = 4\pi r \sin\l[\a + \arctan\l(\f{\eta_0-\eta_t}{\xi_t-\xi_0}\r)\r]$$
%Seperator
%Seperator
\subsection{Part c}
If the influence of gravity is assumed to be negligible within the scope of the fluid in the problem, then Bernoulli's equation,
$$p_\infty + \f{1}{2}\rho U_{\infty}^2 = p + \f{1}{2}\rho |W|^2$$
Re-arranging Bernoulli's equation,
$$\f{1}{2}\rho U_{\infty}^2 - \f{1}{2}\rho |W|^2 = p - p_\infty$$
wherein $|W|$ is the magnitude of the local fluid velocity. Coefficient of pressure is typically defined as,
$$c_p = \f{p-p_{\infty}}{\f{1}{2}\rho U_{\infty}^2}$$
Substituting for $p-p\infty$ and assuming that the fluid is incompressible,
$$c_p = \f{\f{1}{2}\rho U_{\infty}^2 - \f{1}{2}\rho |W|^2}{\f{1}{2}\rho U_{\infty}^2}$$
Simplifying,
$$c_p = \f{\f{1}{2}\rho U_{\infty}^2}{\f{1}{2}\rho U_{\infty}^2} - \f{\f{1}{2}\rho |W|^2}{\f{1}{2}\rho U_{\infty}^2}$$    
$$c_p = 1 - \f{|W|^2}{U_{\infty}^2}$$
$$c_p = 1 - \l[\f{|W|}{U_{\infty}}\r]^2$$
The coefficient of pressure in the $z$-plane,
$$c_{p,z} = 1 - \l[\f{|W_z|}{U_{\infty}}\r]^2$$


Reiterating the complex velocity of the fluid flow around the circle,
$$W_{\zeta} = \f{dF_{\zeta}}{dz} = U_\infty\l(1-\f{r^2}{z^2}\r) + i\f{\Gamma_a}{2\pi z}$$
Non-dimensionalizing the fluid complex velocity to the free-stream velocity,
$$\f{W_{\zeta}}{U_\infty} = 1-\f{r^2}{z^2} + i\f{1}{2\pi z}\l(\f{\Gamma_a}{U_\infty}\r)$$
Here $z$ represents an arbitrary complex number. To apply the expression into the problem, the arbitrary complex number $z$ is a complex number in the $\zeta$ plane. Hence,
$$\f{W_{\zeta}}{U_\infty} = 1-\f{r^2}{\zeta^2} + i\f{1}{2\pi \zeta}\l(\f{\Gamma_a}{U_\infty}\r)$$
Now the flow above represents flow over a cylinder in the $\zeta$ plane that is nested in the origin of the coordinate system. To shift the origin of the cylinder in the $\zeta$ plane,
$$\f{W_{\zeta}}{U_\infty} = 1-\f{r^2}{(\zeta-\zeta_0)^2} + i\f{1}{2\pi (\zeta-\zeta_0)}\l(\f{\Gamma_a}{U_\infty}\r)$$
wherein $\zeta_0$ represents the origin of the cylinder in the $\zeta$ plane. For clarity's sake,
$$\zeta_0 = \xi_0 + i\eta_0$$

The relation between the fluid velocities in the $\zeta$ plane to the $z$ plane is shown below,
$$W_z = \f{W_{\zeta}}{\dst{\l(\f{d\zeta'}{d\zeta}\r)\l(\f{dz}{d\zeta'}\r)}}$$
Non-dimensionalizing fluid velocity by free-stream velocity to make the conformal mapping relations compatible to the complex velocity relations,
$$\f{W_z}{U_\infty} = \f{1}{\dst{\l(\f{d\zeta'}{d\zeta}\r)\l(\f{dz}{d\zeta'}\r)}}\times\f{W_{\zeta}}{U_\infty}$$


The $2$ conformal mappings relating points in the $\zeta$ plane to the $\zeta'$ plane and from the $\zeta'$ plane to the $z$ plane is shown below,
$$\zeta' = \zeta - \f{\epsilon}{\zeta-d} \quad,\quad z = \zeta' + \f{1}{\zeta'}$$
Computing the derivatives,
$$\f{d\zeta'}{d\zeta} = \f{d}{d\zeta}\l(\zeta\r) - \f{d}{d\zeta}\l(\f{\epsilon}{\zeta-d}\r) \quad,\quad \f{dz}{d\zeta'} = \f{d}{d\zeta'}\l(\zeta'\r) + \f{d}{d\zeta'}\l(\f{1}{\zeta'}\r)$$
$$\f{d\zeta'}{d\zeta} = 1 + \f{\epsilon}{\l(\zeta-d\r)^2} \quad,\quad \f{dz}{d\zeta'} = 1 - \f{1}{\zeta'^2}$$




   

     










\end{center}

\end{document}
