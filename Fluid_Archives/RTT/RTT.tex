%Type of Document
\documentclass[class=report, 12pt, crop=false]{standalone}

%Load Pre-ambles
\usepackage{../Environment/Packages}
\usepackage{../Environment/Conventions}
\usepackage{../Environment/Hyahoos}

\begin{document}

\begin{center}
\chapter{Reynold's Transport Theorem}
\begin{comment}
Liebnitz Theorem
\end{comment}
One variation of Liebniz Rule applicable for volumetric integrals is shown below. for the variable $T$ wherein $T$ may represent a time dependent scalar, vector, or tensor.
$$\f{d}{dt}\iiint_{R(t)}T\,dV_o = \iiint_{R(t)}\f{\p}{\p t}[T]dV_o + \iint_{S(t)}T\b{v_s}\b{n}dS$$
wherein $R(t)$ represents an arbitray region of space, $V_o$ represents volume, $S(t)$ represents the surface of the region defined by $R(t)$, $\b{v_s}$ represents the velocity of the moving surface, $\b{n}$ represents normal vector of the surface. Depending on the variable type $T$, the operation $\dst{T\b{v_s}\b{n}}$ would depend on a case to case basis.
%Seperator
%Seperator
%Seperator
%Seperator
%Seperator
\stdsection{Substantive Derivative}
\begin{comment}
Substantive Derivative
\end{comment}
Suppose a quantity $b$ is dependent on the the variable time $t$ and the typical cartesian coordinates $x$, $y$, $z$. Taking the derivative of variable $a$ with respect to time yields the following based on chain rule,
$$\f{d}{dt}[b] = \f{\p}{\p t}[b] + \f{\p}{\p x}[b]\times\f{\p}{\p t}[x] + \f{\p}{\p y}[b]\times\f{\p}{\p t}[y] + \f{\p}{\p z}[b]\times\f{\p}{\p t}[z]$$
Taking note that the partial derivatives of the cartesian coordinates defines velocity in the cartesian coordinates. Therefore, 
$$\f{\p}{\p t}[x] = u\quad,\quad \f{\p}{\p t}[y] = v\quad,\quad \f{\p}{\p t}[z] = w$$
wherein $u$, $v$, and $w$ typically represents velocity in the $x$, $y$, and $z$ directions respectively. Therefore, the derivative of $y$ with respect to time $t$ would take the form,
$$\f{d}{dt}[b] = \f{\p}{\p t}[b] + u\f{\p}{\p x}[b] + v\f{\p}{\p y}[b] + w\f{\p}{\p z}[b]$$
If the $\nabla$ operator is defined as
$$\nabla = \begin{pmatrix} \dst{\f{\p}{\p x}}& \dst{\f{\p}{\p y}}& \dst{\f{\p}{\p z}} \end{pmatrix}^T$$
Therefore, the derivative of $y$ with respect to time $t$ would take the form
$$\f{d}{dt}[b] = \f{\p}{\p t}[b] + u\f{\p}{\p x}[b] + v\f{\p}{\p y}[b] + w\f{\p}{\p z}[b]$$
Let the velocity vector be defined as 
$$\b{v} = \begin{pmatrix}u\\v\\w\\ \end{pmatrix}$$
It follows that the derivative of $b$ with respect to time $t$ would take the form
$$\f{d}{dt}[b] = \f{\p}{\p t}[b] + \b{v}\d\nabla b$$
%Seperator
%Seperator
%Seperator
%Seperator
%Seperator
\section{Divergence Theorem}
\begin{comment}
Divergence Theorem
\end{comment}
The Divergence Theorem is stated below. The variable $\b{F}$ must represent a vector in $R^3$
$$\iiint_{R(t)}  \nabla \d \b{F} \,dV_o = \iint_{S(t)}\b{F}\d \b{n}dS$$
Alternately,
$$\iiint_{R(t)}  \f{\p}{\p x}[\b{F}] + \f{\p}{\p y}[\b{F}] + \f{\p}{\p z}[\b{F}] \,dV_o = \iint_{S(t)}\b{F}\d \b{n}dS$$
wherein $dV_o$ represents an infinitesmially small volume. $S(t)$ is the surface encapsulating the region $R(t)$. $\b{n}$ is the normal vector of the control volume, and $dS$ is an infinitesmial area of surface $S(t)$.
\end{center}

\end{document}
