%Type of Document
\documentclass[a4paper, 12pt]{report}

%Load Pre-ambles
\usepackage{../Environment/Packages}
\usepackage{../Environment/Conventions}
\usepackage{../Environment/Hyahoos}

\begin{document}

\begin{center}
\chapter{Basic Definitions}
\begin{comment}
\end{comment}
%Seperator
%Seperator
%Seperator
%Seperator
%Seperator
\section{Dynamic Variables}
\begin{comment}
\end{comment}
\begin{tabular}{|m{\tabsize}|m{\tabsize}|m{\tabsize}|m{\tabsizel}|}
\hline
Name & Symbollic Representation & Units & Description \\  
\hline
Lift & L & N & Upward force experienced by the aircraft\\   
\hline
Drag & D & N & Backward force experienced by the aircraft\\
\hline
\end{tabular}
%Seperator
%Seperator
%Seperator
%Seperator
%Seperator
\section{Geometrical variables}
\begin{comment}
\end{comment}
\begin{tabular}{|m{\tabsize}|m{\tabsize}|m{\tabsize}|m{\tabsizel}|}
\hline
Name & Symbollic Representation & Units & Description \\                                                  \hline                                                                                                    
Angle of attack & $\a$ & rad & How pitched up or down the wing or horizontal stabilizer is usually, could represent more than just wings or horizontal stabilizers though\\ 
\hline
Leading Edge & - & - & The front-most edge of the airfoil \\
\hline
Trailing Edge & - & - & The back-most edge of the airfoil\\
\hline
Chord length &  & m & Length of the chord line, wherein chord line is a line joining the leading edge and trailing edge\\
\hline
Span length & & m & The sideways length of the wing. The distance between one wing tip to another wing tip\\
\hline
Mean Camber line & & & \\
\hline
Chord line & & & \\
\hline
\end{tabular}
%Seperator
%Seperator
%Seperator
%Seperator
%Seperator
\section{Processed Geometry}
\begin{comment}
\end{comment}
\begin{tabular}{|m{\tabsize}|m{\tabsize}|m{\tabsize}|m{\tabsizel}|}
\hline
Name & Symbollic Representation & Units & Description \\
\hline
Aerodynamic Center &  & - & A specific point in the airfoil wherein the moments acting on the airfoil due to fluid pressures is unchanging with angle of attack \\ 
\hline
Center of Pressure & & - & A specific point in the airfoil wherein the airfoil experiences no resultant moment about this point\\ 
\hline
Neutral Point & & & \\ 
\hline
Aspect Ratio & & & \\ 
\hline
\end{tabular}
%Seperator
%Seperator
%Seperator
%Seperator
%Seperator
\section{Dimensionless Coefficients}
\begin{comment}
\end{comment}
\begin{tabular}{|m{\tabsize}|m{\tabsize}|m{\tabsize}|m{\tabsize}|}
\hline
Name & Symbollic Representation & Units & Description \\
\hline
Coefficient of Lift & & & \\ 
\hline
Coefficient of Drag & & & \\ 
\hline
Coefficient of Moments & & & \\ 
\hline
\end{tabular}
%Seperator
%Seperator
%Seperator
%Seperator
%Seperator
\section{Definition of Processes}
\begin{comment}
\end{comment}
\begin{tabular}{|m{\tabsize}|m{\tabsize}|m{\tabsizel}|}
\hline
Name & Symbollic Representation & Description \\
\hline
Isothermal & $it$ & Constant temperature \\
\hline
Isobaric & $ib$ & Constant pressure \\
\hline
Isochoric & $ic$ & Constant volume \\
\hline
Adiabatic & $ad$ & No heat exchange with external system \\
\hline
Reversible & $rev$ & No dissipative phenomena, no mass diffusion, no thermal conducitvity, no viscosity \\
\hline
Isentropic & $ise$ & Both Adiabatic and Reversible \\
\hline
\end{tabular}
\end{center}

\end{document}
