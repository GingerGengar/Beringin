%Type of Document
\documentclass[class=report, 12pt, crop=false]{standalone}

%Load Pre-ambles
\usepackage{../Environment/Packages}
\usepackage{../Environment/Conventions}
\usepackage{../Environment/Hyahoos}

\begin{document}

\begin{center}
%Seperator
%Seperator
%Seperator
%Seperator
%Seperator
%Seperator
\chapter{Tensors}
\begin{comment}
\end{comment}
%Seperator
%Seperator
%Seperator
%Seperator
%Seperator
\section{Tensor Index Notation}
\begin{comment}
\end{comment}
Tensors are a generalization of scalars, vectors, and matrices. The order of a tensor represents how many 'axis' the tensor has. For example, a scalar would be a $0^{th}$ order tensor meanwhile a vector would be a $1^{st}$ order tensor and a matrix would be $2^{nd}$ order tensor. Tensors of higher orders are permitted though a visual representation of them is meaningless. One can alternatively imagine tensors as multi-dimensional arrays, much like the case in a programming language.
 \\~\\The tensor index notation comprises of $2$ main indices: A free index and a dummy index. A free index corresponds to the positioning of a certain value in a tensor. For example, the $i^{th}$ component of a vector $\bar{v}$ is usually represented as $v_{i}$. That is an example of a free index usage. A dummy index is an index that is used for summation. Dummy indices occur in pairs and a pair of dummy indices imply summation. For example in the case of a dot product, $A_{j}B_{j}$ represents scalar multiplication between the $j^{th}$ components of vectors $\bar{A}$ and $\bar{B}$, added all together for the enirety of the length of vector $\bar{A}$ and vector $\bar{B}$.
\\~\\Since what specific name one gives to a an index is arbitrary, this leads to index renaming rules. Dummy indices may be renamed within a single term. For example $A_{j}B_{j} = A_{i}B_{i}$. Free indices however, must be renamed across all algebraically summed terms. For example, $A_{i}B_{p}C_{p} + D_{i}E_{q}F_{q} = A_{j}B_{p}C_{p} + D_{j}E_{q}F_{q}$
%Seperator
%Seperator
%Seperator
%Seperator
%Seperator
\section{Kronecker-Delta \& Permutation Tensor}
\begin{comment}
\end{comment}
The kronecker-delta is a function that maps $2$ integers to a $1$ or $0$. A mathematical description of the kronecker-delta function $\delta_{ij}$ is shown below,
$$\delta_{ij} = \begin{cases}
1,\quad \text{if } i = j \\
0,\quad \text{if } i\neq j
\end{cases}$$
The permutation tensor $\epsilon_{ijk}$ a $3^{rd}$ order tensor and is anti-symmetric in any $2$ of the indices. The indices can accept a range of integers from $1$ until $3$. Therefore, $\epsilon_{123}$, $\epsilon_{213}$ are both valid but $\epsilon_{352}$ is not. The permutation tensor has a cyclic property described below,
$$\epsilon_{ijk} = \epsilon_{kij} = \epsilon_{jki}$$
Switching any $2$ index of the permutaton tensor makes it negative. This property is described below,
$$\epsilon_{ijk} = -\epsilon_{jik} = -\epsilon_{ikj} = -\epsilon_{kji}$$
The exact value of the permutation tensor,
$$\epsilon_{123} = 1 \quad,\quad \epsilon_{213} = -1$$
The other cases of $i$, $j$, and $k$ are all obtainable by applying the properties above.
%Seperator
%Seperator
%Seperator
%Seperator
%Seperator
\section{Common Vector Operations}
\begin{comment}
\end{comment}
Let $\b{A}$ and $\b{B}$ be vector fields, and $\phi$, $\psi$ be scalar fields. Let $A_{i}$ and $B_{i}$ represent the $i^{th}$ component of the vector $\bar{A}$ and $\bar{B}$ respectively.
%Seperator
%Seperator
%Seperator
%Seperator
\subsection{Scalar Multiplication}
\begin{comment}
\end{comment}
Since scalar multiplication simplt multiples all components of a vector by some scalar,
$$\left[\phi\bar{A}\right]_{i} = \phi A_{i}$$
wherein the $LHS$ represents the vector notation and the $RHS$ represents the index notation equivalent. Note that the $[]_{i}$ is used to denote the $i^{th}$ index of the vector notation.
%Seperator
%Seperator
%Seperator
%Seperator
\subsection{Dot Product}
\begin{comment}
\end{comment}
Dot products can be represented very elegantly in tensor index notation,
$$\bar{A}\cdot\bar{B} = A_{j}B_{j}$$
The repeated index $j$ here makes $j$ a dummy index which is used for counting. A repeated index such as $j$, implies summation. Therefore,
$$A_{j}B_{j} = A_{1}B_{1}+A_{2}B_{2}+A_{3}B_{3}$$
%Seperator
%Seperator
%Seperator
%Seperator
\subsection{Cross Product}
\begin{comment}
\end{comment}
The cross product of $2$ vectors is defined with the permutation tensor,
$$\left[\bar{A}\times\bar{B}\right]_{i} = \epsilon_{ijk}A_{j}B_{k}$$
%Seperator
%Seperator
%Seperator
%Seperator
%Seperator
\section{Tensor Index Identities}
\begin{comment}
\end{comment}
Let $\b{A}$ and $\b{B}$ be vector fields, and $\phi$, $\psi$ be scalar fields. Let $\bar{\bar{\mu}}$ and $\bar{\bar{\gamma}}$ represent second order tensors,
%Seperator
%Seperator
%Seperator
%Seperator
\subsection{Symmetric-Antisymmetric Tensor}
\begin{comment}
\end{comment}
Let $\bar{\bar{\mu}}$ be a symmetric tensor and $\bar{\bar{\gamma}}$ be an anti-symmetric tensor. By the properties of the symmetric and anti-symmetric tensors,
$$\mu_{ij} = \mu_{ji} \quad,\quad \gamma_{ij} = -\gamma_{ji}$$
Consider the following,
$$\mu_{ij}\gamma_{ij} = -\mu_{ji}\gamma_{ji}$$
Here, the dummy indices have been switched, and this is true due to the symemtric and anti-symmetric definitions of $\mu$ and $\gamma$. The dummy indices are renamed, $j \to p$, $i \to q$,
\begin{equation}\mu_{ij}\gamma_{ij} = -\mu_{pq}\gamma_{pq}\label{sym-anti-1}\end{equation}
Next, start with $\bar{\bar{mu}}$ and $\bar{\bar{\gamma}}$ again, but this time rename them based on a different set of variable change. $i \to p$ and $j \to q$. This seems illegal, but it is not. Remember, the naming are arbitrary and we have not violated any of the rules. Therefore,
\begin{equation}\mu_{ij}\gamma_{ij} = \mu_{pq}\gamma_{pq}\label{sym-anti-2}\end{equation}
Susbtituting $\mu_{ij}\gamma_{ij}$ out from euqation $\ref{sym-anti-1}$ and equation $\ref{sym-anti-2}$,
$$\mu_{pq}\gamma_{pq} = -\mu_{pq}\gamma_{pq}$$
Therefore,
$$0 = \mu_{pq}\gamma_{pq}$$
Hence, the element-wise multiplication of a symmetric and anti-symmetric tensor added together for the entire tensor would yield zero.
%Seperator
%Seperator
%Seperator
%Seperator
\subsection{Double Permutation Tensor}
\begin{comment}
\end{comment}
Arguably one of the most important identities for tensor indices,
$$\epsilon_{ijk}\epsilon_{imn} = \delta_{jm}\delta_{kn}-\delta_{jn}\delta_{km}$$
%Seperator
%Seperator
%Seperator
%Seperator
\subsection{Kronecker-Delta Renaming}
\begin{comment}
\end{comment}
The kronecker-delta function can be used to rename the indices of a tensor,
$$\delta_{ij}A_{i} = A_{j}$$
This is because when $i\neq j$, the kronecker-delta function is zero, which means that $\delta_{ij}A_{i}$ is only non-zero when $i = j$, which renames the dummy variable of $i$ in $A_{i}$ into $j$.
%Seperator
%Seperator
%Seperator
%Seperator
\subsection{Curl of Scalar Gradient}
\begin{comment}
\end{comment}
The curl of a scalar gradient is zero,
$$0 = \nabla\times(\nabla\phi)$$
Let ,
$$LHS = 0 \quad,\quad RHS = \nabla\times(\nabla\phi)$$
Converting $LHS$ and $RHS$ into index notation,
$$LHS_i = 0 \quad,\quad RHS_i =  \epsilon_{ijk}\f{\p}{\p x_j}\l[\f{\p \phi}{\p x_k}\r] = \epsilon_{ijk}\f{\p^2 }{\p x_j \p x_k}(\phi)$$
Since partial derivative operators are commutative, $\dst \f{\p^2 }{\p x_j \p x_k}(\phi)$ is a symmetry tensor. If $i$ is held constant, the permutation tensor $\epsilon_{ijk}$ is anti-symmetric. The element-wise multiplication of a symmetric tensor and anti-symmetric tensor added up together yields zero. Therefore,
$$RHS_i = 0$$
Since $LHS_i = RHS_i$, the claim is proven to be true.
%Seperator
%Seperator
%Seperator
%Seperator
\subsection{Divergence of Vector Curl}
\begin{comment}
\end{comment}
The divergence of the curl of a vector field is zero,
$$0 = \nabla\d(\nabla\times \b{A})$$
Let,
$$LHS = 0 \quad,\quad RHS = \nabla\d(\nabla\times \b{A})$$
Converting $LHS$ and $RHS$ into index notation,
$$LHS_i = 0 \quad,\quad RHS_i = \f{\p}{\p x_j}\l[\epsilon_{jkl} \f{\p}{\p x_k}(A_l)\r] = \epsilon_{jkl}\f{\p}{\p x_j}\l[\f{\p}{\p x_k}(A_l)\r] = \epsilon_{jkl}\f{\p^2}{\p x_j\p x_k}(A_l)$$
Since $\epsilon_{jkl}$ is an anti-symmetric tensor and $\dst \f{\p^2}{\p x_j\p x_k}(A_l)$ is a symmetric tensor, then $RHS_i = 0$. Since $LHS_i = RHS_i$, then the claim is proven to be true.
%Seperator
%Seperator
%Seperator
%Seperator
\subsection{Curl of $2$ Vector Cross Products}
\begin{comment}
\end{comment}
$$\nabla\times (\b{A}\times \b{B}) = \b{B}\d\nabla \b{A} + \b{A}\nabla\d \b{B} - \b{A}\d\nabla \b{B} - \b{B}\nabla\d \b{A}$$
Let,
$$LHS = \nabla\times (\b{A}\times \b{B}) \quad,\quad RHS = \b{B}\d\nabla \b{A} + \b{A}\nabla\d \b{B} - \b{A}\d\nabla \b{B} - \b{B}\nabla\d \b{A}$$
Converting $RHS$ into index notation,
$$RHS_i = B_j \f{\p }{\p x_j}(A_i) + A_i \f{\p}{\p x_j}(B_j) - A_j\f{\p}{\p x_j}(B_i) - B_i\f{\p}{\p x_j}(A_j)$$
Converting $LHS$ into index notation,
$$LHS_i = \epsilon_{ijk} \f{\p}{\p x_j}\l[\epsilon_{klm} A_l B_m\r] = \epsilon_{ijk}\epsilon_{klm} \f{\p}{\p x_j}\l[ A_l B_m\r]$$
Using the cylcic permutation property of the permutation tensor $\epsilon_{ijk} = \epsilon_{kij}$. Therefore,
$$\epsilon_{ijk}\epsilon_{klm} = \epsilon_{kij}\epsilon_{klm}$$
Using the double permutation tensor identity,
$$\epsilon_{ijk}\epsilon_{klm} = \epsilon_{kij}\epsilon_{klm} = \delta_{il}\delta_{jm} - \delta_{im}\delta_{jl}$$
Substituting into $LHS_i$,
$$LHS_i = \epsilon_{ijk}\epsilon_{klm} \f{\p}{\p x_j}\l[ A_l B_m\r] = \l[\delta_{il}\delta_{jm} - \delta_{im}\delta_{jl}\r]\f{\p}{\p x_j}\l[ A_l B_m\r] = \delta_{il}\delta_{jm}\f{\p}{\p x_j}\l[ A_l B_m\r] - \delta_{im}\delta_{jl}\f{\p}{\p x_j}\l[ A_l B_m\r]$$
$$LHS_i = \delta_{jm}\f{\p}{\p x_j}\l[ A_i B_m\r] - \delta_{jl}\f{\p}{\p x_j}\l[ A_l B_i\r] = \f{\p}{\p x_j}\l[ A_i B_j\r] - \f{\p}{\p x_j}\l[ A_j B_i\r]$$
Expanding using product rule,
$$LHS_i = A_i\f{\p}{\p x_j}\l[B_j\r] + B_j\f{\p}{\p x_j}\l[A_i\r] - \l\{A_j\f{\p}{\p x_j}\l[B_i\r] + B_i\f{\p}{\p x_j}\l[ A_j\r]\r\}$$
$$LHS_i = A_i\f{\p}{\p x_j}\l[B_j\r] + B_j\f{\p}{\p x_j}\l[A_i\r] - A_j\f{\p}{\p x_j}\l[B_i\r] - B_i\f{\p}{\p x_j}\l[ A_j\r]$$
Since $LHS_i = RHS_i$, the vector identity is proven to be true.
%Seperator
%Seperator
%Seperator
%Seperator
\subsection{Double Curl of Vector}
\begin{comment}
\end{comment}

$$\nabla\times\left(\nabla\times\bar{A}\right) = \nabla\left(\nabla \cdot\bar{A}\right) - \nabla^{2}\bar{A}$$
Let 
$$LHS = \nabla\times\left(\nabla\times\bar{A}\right) \quad,\quad RHS = \nabla\left(\nabla \cdot\bar{A}\right) - \nabla^{2}\bar{A}$$
Converting $LHS$ into index notation,
$$LHS_{i} = \epsilon_{ijk}\frac{\partial}{\partial x_{j}}\epsilon_{kmn}\frac{\partial}{\partial x_{m}}A_{n}$$
Since the permutation tensor $\epsilon_{kmn}$ is a constant in $x_{j}$ and $x_{m}$,
$$LHS_{i} = \epsilon_{ijk}\epsilon_{kmn}\frac{\partial}{\partial x_{j}}\frac{\partial}{\partial x_{m}}A_{n}$$
Using the permutation tensor cyclic identity,
$$\epsilon_{ijk}\epsilon_{kmn} = \epsilon_{kij}\epsilon_{kmn}$$
Using the double permutation tensor identity,
$$\epsilon_{ijk}\epsilon_{kmn} = \epsilon_{kij}\epsilon_{kmn} = \delta_{im}\delta_{jn}-\delta_{in}\delta_{jm}$$
Substituting,
$$LHS_{i} = [\delta_{im}\delta_{jn}-\delta_{in}\delta_{jm}]\frac{\partial}{\partial x_{j}}\frac{\partial}{\partial x_{m}}(A_{n})$$
$$LHS_{i} =  \delta_{im}\delta_{jn}\frac{\partial}{\partial x_{j}}\frac{\partial}{\partial x_{m}}(A_{n})-\delta_{in}\delta_{jm}\frac{\partial}{\partial x_{j}}\frac{\partial}{\partial x_{m}}(A_{n}) $$
Using the renaming identity of the kronecker-delta function,
$$LHS_{i} =  \frac{\partial}{\partial x_{j}}\frac{\partial}{\partial x_{i}}(A_{j}) - \frac{\partial}{\partial x_{j}}\frac{\partial}{\partial x_{j}}(A_{i}) $$
Since partial derivatives are commutative with one another, $\displaystyle \frac{\partial}{\partial x_{j}}\frac{\partial}{\partial x_{i}}(A_{j}) = \frac{\partial}{\partial x_{i}}\frac{\partial}{\partial x_{j}}(A_{j})$. Susbtituting,
$$LHS_{i} = \frac{\partial}{\partial x_{i}}\frac{\partial}{\partial x_{j}}(A_{j}) - \frac{\partial}{\partial x_{j}}\frac{\partial}{\partial x_{j}}(A_{i}) $$
Reiterating $RHS$,
$$RHS = \nabla\left(\nabla \cdot\bar{A}\right) - \nabla^{2}\bar{A}$$
Converting $RHS$ into index notation,
$$RHS_{i} = \frac{\partial}{\partial x_{i}}\frac{\partial}{\partial x_{j}}(A_{j}) - \frac{\partial}{\partial x_{j}}\frac{\partial}{\partial x_{j}}(A_{i})$$
Since $LHS_{i} = RHS_{i}$, then the identity is proven to be true.

%Seperator
%Seperator
%Seperator
%Seperator
\subsection{Curl of Vector Scalar}
\begin{comment}
\end{comment}
$$\nabla\times(\phi\bar{A}) = \phi\nabla\times\bar{A} + (\nabla\phi)\times\bar{A}$$
Let
$$LHS = \nabla\times(\phi\bar{A}) \quad,\quad RHS = \phi\nabla\times\bar{A} + (\nabla\phi)\times\bar{A}$$
Converting $LHS$ into index notation,
$$LHS_{i} = \epsilon_{ijk}\frac{\partial}{\partial x_{j}}(\phi A_{k})$$
Using product rule,
$$LHS_{i} = \epsilon_{ijk}\left[\phi\frac{\partial}{\partial x_{j}}( A_{k}) + A_{k}\frac{\partial}{\partial x_{j}}(\phi )\right]$$
$$LHS_{i} = \left \epsilon_{ijk}\phi\frac{\partial}{\partial x_{j}}( A_{k}) + \epsilon_{ijk}A_{k}\frac{\partial}{\partial x_{j}}(\phi )\right $$
Converting $RHS$ into index notation,
$$RHS_{i} = \phi\epsilon_{ijk}\frac{\partial}{\partial x_{j}}(A_{k}) + \epsilon_{ijk}\left[\frac{\partial}{\partial x_{j}}(\phi)\right]A_{k}$$
$$RHS_{i} = \phi\epsilon_{ijk}\frac{\partial}{\partial x_{j}}(A_{k}) + \epsilon_{ijk}A_{k}\left[\frac{\partial}{\partial x_{j}}(\phi)\right]$$
Since $LHS_{i} = RHS_{i}$, the identity is proven to be true.
%Seperator
%Seperator
%Seperator
%Seperator
\subsection{Triple Curl of Vector}
\begin{comment}
\end{comment}
$$\nabla\times[\nabla\times(\nabla\times\bar{A})] = -\nabla^{2}(\nabla\times\bar{A})$$
Let,
$$LHS = \nabla\times[\nabla\times(\nabla\times\bar{A})] \quad,\quad RHS = -\nabla^{2}(\nabla\times\bar{A})$$
In index notation,
$$(\nabla\times\bar{A})_{i} = \epsilon_{ijk}\frac{\partial}{\partial x_{j}}(A_{k})$$
$$[\nabla\times(\nabla\times\bar{A})]_{l} = \epsilon_{lmi}\frac{\partial}{\partial x_{m}}\left[\epsilon_{ijk}\frac{\partial}{\partial x_{j}}(A_{k})\right]$$
Since $\epsilon_{ijk}$ is simply a constant in $x_{m}$ or $x_{j}$,
$$[\nabla\times(\nabla\times\bar{A})]_{l} = \epsilon_{lmi}\epsilon_{ijk}\frac{\partial}{\partial x_{m}}\left[\frac{\partial}{\partial x_{j}}(A_{k})\right]$$
Using the cyclic property of the permutation tensor,
$$\epsilon_{lmi}\epsilon_{ijk} = \epsilon_{ilm}\epsilon_{ijk}$$
Using the double permutation tensor identity,
$$\epsilon_{lmi}\epsilon_{ijk} = \epsilon_{ilm}\epsilon_{ijk} = \delta_{lj}\delta_{mk}-\delta_{lk}\delta_{jm}$$
Substituting for the double permutation tensor identity,
$$[\nabla\times(\nabla\times\bar{A})]_{l} = [\delta_{lj}\delta_{mk}-\delta_{lk}\delta_{jm}]\frac{\partial}{\partial x_{m}}\left[\frac{\partial}{\partial x_{j}}(A_{k})\right]$$
$$[\nabla\times(\nabla\times\bar{A})]_{l} =  \delta_{lj}\delta_{mk}\frac{\partial}{\partial x_{m}}\left[\frac{\partial}{\partial x_{j}}(A_{k})\right]-\delta_{lk}\delta_{jm}\frac{\partial}{\partial x_{m}}\left[\frac{\partial}{\partial x_{j}}(A_{k})\right] $$
$$[\nabla\times(\nabla\times\bar{A})]_{l} = \frac{\partial}{\partial x_{k}}\left[\frac{\partial}{\partial x_{l}}(A_{k})\right]-\frac{\partial}{\partial x_{j}}\left[\frac{\partial}{\partial x_{j}}(A_{l})\right] $$
$$\left\{\nabla\times[\nabla\times(\nabla\times\bar{A})]\right\}_{p} = \epsilon_{pql}\frac{\partial}{\partial x_{q}}\left\{\frac{\partial}{\partial x_{k}}\left[\frac{\partial}{\partial x_{l}}(A_{k})\right]-\frac{\partial}{\partial x_{j}}\left[\frac{\partial}{\partial x_{j}}(A_{l})\right]\right\}$$
$$\left\{\nabla\times[\nabla\times(\nabla\times\bar{A})]\right\}_{p} = \epsilon_{pql}\frac{\partial}{\partial x_{q}}\left\{\frac{\partial}{\partial x_{k}}\left[\frac{\partial}{\partial x_{l}}(A_{k})\right]\right\}  -  \epsilon_{pql}\frac{\partial}{\partial x_{q}}\left\{\frac{\partial}{\partial x_{j}}\left[\frac{\partial}{\partial x_{j}}(A_{l})\right]\right\}$$
Since partial derivative operations are commutative with one another,
$$\epsilon_{pql}\frac{\partial}{\partial x_{q}}\left\{\frac{\partial}{\partial x_{k}}\left[\frac{\partial}{\partial x_{l}}(A_{k})\right]\right\} = \epsilon_{pql}\frac{\partial}{\partial x_{q}}\left\{\frac{\partial}{\partial x_{l}}\left[\frac{\partial}{\partial x_{k}}(A_{k})\right]\right\}$$
The permutation tensor is anti-symmetric in any $2$ of its indices, and $\displaystyle \frac{\partial}{\partial x_{q}}\left\{\frac{\partial}{\partial x_{l}}\left[\frac{\partial}{\partial x_{k}}(A_{k})\right]\right\}$ is symmetric in $q$ and $l$ due to the commutativity of the partial differential operator. Since this would mean a symmetric tensor multiplied by an anti-symmetric element-wise and added together,
$$0 = \epsilon_{pql}\frac{\partial}{\partial x_{q}}\left\{\frac{\partial}{\partial x_{k}}\left[\frac{\partial}{\partial x_{l}}(A_{k})\right]\right\}$$
Therefore,
$$\left\{\nabla\times[\nabla\times(\nabla\times\bar{A})]\right\}_{p} =  -  \epsilon_{pql}\frac{\partial}{\partial x_{q}}\left\{\frac{\partial}{\partial x_{j}}\left[\frac{\partial}{\partial x_{j}}(A_{l})\right]\right\}$$
renaming the free index $p \to i$,
$$\left\{\nabla\times[\nabla\times(\nabla\times\bar{A})]\right\}_{i} =  -  \epsilon_{iql}\frac{\partial}{\partial x_{q}}\left\{\frac{\partial}{\partial x_{j}}\left[\frac{\partial}{\partial x_{j}}(A_{l})\right]\right\}$$
Since $LHS_{i} = \left\{\nabla\times[\nabla\times(\nabla\times\bar{A})]\right\}_{i}$,
$$LHS_{i} =  -  \epsilon_{iql}\frac{\partial}{\partial x_{q}}\left\{\frac{\partial}{\partial x_{j}}\left[\frac{\partial}{\partial x_{j}}(A_{l})\right]\right\}$$
Reiterating definition of $RHS$,
$$RHS = -\nabla^{2}(\nabla\times\bar{A})$$
Converting $RHS$ into index notation,
$$RHS_{i} = -\frac{\partial}{\partial x_{j}}\left\{\frac{\partial}{\partial x_{j}}\left[\epsilon_{iql}\frac{\partial}{\partial x_{q}}(A_{l})\right]\right\}$$
Since the permutation tensor is a constant in $x_{j}$ and $x_{q}$ and that partial derivative operations are commutative with one another,
$$RHS_{i} = -\epsilon_{iql}\frac{\partial}{\partial x_{q}}\left\{\frac{\partial}{\partial x_{j}}\left[\frac{\partial}{\partial x_{j}}(A_{l})\right]\right\}$$
Since $LHS_{i} = RHS_{i}$, the identity is proven to be true.
%Seperator
%Seperator
%Seperator
%Seperator
%Seperator
%Seperator
\end{center}

\end{document}
