%Type of Document
\documentclass[a4paper, 12pt]{report}

%Load Pre-ambles
\usepackage{../Environment/Packages}
\usepackage{../Environment/Conventions}
\usepackage{../Environment/Hyahoos}

\begin{document}

\begin{center}
\chapter{Laplace Transform}
\begin{comment}
Laplace Transform Header
\end{comment}
%Seperator
%Seperator
%Seperator
%Seperator
%Seperator
\stdsection{Definition of Laplace Transform}
\begin{comment}
\end{comment}
Laplace Transform is defined as the following,
$$\lap[f(t)] = \int^{\infty}_{0}e^{-st}f(t)\,dt = F(s)$$
The Laplace Transform is a linear transform since the integral and product operations are both linear operations as well.
$$\lap[\a f(t)] = \a\lap[f(t)]\quad,\quad\lap[f(t) + g(t)] = \lap[f(t)] + \lap[g(t)]$$
$$\lap[\a f(t) + \be g(t)] = \a\lap[f(t)] + \be\lap[g(t)]$$
wherein $\a$, $\be$ represent constants and $f(t)$, $g(t)$ represent functions of t.
%Seperator
%Seperator
%Seperator
%Seperator
%Seperator
\section{Transforms of Derivatives}
\begin{comment}
\end{comment}
In General form,
$$\lap[\overset{n}{f(t)}] = s^n\lap[f(t)] - \sum^{n - 1}_{k = 0}\l[s^{n - 1 - k}\overset{k}{f(0)}\r]$$
wherein $\overset{n}{f(t)}$ represents the $n^{th}$ derivative of the function $f(t)$. Proof is shown below,
$$\lap[\overset{n}{f(t)}] = \int^{\infty}_{0}e^{-st}\overset{n}{f(t)}\,dt$$
$$\int uv' \,dt = uv - \int u'v \,dt$$
$$u = e^{-st}\quad,\quad u' = -se^{-st} \quad,\quad v' = \overset{n}{f(t)} \quad,\quad v = \overset{n-1}{f(t)}$$
$$\lap[\overset{n}{f(t)}] = \int^{\infty}_{0}e^{-st}\overset{n}{f(t)}dt = \l[e^{-st}\overset{n-1}{f(t)}\r]^{\infty}_{0} - \int^{\infty}_{0}-se^{-st}\overset{n-1}{f(t)} \,dt = -\overset{n-1}{f(0)} + s\int^{\infty}_{0}e^{-st}\overset{n-1}{f(t)} \,dt $$
Generalizing for the second integral term,
$$\int^{\infty}_{0}e^{-st}\overset{n-i}{f(t)} \,dt = \l[e^{-st}\overset{n-1-i}{f(t)}\r]^{\infty}_{0} + s\int^{\infty}_{0}e^{-st}\overset{n-1-i}{f(t)} \,dt = -\overset{n-1-i}{f(0)} + s\int^{\infty}_{0}e^{-st}\overset{n-1-i}{f(t)} \,dt$$
By applying substitution recursively,
$$\lap[\overset{n}{f(t)}] = -\sum^{k}_{i = 0}\l[s^i\overset{n-1-i}{f(0)}\r] + \prod^{k}_{i = 0}\l[s\r]\int^{\infty}_{0}e^{-st}\overset{n-1-k}{f(t)} \,dt = -\sum^{k}_{i = 0}\l[s^i\overset{n-1-i}{f(0)}\r] + s^{k + 1}\int^{\infty}_{0}e^{-st}\overset{n-1-k}{f(t)} \,dt$$
Substituting the value for $k = n-1$,
$$\lap[\overset{n}{f(t)}] = -\sum^{n-1}_{i = 0}\l[s^i\overset{n-1-i}{f(0)}\r] + s^{n}\int^{\infty}_{0}e^{-st}f(t) \,dt$$
A few things should be noted,
$$\int^{\infty}_{0}e^{-st}f(t) \,dt = \lap[f(t)]\quad,\quad \sum^{n-1}_{i = 0}\l[s^i\overset{n-1-i}{f(0)}\r] = \sum^{n-1}_{i = 0}\l[s^{n-1-i}\overset{i}{f(0)}\r]$$
By substitution of the counting variable $i$ with $k$,
$$\lap[\overset{n}{f(t)}] = -\sum^{n-1}_{i = 0}\l[s^i\overset{n-1-i}{f(0)}\r] + s^{n}\int^{\infty}_{0}e^{-st}f(t) \,dt = s^n \lap[f(t)] + \sum^{n-1}_{k = 0}\l[s^{n-1-k}\overset{k}{f(0)}\r]$$
%Seperator
%Seperator
%Seperator
%Seperator
%Seperator
\section{Transforms of Integrals}
\begin{comment}
\end{comment}
$$\lap\l[\int^{t}_{0}f(\tau)d\tau\r] = \f{1}{s}\lap[f(t)]$$
$$\lap\l[\int^{t}_{0}f(\tau)d\tau\r] = \int^{\infty}_{0}e^{-st}\int^{t}_{0}f(\tau)d\tau\,dt$$
$$\int uv' \,dt = uv - \int u'v \,dt$$
$$u = \int^{t}_{0}f(\tau)d\tau \quad,\quad u' = f(t) \quad,\quad v' = e^{-st} \quad,\quad v = -\f{1}{s}e^{-st}$$
$$\int^{\infty}_{0}e^{-st}\int^{t}_{0}f(\tau)d\tau\,dt = -\l[\f{1}{s}e^{-st}\int^{t}_{0}f(\tau)d\tau\r]^{\infty}_{0} + \int^{\infty}_{0}\f{1}{s}e^{-st}f(t)\,dt$$
It should be noted that since the function $f(t)$ is in exponential order,
$$\l[\f{1}{s}e^{-st}\int^{t}_{0}f(\tau)d\tau\r]^{\infty}_{0} = 0$$
Substituting the $uv$ term with zero,
$$\int^{\infty}_{0}e^{-st}\int^{t}_{0}f(\tau)d\tau\,dt = \int^{\infty}_{0}\f{1}{s}e^{-st}f(t)\,dt = \f{1}{s}\int^{\infty}_{0}e^{-st}f(t)\,dt = \f{1}{s}\lap[f(t)]$$
%Seperator
%Seperator
%Seperator
%Seperator
%Seperator
\section{Derivative of Transforms}
\begin{comment}
\end{comment}
$$\lap[t^nf(t)] = (-1)^n\overset{n}{F(s)} = (-1)^n\f{d^n}{ds^n}\l\{\lap[f(t)]\r\}$$
wherein $F(s)$ represents the laplace transform of the function $f(t)$. By the definition of Laplace Transforms discussed earlier,
$$\lap[t^nf(t)] = \int^{\infty}_{0}e^{-st}t^nf(t)\,dt\quad,\quad F(s) = \lap[f(t)] = \int^{\infty}_{0}e^{-st}f(t)\,dt$$
Differentiating the Laplace Transform of $f(t)$ with respect to $s$ iteratively $n$ times,
$$\overset{n}{F(s)} = \f{d^n}{ds^n}\{\lap[f(t)]\} = \f{d^n}{ds^n}\int^{\infty}_{0}e^{-st}f(t)\,dt = (-1)^n\int^{\infty}_{0}e^{-st}t^n f(t)\,dt$$
$$(-1)^n\overset{n}{F(s)} = \int^{\infty}_{0}e^{-st}t^n f(t)\,dt$$
By substituting the definition for the Laplace Transform of $\dst{t^nf(t)}$,
$$(-1)^n\overset{n}{F(s)} = \lap[t^nf(t)]$$
%Seperator
%Seperator
%Seperator
%Seperator
%Seperator
\section{Integration of Transforms}
\begin{comment}
\end{comment}
$$\lap\l[\f{f(t)}{t}\r] = \int^{\infty}_{s}F(\tau)\,d\tau$$
wherein $F(s)$ represents the laplace transform of the function $f(t)$.
$$\lap\l[\f{f(t)}{t}\r] = \int^{\infty}_{0}\f{e^{-st}f(t)}{t}\,dt\quad,\quad F(s) = \lap[f(t)] = \int^{\infty}_{0}e^{-st}f(t)\,dt$$
$$\int^{\infty}_{s}F(\tau)\,d\tau = \int^{\infty}_{s}\int^{\infty}_{0}e^{-\tau t}f(t)\,dt\,d\tau = \int^{\infty}_{0}\int^{\infty}_{s}e^{-\tau t}f(t)\,d\tau\,dt = \int^{\infty}_{0}\l[-\f{1}{t}e^{-\tau t}f(t)\r]^{\tau = \infty}_{\tau = s}\,dt$$
$$\int^{\infty}_{s}F(\tau)\,d\tau = -\int^{\infty}_{0}\f{f(t)}{t}\l[e^{-\tau t}\r]^{\tau = \infty}_{\tau = s}\,dt = -\int^{\infty}_{0}\f{f(t)}{t}\l[\lim_{\tau\to\infty}(e^{-\tau t}) - e^{-s t}\r]\,dt$$
Taking into account that, $\dst{\lim_{\tau\to\infty}(e^{-\tau t})} = 0,$
$$\int^{\infty}_{s}F(\tau)\,d\tau = -\int^{\infty}_{0}\f{f(t)}{t}\l[- e^{-s t}\r]\,dt = \int^{\infty}_{0}\f{e^{-s t}f(t)}{t}\,dt$$
By substituting the definition of the laplace transform of $\dst{\f{f(t)}{t}}$,
$$\int^{\infty}_{s}F(\tau)\,d\tau = \lap\l[\f{f(t)}{t}\r]$$
%Seperator
%Seperator
%Seperator
%Seperator
%Seperator
\section{Translation of Transforms}
\begin{comment}
\end{comment}
$$\lap\l[u(t-c)f(t)\r] = e^{-cs}\lap\l[f(t+c)\r]$$
By definition of Laplace Transform,
$$\lap\l[u(t-c)f(t)\r] = \int^{\infty}_{0}e^{-st}u(t-c)f(t)\,dt = \int^{\infty}_{c}e^{-st}f(t)\,dt + \int^{c}_{0}e^{-st}\times0\,dt$$
$$\lap\l[u(t-c)f(t)\r] = \int^{\infty}_{c}e^{-st}f(t)\,dt$$
Using the substitution $\dst{t = \tau + c}$. When $\dst{t = \infty\quad,\quad \tau = \infty \quad}$ and when $\dst{t = c\quad,\quad \tau = 0}$. Therefore,
$$\lap\l[u(t-c)f(t)\r] = \int^{t = \infty}_{t = c}e^{-s(\tau + c)}f(\tau + c)\,dt = \int^{\tau = \infty}_{\tau = 0}e^{-s(\tau + c)}f(\tau + c)\,d\tau$$
$$\lap\l[u(t-c)f(t)\r] = \int^{\tau = \infty}_{\tau = 0}e^{-s\tau -sc}f(\tau + c)\,d\tau = \int^{\tau = \infty}_{\tau = 0}e^{-cs}e^{-s\tau}f(\tau + c)\,d\tau$$
Since variables $s$ and $c$ are not changing with time, the term $e^{-cs}$ could be treated as some form of constant. Therefore,
$$\lap\l[u(t-c)f(t)\r] = e^{-cs}\int^{\infty}_{0}e^{-s\tau}f(\tau + c)\,d\tau$$
It should be noted that the change of variables allows,
$$\lap\l[f(t+c)\r] = \int^{\infty}_{0}e^{-st}f(t + c)\,dt = \int^{\infty}_{0}e^{-s\tau}f(\tau + c)\,d\tau$$
By substitution,
$$\lap\l[u(t-c)f(t)\r] = e^{-cs}\lap\l[f(t+c)\r]$$
%Seperator
%Seperator
%Seperator
%Seperator
%Seperator
\section{Transforms of Translated Functions}
\begin{comment}
\end{comment}
$$\lap[e^{ct}f(t)] = F(s-c)$$
Reiterating the definition of laplace transforms,
$$\lap[f(t)] = \int^{\infty}_{0}e^{-st}f(t)\,dt = F(s)$$
$$\lap[e^{ct}f(t)] = \int^{\infty}_{0}e^{ct}e^{-st}f(t)\,dt = \int^{\infty}_{0}e^{-st + ct}f(t)\,dt$$
$$\lap[e^{ct}f(t)] = \int^{\infty}_{0}e^{-(s-c)t}f(t)\,dt = F(s-c)$$
%Seperator
%Seperator
%Seperator
%Seperator
%Seperator
\section{Convolution}
\begin{comment}
\end{comment}
$$f*g(t) = \int^{t}_{0}f(\tau)g(t - \tau)\,d\tau$$
The convolution is a commutative transformation. Therefore,
$$f*g(t) = g*f(t) = \int^{t}_{0}f(\tau)g(t - \tau)\,d\tau = \int^{t}_{0}g(\tau)f(t - \tau)\,d\tau$$
One useful property of the convolution function,
$$\lap\l[f*g(t)\r] = \lap\l[f(t)\r]\times\lap\l[g(t)\r]$$
wherein
$$F(s) = \lap[f(t)] = \int^{\infty}_{0}e^{-st}f(t)\,dt\quad,\quad G(s) = \lap[g(t)] = \int^{\infty}_{0}e^{-st}g(t)\,dt$$
By a substitution of variables $t = u$ it could be re-written,
$$F(s) = \lap[f(u)] = \int^{\infty}_{0}e^{-su}f(u)\,du\quad,\quad G(s) = \lap[g(u)] = \int^{\infty}_{0}e^{-su}g(u)\,du$$
Examining the Lapalce Transform of $g(u)$, and making the substitution $u = t-\tau$
$$\lap[g(t-\tau)] = \int^{u = \infty}_{u = 0}e^{-s(t-\tau)}g(t-\tau)\,dt$$
When $\dst{u = \infty\quad,\quad t=\infty}\quad$ and when $\quad\dst{u = 0 \quad,\quad t = \tau}$. Therefore,
$$\lap[g(t-\tau)] = \int^{t = \infty}_{t = \tau}e^{-s(t-\tau)}g(t-\tau)\,dt$$
The $\dst{e^{\tau s}}$ term could be isolated because both variables $\tau$ and $s$ in this case are non-changing with $t$. The next form is identical to the laplace transform at the Translation of Transforms section,
$$\int^{\tau = \infty}_{\tau = 0}e^{-s(\tau + c)}f(\tau + c)\,d\tau = e^{-cs}\int^{\infty}_{0}e^{-s\tau}f(\tau + c)\,d\tau$$
By substituting $\tau$ in the Translation of Transforms section with $t$, substituting $c$ with $-\tau$, and substituting the arbitrary function $g$ with the arbitrary function $f$,
$$\int^{t = \infty}_{t = 0}e^{-s(t -\tau)}g(t -\tau)\,dt = e^{\tau s}\int^{\infty}_{0}e^{-st}g(t - \tau)\,dt$$
Therefore,
$$\lap[g(t-\tau)] = G(s) = e^{\tau s}\int^{\infty}_{0}e^{-st}g(t - \tau)\,dt$$
Proving the Convolution Property by first examining the product of the two Laplace Transforms,
$$F(s)\times G(s) = G(s)\int^{\infty}_{0}e^{-su}f(u)\,du = \int^{\infty}_{0}e^{-su}G(s)f(u)\,du$$
The above would be perfectly legal operations because $G(s)$ is a function in terms of $s$ and is unchanging with respect to variable $t$. Therefore, the function $G(s)$ could be treated as a constant that can be place inside and outside of the integral.
$$F(s)\times G(s) = \int^{\infty}_{0}e^{-s\tau}f(\tau) \times e^{\tau s}\int^{\infty}_{0}e^{-st}g(t - \tau)\,dt\,d\tau$$
$$F(s)\times G(s) = \int^{\infty}_{0}\int^{\infty}_{0}e^{-st}f(\tau)g(t - \tau)\,dt\,d\tau$$
By chaging the order of integration,
$$F(s)\times G(s) = \int^{\infty}_{0}e^{-st}\int^{\infty}_{0}f(\tau)g(t - \tau)\,d\tau\,dt = \lap\l[\int^{\infty}_{0}f(\tau)g(t - \tau)\,d\tau\r]$$
$$F(s)\times G(s) = \lap\l[f*g(t)\r]$$
\end{center}

\end{document}
