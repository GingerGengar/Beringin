%Type of Document
\documentclass[a4paper, 12pt]{report}

%Load Pre-ambles
\usepackage{../Environment/Packages}
\usepackage{../Environment/Conventions}
\usepackage{../Environment/Hyahoos}

\begin{document}

\begin{center}
%Seperator
%Seperator
%Seperator
%Seperator
%Seperator
%Seperator
\chapter{Laws of Motions}
\begin{comment}
Probably should rename rigid body to system of particles. There has to be a distinction between the two. Rigid bodies have the constraint that particles cannot move with respect to one another meanwhile for a system of particles, particles can still move relative to one another. Think of an aircraft as a rigid body meanwhile the earth and moon as a system of particles.


\end{comment}
%Seperator
%Seperator
%Seperator
%Seperator
%Seperator
\section{Newton's $2^{nd}$ Law}
\begin{comment}
\end{comment}
%Seperator
%Seperator
%Seperator
%Seperator
\subsection{Point Particle}
\begin{comment}
\end{comment}
%Seperator
%Seperator
%Seperator
\subsubsection{Inertial Reference Frame}
\begin{comment}
\end{comment}
Newton's $2^{nd}$ law for a point particle $p$ is shown below,
$$\sum^{n}_{i = 1}\left[{}^{e}\bar{F}^{p}_{i}\right] = m_{p}\overset{e}{\frac{\partial^{2}}{\partial t^{2}}}[\bar{r}^{o_{e}p}]$$
$LHS$ represents the summation of $n$ number of forces acting on particle $p$. The forces here are perceived through an inertial frame $e$, and hence, represent true forces and not 'fictitious forces'. $RHS$ represents the acceleration of particle $p$ relative to the origin of the inertial frame $e$ and perceived from the inertial frame $e$. $m_{p}$ represents the mass of the particle $p$ and $\bar{r}^{o_{e}p}$ represents the position vector of particle $p$ relative to the origin of the inertial frame $e$. Note that the left-right sub-superscripts follow the notation convention described in the terminology. $o_{e}p$ represents the position vector starting from the origin of frame $e$ and ending at the position of particle $p$. 
%Seperator
%Seperator
%Seperator
\subsubsection{Non-Inertial Reference Frame}
\begin{comment}
\end{comment}
%Seperator
%Seperator
%Seperator
%Seperator
\subsection{Rigid Bodies}
\begin{comment}
\end{comment}
%Seperator
%Seperator
%Seperator
\subsubsection{Inertial Reference Frame}
\begin{comment}
\end{comment}
Newton's $2^{nd}$ law for the $i^{th}$ point particle in a rigid body is shown below,
$$\sum^{n_{1}}_{j = 1}\left[{}^{e}\bar{F}^{p_{i}}_{j}\right] = m_{p_{i}}\overset{e}{\frac{\partial^{2}}{\partial t^{2}}}[\bar{r}^{o_{e}p_{i}}]$$
Making the substitutions,
\begin{equation}\overset{e}{\frac{\partial^{2}}{\partial t^{2}}}[\bar{r}^{o_{e}p_{i}}] = {}^{e}\bar{a}^{o_{e}p_{i}}\quad,\quad \sum^{n_{1}}_{j = 1}\left[{}^{e}\bar{F}^{p_{i}}_{j}\right] = {}^{e}\bar{F}^{p_{i}}_{ext} + \sum^{n_{1}}_{j = 1}\left[{}^{e}\bar{F}^{p_{i}}_{int,j}\right]\label{rigid-forces}\end{equation}
Newton's $2^{nd}$ law for the $i^{th}$ point particle in a rigid body becomes,
$${}^{e}\bar{F}^{p_{i}}_{ext} + \sum^{n_{1}}_{j = 1}\left[{}^{e}\bar{F}^{p_{i}}_{int,j}\right] = m_{i}{}^{e}\bar{a}^{o_{e}p_{i}}$$
wherein ${}^{e}\bar{F}^{p_{i}}_{ext}$ represents the resultant external force. The 'B' specifier represents that the external force is acting on the $i^{th}$ particle. The 'A' specifier represents that the external force referred to are perceived in an inertial reference frame. The external forces expressed here are real and does not contain fictitious elements.
\\~\\${}^{e}\bar{F}^{p_{i}}_{int,j}$ represents the forces acting on the $i^{th}$ particle due to intermolecular interaction between the particles in the rigid body. The 'A' specifier represents the intermolecular forces are perceived in an inertial reference frame. The 'B' specifier represents that the intermolecular forces are acting on the $i^{th}$ particle. The 'D' specifier holds the counter variable $j$ that is used to sum over all the intermolecular forces experienced by the $i^{th}$ particle. 
\\~\\$m_{i}$ represents the mass of the $i^{th}$ particle and $n_{2}$ represents the number of adjacent particles that has a force interaction with the $i^{th}$ particle.
\\~\\${}^{e}\bar{a}^{o_{e}p_{i}}$ represents acceleration of the $i^{th}$ particle. The 'A' specifier represents the reference frame that acceleration is perceived in. The 'B' specifier represents that this acceleration is for $i^{th}$ particle taken with reference to the origin of the inertial frame $e$.
\\~\\To find the resultant force on a rigid body, one must find the summation of all forces acting on each particle of the rigid body. Taking the summation of Newton's $2^{nd}$ law for point particles on the entire rigid body,
$$\sum^{n_{2}}_{i = 1}\left[{}^{e}\bar{F}^{p_{i}}_{ext}\right] + \sum^{n_{2}}_{i = 1}\sum^{n_{1}}_{j = 1}\left[{}^{e}\bar{F}^{p_{i}}_{int,j}\right] = \sum^{n_{2}}_{i = 1}\left[m_{i}{}^{e}\bar{a}^{o_{e}p_{i}}\right]$$
Here $n_{2}$ represents the total number of discrete particles that exist within the rigid body. There are $2$ statements that can be made regarding the intermolecular forces:
\begin{itemize}
    \item A particle cannot exert an intermolecular force on itself. Therefore the intermolecular force a particle exerts on itself is zero.
    \item By Newton's third law, the force acting on the $i^{th}$ particle by the $j^{th}$ particle is exactly equal and opposite to the force acting on the $j^{th}$ particle by the $i^{th}$ particle.
\end{itemize}
These $2$ statements lead to the summation of intermolecular forces within the rigid body evaluating to zero,
$$\sum^{n_{2}}_{i = 1}\sum^{n_{1}}_{j = 1}\left[{}^{e}\bar{F}^{p_{i}}_{int,j}\right] = 0$$
Simplifying Newton's $2^{nd}$ law,
\begin{equation}\sum^{n_{2}}_{i = 1}\left[{}^{e}\bar{F}^{p_{i}}_{ext}\right] = \sum^{n_{2}}_{i = 1}\left[m_{i}{}^{e}\bar{a}^{o_{e}p_{i}}\right]\label{mambah-1}\end{equation}

The center of mass definition for the rigid body is shown below,
$$\bar{r}^{o_{e}c} = \frac{\displaystyle \sum^{n_{2}}_{i = 1}\left[m_{i}\bar{r}^{o_{e}p_{i}}\right]}{\displaystyle \sum^{n_{2}}_{i = 1}\left[m_{i}\right]}$$
$\displaystyle \sum^{n_{2}}_{i = 1}\left[m_{i}\right]$ represents the total mass of the rigid body. This is a scalar, and is constant due to the conservation of mass. Manipulating the equation,
$$\left\{\sum^{n_{2}}_{i = 1}\left[m_{i}\right]\right\}\bar{r}^{o_{e}c} = \sum^{n_{2}}_{i = 1}\left[m_{i}\bar{r}^{o_{e}p_{i}}\right]$$
Taking the frame derivative with respect to the inertial $e$ frame,
$$\overset{e}{\frac{\partial}{\partial t}}\left\{\left[\displaystyle \sum^{n_{2}}_{i = 1}\left(m_{i}\right)\right]\bar{r}^{o_{e}c}\right\} = \overset{e}{\frac{\partial}{\partial t}}\left\{\sum^{n_{2}}_{i = 1}\left[m_{i}\bar{r}^{o_{e}p_{i}}\right]\right\}$$
Since $m_{i}$ is a constant due to conservation of mass,
$$\left[\displaystyle \sum^{n_{2}}_{i = 1}\left(m_{i}\right)\right]\overset{e}{\frac{\partial}{\partial t}}\left\{\bar{r}^{o_{e}c}\right\} = \sum^{n_{2}}_{i = 1}\left[\overset{e}{\frac{\partial}{\partial t}}\left\{m_{i}\bar{r}^{o_{e}p_{i}}\right\}\right]$$
$$\left[\displaystyle \sum^{n_{2}}_{i = 1}\left(m_{i}\right)\right]{}^{e}\bar{v}^{o_{e}c} = \sum^{n_{2}}_{i = 1}\left[m_{i}\overset{e}{\frac{\partial}{\partial t}}\left\{\bar{r}^{o_{e}p_{i}}\right\}\right]$$
Here we introduce velocity of the center of mass perceived in the $e$-frame and relative to the origin of frame $e$ as ${}^{e}\bar{v}^{o_{e}c}$. Simplifying further,
$$\left[\sum^{n_{2}}_{i = 1}\left(m_{i}\right)\right]{}^{e}\bar{v}^{o_{e}c} = \sum^{n_{2}}_{i = 1}\left[m_{i}{}^{e}\bar{v}^{o_{e}p_{i}}\right]$$
Taking the frame derivative with respect to the $e$-frame once more,
$$\overset{e}{\frac{\partial}{\partial t}}\left\{\left[\sum^{n_{2}}_{i = 1}\left(m_{i}\right)\right]{}^{e}\bar{v}^{o_{e}c}\right\} = \overset{e}{\frac{\partial}{\partial t}}\left\{\sum^{n_{2}}_{i = 1}\left[m_{i}{}^{e}\bar{v}^{o_{e}p_{i}}\right]\right\}$$
Performing similar operations as before,
$$\left[\sum^{n_{2}}_{i = 1}\left(m_{i}\right)\right]\overset{e}{\frac{\partial}{\partial t}}\left\{{}^{e}\bar{v}^{o_{e}c}\right\} = \sum^{n_{2}}_{i = 1}\left[m_{i}\overset{e}{\frac{\partial}{\partial t}}\left\{{}^{e}\bar{v}^{o_{e}p_{i}}\right\}\right]$$
\begin{equation}\left[\sum^{n_{2}}_{i = 1}\left(m_{i}\right)\right]{}^{e}\bar{a}^{o_{e}c} = \sum^{n_{2}}_{i = 1}\left[m_{i}{}^{e}\bar{a}^{o_{e}p_{i}}\right]\label{mambah-2}\end{equation}
Here we introduce the acceleration of the center of mass perceived in the $e$-frame and relative to the origin of frame $e$ as ${}^{e}\bar{a}^{o_{e}p_{i}}$. Substituting equation $\ref{mambah-2}$ to equation $\ref{mambah-1}$,
$$\sum^{n_{2}}_{i = 1}\left[{}^{e}\bar{F}^{p_{i}}_{ext}\right] = \sum^{n_{2}}_{i = 1}\left[m_{i}{}^{e}\bar{a}^{o_{e}p_{i}}\right] = \left[\sum^{n_{2}}_{i = 1}\left(m_{i}\right)\right]{}^{e}\bar{a}^{o_{e}c}$$
Let $\displaystyle M = \left[\sum^{n_{2}}_{i = 1}\left(m_{i}\right)\right]$. Here $M$ represents the total mass of the rigid body. Substituting for $M$,
$$\sum^{n_{2}}_{i = 1}\left[{}^{e}\bar{F}^{p_{i}}_{ext}\right] = M{}^{e}\bar{a}^{o_{e}c}$$
This is the final form for Newton's $2^{nd}$ law for a rigid body. On the $LHS$ there is the resultant external force acting on the rigid body. On the $RHS$ there is the total mass of the rigid body multiplied by acceleration of the center of mass. Due to Newton's $3^{rd}$ law, the internal forces of a rigid body can be safely ignored. The rigid body form of Newton's $2^{nd}$ law replaces forces on individual point particles with the summation of external forces and also replaces acceleration of single particles with the acceleration of the rigid body's center of mass. The rigid body form of Newton's $2^{nd}$ law replaces mass of the $i^{th}$ particle with the total mass of the rigid body $M$ when compared to Newton's $2^{nd}$ law for a single particle.
%Seperator
%Seperator
%Seperator
\subsubsection{Non-Inertial Reference Frame}
\begin{comment}
\end{comment}

%Seperator
%Seperator
%Seperator
%Seperator
%Seperator
\section{Euler's Law}
\begin{comment}
\end{comment}
%Seperator
%Seperator
%Seperator
%Seperator
\subsection{Point Particle}
\begin{comment}
\end{comment}
%Seperator
%Seperator
%Seperator
\subsubsection{Inertial Reference Frame}
\begin{comment}
\end{comment}
Newton's $2^{nd}$ law could be extended to rotation as well. Taking the cross product of Newton's $2^{nd}$ law with position vector $\bar{r}^{o_{e}p}$,
$$\bar{r}^{o_{e}p}\times\sum^{n}_{i = 1}\left[{}^{e}\bar{F}^{p}_{i}\right] = m_{p}\bar{r}^{o_{e}p}\times\overset{e}{\frac{\partial^{2}}{\partial t^{2}}}[\bar{r}^{o_{e}p}]$$
The summation $\displaystyle\sum$ operator and the vector cross-product operator are commutative with one another. Therefore,
\begin{equation}\sum^{n}_{i = 1}\left[\bar{r}^{o_{e}p}\times{}^{e}\bar{F}^{p}_{i}\right] = m_{p}\bar{r}^{o_{e}p}\times\overset{e}{\frac{\partial^{2}}{\partial t^{2}}}[\bar{r}^{o_{e}p}]\label{euler-sub-1}\end{equation}
Torque $\bar{\Gamma}$ is defined as shown below,
\begin{equation}{}^{e}\bar{\Gamma}^{o_{e}}_{p,i} = \bar{r}^{o_{e}p}\times{}^{e}\bar{F}^{p}_{i}\label{torque-definition}\end{equation}
The 'A' specifier on torque $\bar{\Gamma}$ represents that the torque is perceived in an inertial reference frame. This means ficitious forces should be exluded. The 'B'specifier on torque $\bar{\Gamma}$ represents that the moment is taken with respect to the origin of the inertial reference frame $e$. The 'D' specifier is to represent that the $i^{th}$ torque produced from the $i^{th}$ force is acting on particle $p$. Substituting the definition for torque to equation $\ref{euler-sub-1}$,
\begin{equation}\sum^{n}_{i = 1}\left[{}^{e}\bar{\Gamma}^{o_{e}}_{p,i}\right] = m_{p}\bar{r}^{o_{e}p}\times\overset{e}{\frac{\partial^{2}}{\partial t^{2}}}[\bar{r}^{o_{e}p}]\label{euler-sub-2}\end{equation}

Here we make the claim,
$$\overset{e}{\frac{\partial}{\partial t}}\left[\bar{r}^{o_{e}p}\times m_{p}\overset{e}{\frac{\partial}{\partial t}}\left(\bar{r}^{o_{e}p}\right)\right] = m_{p}\bar{r}^{o_{e}p}\times\overset{e}{\frac{\partial^{2}}{\partial t^{2}}}[\bar{r}^{o_{e}p}]$$
Let $LHS$ and $RHS$ be defined below,
$$LHS = \overset{e}{\frac{\partial}{\partial t}}\left[\bar{r}^{o_{e}p}\times m_{p}\overset{e}{\frac{\partial}{\partial t}}\left(\bar{r}^{o_{e}p}\right)\right] \quad,\quad RHS = m_{p}\bar{r}^{o_{e}p}\times\overset{e}{\frac{\partial^{2}}{\partial t^{2}}}[\bar{r}^{o_{e}p}]$$
Epxanding $LHS$ based on product rule,
$$LHS = \bar{r}^{o_{e}p}\times\overset{e}{\frac{\partial}{\partial t}}\left[ m_{p}\overset{e}{\frac{\partial}{\partial t}}\left(\bar{r}^{o_{e}p}\right)\right] + \overset{e}{\frac{\partial}{\partial t}}\left[\bar{r}^{o_{e}p}\right]\times m_{p}\overset{e}{\frac{\partial}{\partial t}}\left(\bar{r}^{o_{e}p}\right)$$
The vector crosss-product with itself is zero. Therefore the term $\displaystyle \overset{e}{\frac{\partial}{\partial t}}\left[\bar{r}^{o_{e}p}\right]\times m_{p}\overset{e}{\frac{\partial}{\partial t}}\left(\bar{r}^{o_{e}p}\right) = 0$. Ignorinng the second term in the equation above,
$$LHS = \bar{r}^{o_{e}p}\times\overset{e}{\frac{\partial}{\partial t}}\left[ m_{p}\overset{e}{\frac{\partial}{\partial t}}\left(\bar{r}^{o_{e}p}\right)\right]$$
Scalar multiplication and frame derivative operations are commutative as long as the scalar remains constant with time. Since mass of particle $m_{p}$ is unchanging in time due to conservation of mass,
$$LHS =  m_{p}\bar{r}^{o_{e}p}\times\overset{e}{\frac{\partial}{\partial t}}\left[\overset{e}{\frac{\partial}{\partial t}}\left(\bar{r}^{o_{e}p}\right)\right]$$
Simplifying to $2^{nd}$ order derivative,
$$LHS =  m_{p}\bar{r}^{o_{e}p}\times \overset{e}{\frac{\partial^{2}}{\partial t^{2}}}\left(\bar{r}^{o_{e}p}\right) $$
Since $LHS = RHS$, the claim is proven to be true. Since the claim is true, the claim can be substituted into equation $\ref{euler-sub-2}$,

\begin{equation}\sum^{n}_{i = 1}\left[{}^{e}\bar{\Gamma}^{o_{e}}_{p,i}\right] = \overset{e}{\frac{\partial}{\partial t}}\left[\bar{r}^{o_{e}p}\times m_{p}\overset{e}{\frac{\partial}{\partial t}}\left(\bar{r}^{o_{e}p}\right)\right]\label{euler-sub-3}\end{equation}

Let angular momentum $\bar{L}$ be defined below,
$${}^{e}\bar{L}^{o_{e}}_{p} = \bar{r}^{o_{e}p}\times m_{p}\overset{e}{\frac{\partial}{\partial t}}\left(\bar{r}^{o_{e}p}\right)$$
Angular momentum is dependent on $\displaystyle \overset{e}{\frac{\partial}{\partial t}}\left(\bar{r}^{o_{e}p}\right)$. This term is dependent on the frame derivative operation and which frame the time derivative of $\displaystyle \bar{r}^{o_{e}p}$ is taken with respect to. The frame used for the frame derivative operation when constructing the angular momentum definition is represented in the 'A' specifier. The angular momentum is also dependent on the position vector of particle $p$, $\displaystyle \bar{r}^{o_{e}p}$. $2$ facts can be deduced from the position vector in this form: the point of reference the particle $p$ is taken with respect to, in this case origin of inertial frame $o_{e}$, and the object the position vector is pointing at, particle $p$. Likewise, the angular momentum vector notation needs to be able to express the point of reference and the particle being referenced. The point of reference is represented in the 'B' specifier meanwhile the object being referenced, particle $p$ is represented in the 'D' specifier. Substituting the definition for angular momentum, into equation $\ref{euler-sub-3}$,
$$\sum^{n}_{i = 1}\left[{}^{e}\bar{\Gamma}^{o_{e}}_{p,i}\right] = \overset{e}{\frac{\partial}{\partial t}}\left[{}^{e}\bar{L}^{o_{e}}_{p}\right]$$
%Seperator
%Seperator
%Seperator
\subsubsection{Non-Inertial Reference Frame}
\begin{comment}
\end{comment}
%Seperator
%Seperator
%Seperator
%Seperator
\subsection{Rigid Bodies}
\begin{comment}
Kind of incomplete, more information about the colinear argument need to be made
\end{comment}
%Seperator
%Seperator
%Seperator
\subsubsection{Inertial Reference Frame}
\begin{comment}
\end{comment}
EUler's law for a point particle $p$,
$$\sum^{n}_{i = 1}\left[{}^{e}\bar{\Gamma}^{o_{e}}_{p,i}\right] = \overset{e}{\frac{\partial}{\partial t}}\left[{}^{e}\bar{L}^{o_{e}}_{p}\right]$$
Since $i$ is just a counting variable, the expression above would still be true if $i$ is renamed to another variable. Let $i \to j$. The expression above assumes that there are $n$ numeber of torque acting on particle $p$. Let $n \to n_{1}$. The expression above is true for the point particle $p$. In a system of rigid bodies however, there are alot of particles. So, let thte equation above hold true for a particular particle $p_{i}$. Substituting these changes,
\begin{equation}\sum^{n_{1}}_{j = 1}\left[{}^{e}\bar{\Gamma}^{o_{e}}_{p_{i},j}\right] = \overset{e}{\frac{\partial}{\partial t}}\left[{}^{e}\bar{L}^{o_{e}}_{p_{i}}\right]\label{moments-rigid-1}\end{equation}
Reiterating the definition of torque defined at equation $\ref{torque-definition}$,
$${}^{e}\bar{\Gamma}^{o_{e}}_{p,i} = \bar{r}^{o_{e}p}\times{}^{e}\bar{F}^{p}_{i}$$
Performing the same substitutions, $i \to j$, $n \to n_{1}$, and $p \to p_{1}$,
$${}^{e}\bar{\Gamma}^{o_{e}}_{p_{i},j} = \bar{r}^{o_{e}p_{i}}\times{}^{e}\bar{F}^{p_{i}}_{j}$$
Taking the summation for all the torque acting on the $i^{th}$ particle,
$$\sum^{n_{1}}_{j = 1}\left[{}^{e}\bar{\Gamma}^{o_{e}}_{p_{i},j}\right] = \sum^{n_{1}}_{j = 1}\left[\bar{r}^{o_{e}p_{i}}\times{}^{e}\bar{F}^{p_{i}}_{j}\right]$$
Since $\bar{r}^{o_{e}p_{i}}$ does not contain a $j$ index, it can be considered a scalar constant in the summation operation. Therefore,
$$\sum^{n_{1}}_{j = 1}\left[{}^{e}\bar{\Gamma}^{o_{e}}_{p_{i},j}\right] = \bar{r}^{o_{e}p_{i}}\times\sum^{n_{1}}_{j = 1}\left[{}^{e}\bar{F}^{p_{i}}_{j}\right]$$
Reiterating the expression for the forces experienced by the $i^{th}$ particle described in equation $\ref{rigid-forces}$,
$$\sum^{n_{1}}_{j = 1}\left[{}^{e}\bar{F}^{p_{i}}_{j}\right] = {}^{e}\bar{F}^{p_{i}}_{ext} + \sum^{n_{1}}_{j = 1}\left[{}^{e}\bar{F}^{p_{i}}_{int,j}\right]\label{rigid-forces}$$
Substituting for the forces experienced by the $i^{th}$ particle described in equation $\ref{rigid-forces}$,
$$\sum^{n_{1}}_{j = 1}\left[{}^{e}\bar{\Gamma}^{o_{e}}_{p_{i},j}\right] = \bar{r}^{o_{e}p_{i}}\times\left\{{}^{e}\bar{F}^{p_{i}}_{ext} + \sum^{n_{1}}_{j = 1}\left[{}^{e}\bar{F}^{p_{i}}_{int,j}\right]\right\}$$
$$\sum^{n_{1}}_{j = 1}\left[{}^{e}\bar{\Gamma}^{o_{e}}_{p_{i},j}\right] =  \bar{r}^{o_{e}p_{i}}\times{}^{e}\bar{F}^{p_{i}}_{ext} + \bar{r}^{o_{e}p_{i}}\times\sum^{n_{1}}_{j = 1}\left[{}^{e}\bar{F}^{p_{i}}_{int,j}\right]$$
The term $\bar{r}^{o_{e}p_{i}}$ is not dependent on the counting variable $j$. Therefore, it acts as a scalar constant in the summation of $j$. Therefore,
$$\sum^{n_{1}}_{j = 1}\left[{}^{e}\bar{\Gamma}^{o_{e}}_{p_{i},j}\right] = \bar{r}^{o_{e}p_{i}}\times{}^{e}\bar{F}^{p_{i}}_{ext} + \sum^{n_{1}}_{j = 1}\left[\bar{r}^{o_{e}p_{i}}\times{}^{e}\bar{F}^{p_{i}}_{int,j}\right]$$
Susbtituting $\displaystyle \sum^{n_{1}}_{j = 1}\left[{}^{e}\bar{\Gamma}^{o_{e}}_{p_{i},j}\right]$ into equation $\ref{moments-rigid-1}$,
$$\overset{e}{\frac{\partial}{\partial t}}\left[{}^{e}\bar{L}^{o_{e}}_{p_{i}}\right] = \sum^{n_{1}}_{j = 1}\left[{}^{e}\bar{\Gamma}^{o_{e}}_{p_{i},j}\right] = \bar{r}^{o_{e}p_{i}}\times{}^{e}\bar{F}^{p_{i}}_{ext} + \sum^{n_{1}}_{j = 1}\left[\bar{r}^{o_{e}p_{i}}\times{}^{e}\bar{F}^{p_{i}}_{int,j}\right]$$
This expression is true for the $i^{th}$ particle. Since the entire rigid body is the summation of all particles, summing the equation above for all particles would yield an equation that is true for the rigid body. Summing the equation above for all particles $n_{2}$,
$$\sum^{n_{2}}_{i = 1}\left\{ \overset{e}{\frac{\partial}{\partial t}}\left[{}^{e}\bar{L}^{o_{e}}_{p_{i}}\right]\right\} = \sum^{n_{2}}_{i = 1}\left\{ \bar{r}^{o_{e}p_{i}}\times{}^{e}\bar{F}^{p_{i}}_{ext} + \sum^{n_{1}}_{j = 1}\left[\bar{r}^{o_{e}p_{i}}\times{}^{e}\bar{F}^{p_{i}}_{int,j}\right]\right\}$$
$$\sum^{n_{2}}_{i = 1}\left\{ \overset{e}{\frac{\partial}{\partial t}}\left[{}^{e}\bar{L}^{o_{e}}_{p_{i}}\right]\right\} = \sum^{n_{2}}_{i = 1}\left\{ \bar{r}^{o_{e}p_{i}}\times{}^{e}\bar{F}^{p_{i}}_{ext}\right\} + \sum^{n_{2}}_{i = 1}  \sum^{n_{1}}_{j = 1}\left[\bar{r}^{o_{e}p_{i}}\times{}^{e}\bar{F}^{p_{i}}_{int,j}\right] $$

%Enter more information about the colinear argument
Due to a colinear argument, $\displaystyle \sum^{n_{2}}_{i = 1}  \sum^{n_{1}}_{j = 1}\left[\bar{r}^{o_{e}p_{i}}\times{}^{e}\bar{F}^{p_{i}}_{int,j}\right] = 0$. Substituting,
$$\sum^{n_{2}}_{i = 1}\left\{ \overset{e}{\frac{\partial}{\partial t}}\left[{}^{e}\bar{L}^{o_{e}}_{p_{i}}\right]\right\} = \sum^{n_{2}}_{i = 1}\left\{ \bar{r}^{o_{e}p_{i}}\times{}^{e}\bar{F}^{p_{i}}_{ext}\right\}$$
The summation and partial derivative operations as commutative. Therefore,
$$\overset{e}{\frac{\partial}{\partial t}}\left[\sum^{n_{2}}_{i = 1}\left\{{}^{e}\bar{L}^{o_{e}}_{p_{i}}\right\}\right] = \sum^{n_{2}}_{i = 1}\left\{ \bar{r}^{o_{e}p_{i}}\times{}^{e}\bar{F}^{p_{i}}_{ext}\right\}$$
Let $\displaystyle {}^{e}\bar{L}^{o_{e}}_{rb} = \sum^{n_{2}}_{i = 1}\left\{{}^{e}\bar{L}^{o_{e}}_{p_{i}}\right\}$. Wherein $\displaystyle {}^{e}\bar{L}^{o_{e}}_{rb}$ represents the total angular momentum of the rigid body $rb$. Substituting,
$$\overset{e}{\frac{\partial}{\partial t}}\left[{}^{e}\bar{L}^{o_{e}}_{rb}\right] = \sum^{n_{2}}_{i = 1}\left\{ \bar{r}^{o_{e}p_{i}}\times{}^{e}\bar{F}^{p_{i}}_{ext}\right\}$$
Let $\displaystyle {}^{e}\bar{\Gamma}^{o_{e}}_{rb} = \sum^{n_{2}}_{i = 1}\left\{ \bar{r}^{o_{e}p_{i}}\times{}^{e}\bar{F}^{p_{i}}_{ext}\right\}\label{euler-law-primitive}$, wherein ${}^{e}\bar{\Gamma}^{o_{e}}_{rb}$ represents the torque that is acting on the rigid body as a whole. From the colinear argument made earlier, the internal forces do not contribute to the torque of the rigid body as a whole. Note that ${}^{e}\bar{F}^{p_{i}}_{ext}$ represents the resultant external forces acting on a single particle $p_{i}$ in the rigid body $rb$. Substituting,
$$\overset{e}{\frac{\partial}{\partial t}}\left[{}^{e}\bar{L}^{o_{e}}_{rb}\right] = {}^{e}\bar{\Gamma}^{o_{e}}_{rb}$$

%Seperator
%Seperator
%Seperator
\subsubsection{Non-Inertial Reference Frame}
\begin{comment}
\end{comment}



%Seperator
%Seperator
%Seperator
%Seperator
%Seperator
%Seperator

\end{center}

\end{document}
