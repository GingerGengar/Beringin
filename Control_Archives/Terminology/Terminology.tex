%Type of Document
\documentclass[class=report, 12pt, crop=false]{standalone}

%Load Pre-ambles
\usepackage{../Environment/Packages}
\usepackage{../Environment/Conventions}
\usepackage{../Environment/Hyahoos}

\begin{document}

\begin{center}
%Seperator
%Seperator
%Seperator
%Seperator
%Seperator
%Seperator
\chapter{Terminology}
\begin{comment}
\end{comment}
%Seperator
%Seperator
%Seperator
%Seperator
%Seperator
\section{Rationale}
\begin{comment}
\end{comment}
When studying the dynamics of motion, variables contain alot of context which defines them uniquely apart. For example, velocity perceived in an inertial frame, would be different than velocity perceived in a non-inertial frame, even when describing the same object. In some of the derivations it might be necessary to reference velocity of object $a$ and velocity of object $b$ in the same equation. If both velocities of object $a$ and object $b$ cannot be distinguished from one another, the equation is ambiguously useless. Therefore, the notation presented in the perceding derivations and workings must be able to express many of the different contexts to tell each variable uniquely apart from one another.
%Seperator
%Seperator
%Seperator
%Seperator
%Seperator
\section{Specifiers}
\begin{comment}
\end{comment}
Variables would typically be represented like shown below,
$${}^{A}_{C}\bar{v}^{B}_{D}$$
Here the variable being represented $\bar{v}$ is a vector. However, this notation could potentially be extended to scalars $v$ and tensors of higher orders $\bar{\bar{v}}$ as well. The 'A' specifier represents the top left superscript. The 'B' specifier represents top right superscript. 'C' specifier represent bottom left subscript, and 'D' specifier represent bottom right subscript. This notation form is quite strange, but has alot of expressive power, with $4$ possible specifiers. Note that specifiers are used flexibly in this document. $1$, $2$, all, or none of the specifiers may be used in variable expression. If a specifier is not used, as in the common case of the 'C' specifier, it means that no attributes are linked to that specifier, or it does not matter. 
%Seperator
%Seperator
%Seperator
%Seperator
%Seperator
\section{Variable Context}
\begin{comment}
\end{comment}
%Seperator
%Seperator
%Seperator
%Seperator
\subsection{Laws of Motion}
\begin{comment}
\end{comment}
%Seperator
%Seperator
%Seperator
\subsubsection{Force vectors}
\begin{comment}
\end{comment}
$${}^{A}_{C}\bar{F}^{B}_{D}$$
\begin{itemize}
    \item 'A': Frame the force is asociated with. If the specified frame is a non-inertial frame, it implies the addition of fictitious forces. If the specified frame is inertial, fictitious forces excluded.
    \item 'B': Object experiencing force. This object could be anything from a particle, to a vehicle, to some structural part. This is to settle some of the ambiguity related to force diagrams.
    \item 'C': Basis vectors. This specifier is to represent which frame or set of vectors is used to represent the force vector.
    \item 'D': Naming, indexing. Additional names, or counting indices such as $i$ or $j$ could be included here.
\end{itemize}
Examples:
%Seperator
%Seperator
%Seperator
\subsubsection{Torque vectors}
\begin{comment}
\end{comment}
$${}^{A}_{C}\bar{\Gamma}^{B}_{D}$$
\begin{itemize}
    \item 'A': Frame the torque is asociated with. If specified frame is inertial, no fictitious torque is included. If the specified frame is non-inertial, fictitious torque is included. Fictitious torque is generated as a consequence of fictitious forces.
    \item 'B': Point of reference. A point in $3$-dimensional space that if a force acts on that point, no torque is generated. The further away a particular force is from this point of reference, the larger the magnitude of the torque vector generated.
    \item 'C': basis vector. The frame or set of vectors used to represent the torque vector.
    \item 'D': Naming, indexing. Additional names, or counting indices such as $i$ or $j$ could be included here.
\end{itemize}
Examples:
%Seperator
%Seperator
%Seperator
%Seperator
\subsection{Kinematics}
\begin{comment}
\end{comment}
%Seperator
%Seperator
%Seperator
\subsubsection{Position vectors}
\begin{comment}
\end{comment}
$${}^{A}_{C}\bar{r}^{B}_{D}$$
\begin{itemize}
    \item 'A': none. This specifier is reserved for the frame used in the frame derivative operation.
    \item 'B': For position vectors, usually this specifier starts with the beginning point of the position vector and ends with the end point of the position vector.
    \item 'C': basis vector. The frame or set of vectors used to represent the position vector.
    \item 'D': Naming, indexing. Additional names, or counting indices such as $i$ or $j$ could be included here.
\end{itemize}
Examples:
%Seperator
%Seperator
%Seperator
\subsubsection{Velocity vectors}
\begin{comment}
\end{comment}
$${}^{A}_{C}\bar{v}^{B}_{D}$$
\begin{itemize}
    \item 'A': Frame that is used when taking the frame derivative of position vector.
    \item 'B': This specifier functions very similarly to its counterpart in the position vectors. This specifier is to represent what position vector this velocity vector is derived from.
    \item 'C': basis vector. The frame or set of vectors used to represent the velocity vector.
    \item 'D': Naming, indexing. Additional names, or counting indices such as $i$ or $j$ could be included here.
\end{itemize}
Examples:
%Seperator
%Seperator
%Seperator
\subsubsection{Acceleration Vectors}
\begin{comment}
\end{comment}
$${}^{A}_{C}\bar{a}^{B}_{D}$$
\begin{itemize}
    \item 'A': Frame that is used when taking the second order frame derivative of position vector.
    \item 'B': This specifier functions very similarly to its counterpart in the position vectors. This specifier is to represent what position vector this acceleration vector is derived from.
    \item 'C': basis vector. The frame or set of vectors used to represent the velocity vector.
    \item 'D': Naming, indexing. Additional names, or counting indices such as $i$ or $j$ could be included here.
\end{itemize}
Examples:
Acceleration vector and momentum vector is similar to the velocity vector notation-wise.
%Seperator
%Seperator
%Seperator
\subsubsection{Angular velocity vectors}
\begin{comment}
\end{comment}
$${}^{A}_{C}\bar{\omega}^{B}_{D}$$
\begin{itemize}
    \item 'A': First frame of reference
    \item 'B': Second frame of reference
    \item 'C': basis vector. The frame or set of vectors used to represent the angular velocity vector.
    \item 'D': Naming, indexing. Additional names, or counting indices such as $i$ or $j$ could be included here.
\end{itemize}
Examples:
%Seperator
%Seperator
%Seperator
%Seperator
\subsection{Conservation Laws}
\begin{comment}
\end{comment}
%Seperator
%Seperator
%Seperator
\subsubsection{Momentum vectors}
\begin{comment}
unfinished
\end{comment}
$${}^{A}_{C}\bar{p}^{B}_{D}$$
\begin{itemize}
    \item 'A': The frame the linear momentum is asociated with. Linear momentum is dependent on velocity which is dependent on the frame it is perceived from. This specifier represents which frame the velocity needed for linear momentum is viewed from. 
    \item 'B': Point of reference. This specifier is very similar to the position vector case. Linear momentum is dependent on velocity vector which is dependent on position vector of object of interest. This specifier represents the beginning point and end point of the position vector.
    \item 'C': none. The set of basis vectors used to represent the momentum vector must be identical to the basis vectors the momentum is taken with respect to.
    \item 'D': Naming, indexing. Additional names, or counting indices such as $i$ or $j$ could be included here.
\end{itemize}
Examples:
%Seperator
%Seperator
%Seperator
\subsubsection{Angular Momentum vectors}
\begin{comment}
\end{comment}
$${}^{A}_{C}\bar{L}^{B}_{D}$$
\begin{itemize}
    \item 'A': Frame the angular momentum vector is asociated with. Angular momentum is dependent on linear momentum, which is dependent on velocity vector, which varies from frame to frame. Hence this specifier is reserved for the frame velocity hence momentum is perceived from.
    \item 'B': Point of reference. Functions very similarly to the torque case.
    \item 'C': basis vector. The frame or set of vectors used to represent the angular momentum vector.
    \item 'D': Naming, indexing. Additional names, or counting indices such as $i$ or $j$ could be included here.
\end{itemize}
Examples:
%Seperator
%Seperator
%Seperator
%Seperator
\subsection{}
\begin{comment}
\end{comment}
%Seperator
%Seperator
%Seperator
%Seperator
%Seperator
%Seperator

\end{center}

\end{document}
