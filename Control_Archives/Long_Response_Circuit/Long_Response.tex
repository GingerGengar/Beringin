%Type of Document
\documentclass[a4paper, 12pt]{report}

%Load Pre-ambles
\usepackage{../Environment/Packages}
\usepackage{../Environment/Conventions}
\usepackage{../Environment/Hyahoos}

\begin{document}

\begin{center}
\chapter{Long-Term Behaviour of Circuits}
\begin{comment}
\end{comment}
%Seperator
%Seperator
%Seperator
%Seperator
%Seperator
\section{Final Value Theorem}
\begin{comment}
\end{comment}
The final value theorem for an arbitrary function $f(t)$ is written as,
$$\lim_{t\to\infty}[f(t)] = \lim_{s\to0}[sF(s)]$$
wherein $F(s)$ represents the laplace transform of $f(t)$. The proof of this particular form of the final value theorem is shown below,
$$\lim_{s\to0}\l\{\lap\l[f'(t)\r]\r\} = \lim_{s\to0}\l\{\int^{\infty}_{0}e^{-st}f'(t)\,dt\r\} = \int^{\infty}_{0}f'(t)\,dt $$
By fundamental theorem of calculus, and change of variable,
$$\lim_{s\to0}\l\{\lap\l[f'(t)\r]\r\} = \lim_{r\to\infty}\l\{f(r) - f(0)\r\} = \lim_{t\to\infty}\l\{f(t) - f(0)\r\}$$
Taking the approach of the Laplace Transform of derivatives,
$$\lap\l[f'(t)\r] = s\lap{f(t)} - f(0) = sF(s) - f(0)$$
Therefore, taking the limits as before,
$$\lim_{s\to0}\l\{\lap\l[f'(t)\r]\r\} = \lim_{s\to0}\l\{sF(s)\r\}\} - f(0)$$
By equating the $\dst{\lim_{s\to0}\l\{\lap\l[f'(t)\r]\r\}}$ to each other,
$$\lim_{t\to\infty}\l\{f(t)\r\} - f(0) = \lim_{s\to0}\l\{sF(s)\r\}\} - f(0)$$
This completes the proof,
$$\lim_{t\to\infty}\l\{f(t)\r\} = \lim_{s\to0}\l\{sF(s)\r\}\}$$
%Seperator
%Seperator
%Seperator
%Seperator
%Seperator
\section{Impulse Response}
\begin{comment}
\end{comment}
The transfer function $G(s)$ is defined as 
$$G(s) = \f{Y(s)}{U(s)}$$
wherein the $Y(s)$ is the output and $U(s)$ is the input, both in the laplace domain. In the laplace domain, the delta-dirac impulse function with arbitrary magnitude $\mu_0$,
$$\lap[\mu_0\,\delta(t-c)] = \mu_0e^{-cs}$$
If the system described by the transfer function $G(s)$ has an input of the dirac delta function of arbitrary magnitude,
$$G(s) = \mu_0e^{-cs}\,Y(s)$$
Manipulating for $Y(s)$ to be the subject of the equation,
$$Y(s) = \l(\f{e^{cs}}{\mu_0}\r)G(s)$$
To find the long term behaviour of the output of the system $\dst{\lim_{t\to\infty}\l[y(t)\r]}$, we should consult the final value theorem. The final value theorem for an arbitrary function $f(t)$,
$$\lim_{t\to\infty}[f(t)] = \lim_{s\to0}[sF(s)]$$
By performing the substitution $f(t) = y(t)$ and $F(s) = Y(s)$, wherein $y(t)$ represents the output function in the time domain and $Y(s)$ represents the output function in the laplace domain accordingly,
$$\lim_{t\to\infty}[y(t)] = \lim_{s\to0}[sY(s)]$$
If the input function is the dirac-delta function as shown previously, then $\dst{Y(s) = \l(\f{e^{cs}}{\mu_0}\r)G(s)}$. Therefore,
$$\lim_{t\to\infty}[y(t)] = \lim_{s\to0}\l[s\l(\f{e^{cs}}{\mu_0}\r)G(s)\r]$$
A famous identity for limits of two arbitrary functions $q(t)$ and $p(t)$,
$$\lim_{t\to t_0}\l[p(t)\times q(t)\r] = \lim_{t\to t_0}\l[p(t)\r]\times \lim_{t\to t_0}\l[q(t)\r]$$
By the identity above,
$$\lim_{t\to\infty}[y(t)] = \lim_{s\to0}\l[sG(s)\r] \times \lim_{s\to0}\l[\f{e^{cs}}{\mu_0}\r] = \f{1}{\mu_0}\lim_{s\to0}\l[sG(s)\r]$$
%Seperator
%Seperator
%Seperator
%Seperator
%Seperator
\section{Step Response}
\begin{comment}
\end{comment}
Let $G(s)$ represent the transfer function of some system in the laplace domain, $Y(s)$ represent the output of some system in the laplace domain, and $U(s)$ represent the input in the laplace domain.
$$Y(s) = G(s)U(s)$$
Consider the heaviside function $u(t-c)$ wherein $c$ is some arbitrary time the heaviside function "turns on". The laplace transform of the arbitray heaviside function,
$$\lap[u(t-c)] = \f{e^-cs}{s}$$
If the input $u(t) = \mu_0$ wherein $\mu_0$ is a constant for all $t$, then
$$\mu_0G(0) = \lim_{t\to\infty}[y(t)]$$
%Seperator
%Seperator
%Seperator
%Seperator
%Seperator
\section{Pure Sinusoid Response}
\begin{comment}
\end{comment}
The convolution of two arbitrary functions $f(t)$ and $g(t)$ are defined as
$$f*g(t) = g*f(t) = \int^{t}_{0}f(\tau)g(t - \tau)\,d\tau = \int^{t}_{0}g(\tau)f(t - \tau)\,d\tau$$
Suppose a system has a transfer function $G(s)$ and the output and input in the laplace domain respectively is $Y(s)$ and $U(s)$.
$$G(s) = \f{Y(s)}{U(s)}$$
Therefore,
$$y(t) = \lap^{-1}\l[G(s)U(s)\r] = \lap^{-1}\l[ G(s)\r]*\lap^{-1}\l[ U(s)\r] = g(t)*u(t) = \int^{t}_{0}g(\tau)u(t - \tau)\,d\tau$$
wherein $\lap^{-1}G(s) = g(t)$, and $\lap^{-1}U(s) = u(t)$. Here, the function $u(t)$ does not represent the heaviside unit function. Assuming that the input function is reasonably well-behaved and without loss of generality, 
$$u(t) = \sum^{\infty}_{k = -\infty}\l[a_ke^{-ik\w_0t}\r]\quad,\quad a_k = \f{1}{\tau}\int^{\tau}_{0}e^{ik\w_0t}f(t)dt\quad,\quad \w_0 = 2\pi/\tau$$
Substituting the Fourier representation of the input function,
$$y(t) = \int^{t}_{0}g(\tau)\sum^{\infty}_{k = -\infty}\l[a_ke^{-ik\w_0(t-\tau)}\r]\,d\tau = \int^{t}_{0}g(\tau)\sum^{\infty}_{k = -\infty}\l[a_ke^{ik\w_0\tau}e^{-ik\w_0t}\r]\,d\tau$$
$$y(t) = \sum^{\infty}_{k = -\infty}\l[\int^{t}_{0}g(\tau)e^{ik\w_0\tau}\,d\tau a_ke^{-ik\w_0t}\r]$$
Observing the long term-behaviour of the output, $t = \infty$. Therefore,
$$y(t) = \sum^{\infty}_{k = -\infty}\l[\int^{\infty}_{0}g(\tau)e^{ik\w_0\tau}\,d\tau a_ke^{-ik\w_0t}\r]$$
The term $\dst{\int^{\infty}_{0}g(\tau)e^{ik\w_0\tau}\,d\tau}$ represents a laplace transform with $s = -ik\w_0$. Therefore, 
$$G(-ik\w_0) = \int^{\infty}_{0}g(\tau)e^{ik\w_0\tau}\,d\tau$$
By substitution,
$$y(t) = \sum^{\infty}_{k = -\infty}\l[G(-ik\w_0) a_ke^{-ik\w_0t}\r]$$
For real sinusoidal inputs $\dst{u(t) = k\sin(\w t)}$,
$$\lim_{t\to\infty}[y(t)] = k|G(i\w)|\sin\{\w t + arg[G(i\w)]\}$$
\end{center}

\end{document}
