%Type of Document
\documentclass[class=report, 12pt, crop=false]{standalone}

%Load Pre-ambles
\usepackage{../Environment/Packages}
\usepackage{../Environment/Conventions}
\usepackage{../Environment/Hyahoos}

\begin{document}

\begin{center}
\chapter{Op-Amplifier Fundamentals}
\begin{comment}
\end{comment}
%Seperator
%Seperator
%Seperator
%Seperator
%Seperator
\section{Design}
\begin{comment}
\end{comment}
%Seperator
%Seperator
%Seperator
%Seperator
%Seperator
\section{Properties}
\begin{comment}
\end{comment}
Practically, the operational amplifier is a $5$-terminal device: source voltage, drain voltage, positive terminal, negative terminal, and output. Both the source voltage and drain voltage are used to power the various nmosfets and pmosfets inside the op amplifier. The ideal operational amplifier is a device with infinite gain. If the voltage at the positive terminal is greater than the voltage at the negative terminal, then the output will be positive infinity. However, practically, the highest output the operational amplifier can output is the source voltage. Therefore, if the voltage at the positive terminal is greater than the negative terminal, then the output will be the source voltage. Likewise, following the same reasoning, if the voltage at the negative terminal is greater than the positive terminal, then the output will be the drain volage. An operational amplifier could act as a comparator, but it is not recommended.
\end{center}

\end{document}
